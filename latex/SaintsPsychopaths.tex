%% Generated by Sphinx.
\def\sphinxdocclass{book}
\IfFileExists{luatex85.sty}
 {\RequirePackage{luatex85}}
 {\ifdefined\luatexversion\ifnum\luatexversion>84\relax
  \PackageError{sphinx}
  {** With this LuaTeX (\the\luatexversion),Sphinx requires luatex85.sty **}
  {** Add the LaTeX package luatex85 to your TeX installation, and try again **}
  \endinput\fi\fi}
\documentclass[a5paper,10pt,english]{book}
\ifdefined\pdfpxdimen
   \let\sphinxpxdimen\pdfpxdimen\else\newdimen\sphinxpxdimen
\fi \sphinxpxdimen=.75bp\relax
\ifdefined\pdfimageresolution
    \pdfimageresolution= \numexpr \dimexpr1in\relax/\sphinxpxdimen\relax
\fi
%% let collapsible pdf bookmarks panel have high depth per default
\PassOptionsToPackage{bookmarksdepth=5}{hyperref}
%% turn off hyperref patch of \index as sphinx.xdy xindy module takes care of
%% suitable \hyperpage mark-up, working around hyperref-xindy incompatibility
\PassOptionsToPackage{hyperindex=false}{hyperref}
%% memoir class requires extra handling
\makeatletter\@ifclassloaded{memoir}
{\ifdefined\memhyperindexfalse\memhyperindexfalse\fi}{}\makeatother

\PassOptionsToPackage{booktabs}{sphinx}
\PassOptionsToPackage{colorrows}{sphinx}

\PassOptionsToPackage{warn}{textcomp}

\catcode`^^^^00a0\active\protected\def^^^^00a0{\leavevmode\nobreak\ }
\usepackage{cmap}
\usepackage{fontspec}
\defaultfontfeatures[\rmfamily,\sffamily,\ttfamily]{}
\usepackage{amsmath,amssymb,amstext}
\usepackage{polyglossia}
\setmainlanguage{english}


\usepackage{fontspec}\setmainfont{TeX Gyre Pagella}



\usepackage{sphinx}
\sphinxsetup{
        HeaderFamily=\bfseries,
        TitleColor={rgb}{0,0,0},
        InnerLinkColor={rgb}{0,0,0},
        hmargin={1.5cm,2cm},
        vmargin={2cm,2cm},
    }
\fvset{fontsize=auto}
\usepackage{geometry}


% Include hyperref last.
\usepackage{hyperref}
% Fix anchor placement for figures with captions.
\usepackage{hypcap}% it must be loaded after hyperref.
% Set up styles of URL: it should be placed after hyperref.
\urlstyle{same}


\usepackage{sphinxmessages}
\setcounter{tocdepth}{0}


        \usepackage{emptypage}
        \usepackage{titling}
        \makeatletter
        \fancypagestyle{normal}{
          \fancyhf{}
          \fancyfoot[LE,RO]{\thepage}
          \fancyfoot[RE,LO]{}
          \fancyhead[RE]{\releasename}
          \fancyhead[LE]{\@title}
          \fancyhead[RO]{\emph{\leftmark}}
          \renewcommand{\headrulewidth}{0.4pt}
          \renewcommand{\footrulewidth}{0pt}
        }
        \fancypagestyle{plain}{
          \renewcommand{\footrulewidth}{0pt}
          \fancyhead{}
          \renewcommand{\headrulewidth}{0pt}
        }
        \makeatother
        \renewcommand{\chaptermark}[1]{\markboth{#1}{}}
        \usepackage[numbered]{bookmark}
    

\title{Saints \& Psychopaths}
\date{Oct 27, 2024}
\release{8397a50}
\author{William L. Hamilton}
\newcommand{\sphinxlogo}{\vbox{}}
\renewcommand{\releasename}{8397a50}
\makeindex
\begin{document}

\pagestyle{empty}

\makeatletter%
\hypersetup{pdfauthor={\@author}, pdftitle={\@title}}%
\makeatother%
\begin{titlepage}%
    \vspace*{\baselineskip}
    \vfill
    \hbox{%
        \hspace*{0.15\textwidth}%
        \rule{1pt}{.95\textheight}
        \hspace*{0.05\textwidth}%
        \parbox[b]{0.8\textwidth}{
            \vbox to.95\textheight{%
                \vspace{.05\textheight}
                {\noindent\Huge\bfseries Saints\\[0.5\baselineskip]
                \&\\[.5\baselineskip]
                Psychopaths}\\[4\baselineskip]
                {\Large\emph{William B. Hamilton}}\par
                \vfill % space{0.3\textheight}
                Other formats (PDF, HTML, ePub, …) available from \href{https://github.com/edhamma/saints}{github.com/edhamma/saints}.
                \\[\baselineskip]
                {\noindent This e-book is a community effort. If you spot an error in the text (such as misspelled word), go to the address above and report or (if you have the skill) fix it. Thanks!}
                \\[\baselineskip]
                {\noindent Revision \releasename, built \today.}
            }% end of vbox
        }% end of parbox
    }% end of hbox
    \vfill
\end{titlepage}

\pagestyle{plain}
\sphinxtableofcontents
\pagestyle{normal}
\phantomsection\label{\detokenize{index::doc}}


\frontmatter
\bgroup
\def\thesection{\arabic{section} }

\sphinxstepscope


\chapter{Covers}
\label{\detokenize{covers:covers}}\label{\detokenize{covers::doc}}

\section{Front}
\label{\detokenize{covers:front}}
\sphinxAtStartPar
\sphinxstyleemphasis{If you have a true saint for a teacher, then you have a real possibility for spiritual attainments, including enlightenment. If your teacher is a psychopath, then you may become a programmed puppet, and you risk being sexually or financially abused. You also may lose your job, family and possibly even your sanity. Eventually you risk disillusionment from pursuing any spiritual quests.}


\bigskip\hrule\bigskip


\begin{DUlineblock}{0em}
\item[] Dharma Audio Network Associates
\item[] San Jacinto California 1995
\end{DUlineblock}

\begin{DUlineblock}{0em}
\item[] Published by
\item[] DANA
\item[] P.O. Box 1527
\item[] Coupeville, WA 98239
\item[] Phone (800) 726\sphinxhyphen{}2421
\item[] Copyright © 1995
\end{DUlineblock}

\sphinxAtStartPar
Printed and bound in the United States of America. First edition All rights reserved.

\sphinxAtStartPar
Reviewers, may quote passages in a review without permission in writing from the publisher.

\sphinxAtStartPar
Although the author and publisher have researched all sources to ensure the accuracy and completeness of information contained in this book, we assume no responsibility for errors, inaccuracies, omissions or any other inconsistency herein. Any slights against people or organizations are unintentional. Readers should use their own judgment or consult a mental health professional when making specific evaluations of individuals.

\sphinxAtStartPar
Cover design: Brian Moucka, Poppy Graphics, Santa Barbara California.

\sphinxAtStartPar
ISBN: 0\sphinxhyphen{}9644904\sphinxhyphen{}0\sphinxhyphen{}4

\sphinxAtStartPar
\sphinxstylestrong{Organizations, Schools of Spiritual Development, and Meditation Centers}: Quantity discounts are offered for bulk purchases of this book for fund raising and educational purposes. Excerpts or special books can be created for special needs. For information contact Dharma Audio Network Associates, P.O. Box 1527, Coupeville WA 98239 or phone 800 726\sphinxhyphen{}2421


\bigskip\hrule\bigskip

\begin{quote}

\sphinxAtStartPar
Dedicated to Binky who’s support in many ways has made my work possible.
\end{quote}


\bigskip\hrule\bigskip


\sphinxAtStartPar
\sphinxstylestrong{Acknowledgements}

\sphinxAtStartPar
To Vivain Darst, John and Ann Rush, Mel Schneider, and Markell Brooks
who provided me with food and shelter while I was writing this book and
participated in proofreading it. Also helping in making corrections: Al
Reed, Janice Gale, Ken Folk, Katy Belt, Michael Freeman, Gretel Shanley,
Tamara Comstock and Bill Andreas. Special thanks to Carmen and Harold
Carlson who provided help in making the first printing possible.

\sphinxAtStartPar
Excerpts in this book from the \sphinxstyleemphasis{Dhammapada} by William L. Hamilton.

\sphinxAtStartPar
Other Books by William L. Hamilton:
\begin{itemize}
\item {} 
\sphinxAtStartPar
\sphinxstyleemphasis{Synergetic Tool Primer}

\item {} 
\sphinxAtStartPar
\sphinxstyleemphasis{Dhammapada (translation)}

\end{itemize}


\section{Back cover}
\label{\detokenize{covers:back-cover}}
\sphinxAtStartPar
There are more psychopaths pretending to be saints than there are real saints. This book will provide you with some understanding of what a psychopath is, what a saint is and how to tell the difference. It is also a sharing of what Mr. Hamilton found in his own search for inner peace and ways to develop his unrealized potential. He provides a candid view of his experiences with a psychopathic teacher and a psychopathic wife which richly illustrate the type of behavior we should all watch out for.

\sphinxAtStartPar
This book is more than just a warning to watch out for false spiritual teachers. Mr. Hamilton draws on the experience of the Snowmass Contemplative Group which is a group of advanced contemplatives from many different religious traditions. They have developed an eclectic understanding of enlightenment from all religious traditions. Mr Hamilton extends and refines this view with his knowledge of Buddhist teaching which he believes provides the most detailed understanding of enlightenment. He cites scientific studies of enlightenment which take it beyond the realm of mystical experience, and develops it into a Western psychological concept.

\sphinxAtStartPar
\sphinxstyleemphasis{Saints and Psychopaths} provides us with an inspiration as to the possibility we all have for spiritual development and self discovery. It is a highly recommended guide book for all spiritual seekers.

\begin{figure}[htbp]
\centering

\noindent\sphinxincludegraphics[width=0.200\linewidth]{{portrait}.jpg}
\end{figure}

\sphinxAtStartPar
\sphinxstyleemphasis{William L. Hamilton has been meditating since 1971 and has done more than seven years of intensive meditation practice in meditation centers and monasteries. He has studied under some of the greatest meditation masters in the world. He has been teaching Buddhist vipassana meditation since 1985.}

\sphinxAtStartPar
\sphinxstyleemphasis{In addition to teaching, he has founded audio cassette production facilities for several nonprofit organizations including the Hanuman Foundation, Insight Meditation Society, and Insight Recordings. He now spends most of his time writing, teaching and developing a meditation retreat center.}

\begin{DUlineblock}{0em}
\item[] DANA
\item[] P.O. Box 1527 Coupeville, WA 98239
\end{DUlineblock}

\sphinxAtStartPar
\sphinxstylestrong{ISBN: 0\sphinxhyphen{}9644904\sphinxhyphen{}0\sphinxhyphen{}4}

\sphinxstepscope


\chapter{Preface}
\label{\detokenize{preface:preface}}\label{\detokenize{preface::doc}}
\setcounter{section}{0}

\sphinxAtStartPar
There are more psychopaths pretending to be saints than there are real
saints. This book will provide you with some understanding of what a
psychopath is, what a saint is and how to tell the difference. It is
also a sharing of what I have found in my search for inner\sphinxhyphen{}peace and for
ways to develop my unrealized potential.

\sphinxAtStartPar
Some paths I followed were more like stepping stones which went part of
the way, but became obstacles to further progress. Some of these paths
were offered by sincere teachers, and others were paths of entrapment
offered by psychopaths. I learned something from all my spiritual
teachers, if only that some teachers and teachings should be avoided. I
hope that sharing my experiences will provide you with some guidelines
for avoiding similar mistakes, while encouraging you on your own
spiritual quest.


\section{My spiritual journey}
\label{\detokenize{preface:my-spiritual-journey}}
\sphinxAtStartPar
My spiritual journey began in 1971. I was then the president of a
national franchise company after having a meteoric rise from being a
stock broker in the early 1960’s. I purchased a small business from a
customer, who was dying of cancer in 1964 and parlayed it into several
other small businesses. In 1965 I resigned as a stock broker to devote
all my time to my growing business empire. Everyone marveled at how
successful I had become, and it seemed assured that I had a great
future.

\sphinxAtStartPar
In 1971 I was given some LSD and some good advice on how to make use of
it. I was told that I should make a firm decision not to go outside my
apartment or answer my phone for 12 hours after I had taken the drug. It
was suggested that I do some inspirational reading and listen to
inspirational music for the hour or so it would take for full effect

\sphinxAtStartPar
I neither recommend nor discourage people from experimenting with
psychedelic drugs. Although they rarely mention it, many of the best
Western spiritual teachers today began their spiritual evolution with
psychedelics. Most people who try them simply have pleasant, unpleasant
or confusing experiences. Although only a minority of people who
experimented with psychedelics had beneficial spiritual experiences,
there were millions who experimented with them from the mid 1960’s to
mid 1970’s. During this time, most of the people who became involved
with meditation or who took spiritual journeys to Asia did so as a
result of their experiences with psychedelics. This is rarely the case
today. It seems that the drug experiences only showed that there were
possibilities of altered states of consciousness. Today we have the
results of the quests for inspiration.

\sphinxAtStartPar
My first experience with LSD was profound. During it, I had two insights
that changed the course of my life. One insight was a transcendent
experience of time and space, during which I briefly touched the stream
of my unconscious mind. This was the basic seed that inspired my
spiritual journey.

\sphinxAtStartPar
The other insight was an awareness that, as I had become outwardly more
successful, I was becoming inwardly more uptight and unhappy. Suddenly I
saw that all my personal relationships revolved around money. People
associated with me because they expected me to invest in schemes, lend,
or give them money. I could offer jobs, buy services or merchandise and
provide money making opportunities. I saw that I had cultivated
unsatisfactory motives in people around me and that I lacked truly
wholesome personal relationships.

\sphinxAtStartPar
At the time that I came to these realizations, I was in the middle of
developing a public offering of my company’s stock. If I had continued
with this process for another three to five years, I could have made a
twelve\sphinxhyphen{}million\sphinxhyphen{}dollar profit on my stock. However, it now seemed to me
that making such a large amount of money would sabotage the type of
relationship I would like to have with people. I soon began a process of
divesting myself of business responsibilities. Within five months, I had
found a new president for my company and set out on a full time
spiritual quest.

\sphinxAtStartPar
I founded the Orphalese Foundation and gathered a group consisting
mostly of psychologists who were interested in creating an intentional
community dedicated to developing human potential. We did not feel that
we had all the answers, but we felt confident that we could find and
share answers. We had a 30\sphinxhyphen{}room house in Denver and a 123\sphinxhyphen{}acre retreat
in the mountains. We explored a wide variety of practices from yoga and
meditation to theories of human potential and cosmic consciousness. As
with the vast majority of intentional community experiments popular at
that time, our group became the victim of greed, aversion and delusion.
We could say the same words as to how the goals of our community were
also our individual goals, but the words had selfish individual
meanings. After two years our group disintegrated, and we went our
separate ways.


\section{Ram Dass}
\label{\detokenize{preface:ram-dass}}
\sphinxAtStartPar
During the final phase of our groups’ disintegration, I discovered the
book \sphinxstyleemphasis{Be Here Now} by Ram Dass. I was particularly inspired by the
audio tapes of his talks. I could identify with his stories of
experiments with LSD at Harvard in the early sixties when he was Dr.
Richard Alpert working with Dr. Timothy Leary. At first they thought
they had discovered a way to attain enlightenment, but after five years
Dr. Alpert decided that drugs were a dead end. After abandoning drugs as
a path, he went to India and found his guru who gave him the name \sphinxstyleemphasis{Baba
Ram Dass}. He felt that his guru offered the answer to what he said
was, “whatever it was I was looking for.”

\sphinxAtStartPar
Although Ram Dass did not claim to be enlightened, he did describe
himself as a seeker offering what he found on the path to enlightenment.
Since he said that Westerners were generally not welcome in the place he
had found in India, I began a daily routine of the various practices he
had described in his books and tapes.

\sphinxAtStartPar
The practices were primarily those of Raja Yoga—a combination of
traditional Hindu practices such as study, reflection, breathing
exercises, hatha yoga, morality and meditation, which could result in a
profound transformation. I was quite impressed with the results I was
getting from these methods, and it seemed that I was on the right track.
As the community of the Orphalese Foundation was disintegrating, I was
becoming more involved with Ram Dass and Raja Yoga.


\section{Distributing audio tapes}
\label{\detokenize{preface:distributing-audio-tapes}}
\sphinxAtStartPar
I wrote to Ram Dass about the high cost of his tapes limiting
their availability to many people. I had a meeting with him and ended up
being responsible for the distribution of his tapes through the
Orphalese Foundation. Later, when the Hanuman Foundation was formed in
1974, the tapes were turned over to this new organization and I
continued to be manager of this project. My work with the tapes and Ram
Dass provided many contacts that became sources for this book.

\sphinxAtStartPar
I was raised as a Christian, but Christianity did not offer all the
answers that I was seeking. Ram Dass had been raised as a Jew, but he
went beyond Judaism because it was not providing the answers he sought.
While I was with Ram Dass, I experienced an eclectic smorgasbord of
spiritual traditions and techniques. Ultimately, I found that among the
Eastern traditions, the Buddhist meditation he introduced me to is the
most beneficial form of practice and, the Buddhist Philosophy is most in
tune with Western culture.


\section{Buddhist vipassana meditation}
\label{\detokenize{preface:buddhist-vipassana-meditation}}
\sphinxAtStartPar
After I left Ram Dass and the Hanuman Foundation in 1977, I gradually
increased my emphasis on Buddhist \sphinxstyleemphasis{Vipassana} Meditation. In 1980 I
did my first three\sphinxhyphen{}month retreat at the Insight Meditation Society (IMS)
in Barre, Massachusetts. The teachers there were Western laypeople who
had spent years studying Buddhism and practicing meditation in Asia.
They intended to establish a new Western style of Buddhism almost
without rites, rituals, or dogma, which emphasized the practice of
meditation as its main feature. The method of meditation they taught was
developed by Mahasi Sayadaw, who was the preeminent Buddhist meditation
teacher in Southeast Asia.

\sphinxAtStartPar
I was inspired enough by the results of my three\sphinxhyphen{}month retreat to
continue this practice for two years as a hermit in a remote area on
Maui, Hawaii, Although I was in a remote undeveloped area, soon after I
started my retreat, a stupa of enlightenment was built on the land next
to my retreat. The stupa brought me into contact with Tibetan Buddhists,
but my practice continued in the Theravada Buddhist tradition. I gained
a deep appreciation for the Tibetan tradition during this two\sphinxhyphen{}year
retreat.

\sphinxAtStartPar
People who made pilgrimages to the stupa would learn that there was a
Buddhist meditator nearby, and when I was not doing intensive practice,
I accepted visitors. Gradually, I was becoming established as a teacher,
but I felt that I had much to learn. I could have continued my retreat
indefinitely as a caretaker of the land that I was on, but I felt that I
should go to Asia to learn more about teaching.

\sphinxAtStartPar
After two years, I returned to IMS intending to do six months of
volunteer staff work before going on to Burma. It turned out that I
could not get into Burma because of the political turmoil there. IMS was
also going through a painful period of crisis because the social
structure of the staff was virtually dysfunctional. Because of my
background in tape recordings, a unique role of being quasi\sphinxhyphen{}staff,
quasi\sphinxhyphen{}meditator was created for me. I founded the IMS Tape Library that
later evolved into the Dharma Seed Tape Library. I had my own private
office at IMS in which I could work quietly and meditatively, mostly
isolated from the turmoil of the regular staff. I also did five hours of
intensive meditation every day.

\sphinxAtStartPar
After managing the IMS Tape library for a year, the Board of Directors
gave me a one year full meditation scholarship in 1984. It was a good
time to be on retreat at IMS because four great Asian meditation masters
taught there that year: Anagarika Munindra, Dipa Ma Barua, Tungpulu
Sayadaw and Sayadaw U Pandita. I had heard endless stories about
Munindra and Dipa Ma as they had been primary teachers of most of my
Western teachers. Tungpulu Sayadaw was widely regarded as being fully
enlightened. U Pandita was the successor to the late Mahasi Sayadaw, and
was regarded as the leading authority on this method of practice. U
Pandita turned out to be an incredibly powerful teacher, and has been
the greatest influence in my developing an advanced meditation practice.

\sphinxAtStartPar
After the full IMS scholarship expired, I was able to continue my
practice with a partial scholarship from IMS, grants from a private
foundation and gifts from several friends. Finally, after being on
retreat for almost five years, I decided to return to the real world. I
spent eighteen months traveling all over the United States doing odd
jobs and teaching meditation. During this period, my teacher Sayadaw U
Pandita taught a ten\sphinxhyphen{}day retreat in California, which I attended.

\sphinxAtStartPar
I thought that I had learned the lesson that there is no particular time
standard for how the practice unfolds. I expected that the purpose of a
trifling ten\sphinxhyphen{}day retreat in 1986 was to renew old acquaintances and
brush up on my meditation practice. I was surprised that this ten\sphinxhyphen{}day
retreat turned out to be one of the pivotal experiences of my life. As
soon as I could fulfill my teaching commitments after this retreat, I
returned to IMS for another year and a half of intensive practice.


\section{Peace Pilgrim}
\label{\detokenize{preface:peace-pilgrim}}
\sphinxAtStartPar
During this retreat, a Tibetan monk visiting IMS did a reading from the
Peace Pilgrim book. After reading the book, I became convinced that
Peace Pilgrim was a rare case of spontaneous enlightenment that Buddhist
texts refer to. She seemed to conform to the Buddhist concept about
these cases: she was an inspiring teacher, but she lacked a complete
methodology for guiding others to the same attainment she had made.
Peace Pilgrim died in 1981, and some followers had compiled the Peace
Pilgrim book from transcripts of her talks, newsletters, and letters.

\sphinxAtStartPar
What particularly inspires me about the Friends of Peace Pilgrim, the
nonprofit organization distributing her book, was that they give books,
audio and video tapes away free. Anyone writing to Friends of Peace
Pilgrim, 43480 Cedar Ave., Hemet, CA 92544 and requesting the Peace
Pilgrim book will receive a free copy. They rely only on unsolicited
donations to continue the distribution of Peace Pilgrim’s message. Peace
Pilgrim believed that spiritual teachings should never be sold. This is
an ideal in the Buddhist tradition followed in Asia, but no one had been
successful with free distribution in the West. For years I had been
trying, without success, to figure out how to distribute audio tapes for
free.

\sphinxAtStartPar
At the end of my retreat in 1988, I went to Hemet, California to do one
year of volunteer work for the Friends of Peace Pilgrim. I did this
partly to support the teachings of Peace Pilgrim, but also to learn how
to do free distribution of spiritual teachings. I learned that it could
be done when free facilities are provided and the volunteers have
independent incomes.


\section{Insight Recordings}
\label{\detokenize{preface:insight-recordings}}
\sphinxAtStartPar
In 1989 I returned to teaching vipassana meditation in different parts
of the United States. In 1990 I gave up teaching in order to found
Insight Recordings that distributed audio tapes of Buddhist teachers. I
took a one year sabbatical in 1993 to do volunteer work for the
Vipassana Support Institute and to do a period of intensive practice
with Sayadaw U Pandita in Burma. The retreat in Burma brought my total
time spent in intensive meditation retreat to over seven and a half
years. When I returned from my sabbatical, I found Insight Recordings in
a dysfunctional state. Since I felt that the best service I could do was
to write, I shut down Insight Recordings and put its equipment in
storage.


\section{My life is my message}
\label{\detokenize{preface:my-life-is-my-message}}
\sphinxAtStartPar
I have included the details from my personal melodrama in this preface
so that you can have some understanding of the bias of my view. Also, I
have been inspired by a quotation from Mahatma Gandhi. Once when Gandhi
was boarding a train, a reporter asked Gandhi to give a message to the
people of his city. Gandhi replied, “My life is my message”. In this
book I have intertwined the message of my life with the messages of many
different religious traditions.

\sphinxAtStartPar
I find that identifying myself precisely in religious terms is difficult
and paradoxical. I consider myself a Christian who finds great value and
truth in the teachings of Jesus. What we know of Jesus and his teachings
gives me a strong suspicion that he was enlightened. I see strong
evidence that contemplative Christian practices have resulted in a few
people attaining enlightenment, and this reaffirms my Christian
tradition. I separate with Christians who believe that an experience or
rites and rituals can assure eternal salvation. I share the Buddha’s
view that eternal salvation comes only from purifying our consciousness
from greed, hatred and delusion. I see strains of purity and true
understanding in Christianity that I can identify with.

\sphinxAtStartPar
At the same time, I have difficulty identifying with many aspects of the
Buddhist tradition. Enlightenment has been defined in part as a
disbelief in rites and rituals, but Buddhism has evolved a plethora of
them. Buddhists believe in previous incarnations of the Buddha where
moral values are exemplified, in my view, in improbable fairy tales. One
of my teachers believed I should not have a buddha statue in my room
because the buddha statue would see dirty, filthy parts of my body if I
undressed in front of it. Buddhists can be fundamental and dogmatic in
their beliefs and negative about other Buddhist traditions and other
religions. Often good Buddhists believe that Buddhism (and possibly only
their particular tradition) has a monopoly on enlightenment

\sphinxAtStartPar
Despite these reservations, I find difficulty disassociating myself from
the Buddhists. Enlightenment is very clearly of central importance in
Buddhism. Logic and personal experiences are officially stated by the
Buddha as being more important than dogmatic belief, scriptures or any
authority. At his death, the Buddha refused to appoint a successor
saying that everyone should be a light unto themselves. The Buddhists,
by far, have the most extensive and systematic understanding of what
enlightenment is and how to attain it.

\sphinxAtStartPar
Joseph Campbell once said, “God is like a computer and religions are
like programs.” “All the programs work” He discretely did not mention
that, as with computer programs, some work much more efficiently than
others. Also, different programs are intended to do different things.
Although I don’t particularly think of myself as a Buddhist, I have
noticed that when I explain concepts to people, I invariably use the
Buddhist program. Today, when I seek teachings and guidance in
meditation, I look for a Buddhist teacher. I am using the WordPerfect
computer program to write this. I must be a Buddhist as well as a
WordPerfectist.


\section{Enlightenment}
\label{\detokenize{preface:enlightenment}}
\sphinxAtStartPar
In this book, I take an eclectic view of enlightenment from many
different religious traditions. However, my area of expertise is in the
Buddhist tradition, and when I describe details of a saint’s evolution I
use the Buddhist model. A couple of the people who read earlier drafts
of this book completely missed the point that when I discuss the process
of evolution in the meditation practice, I am describing the Buddhist
view of the evolution of a saint. Many people who are qualified to judge
the relationship between Buddhism and Christianity have concluded that
there is an essential relationship between enlightenment and saints.

\sphinxAtStartPar
Enlightenment is emerging from being a vague experience of Eastern
mysticism. It is becoming a scientifically verified, quantified and
qualified experience as it enters Western culture. In the past, when
Buddhism entered new cultures, enlightenment eventually became the
highest ideal of the culture, and perhaps even a fad. It could happen
again.

\sphinxAtStartPar
If it does, there will be an abundance of false teachings and teachers
that will go along with it. This is why I chose to write on the subject
of Saints and Psychopaths. I want to share my experiences to help others
avoid the mistakes I made. Also, I want to make a clear statement, in
Western terms, as to what enlightenment is, in order to help people
determine which teachers and teachings are leading to freedom and which
are leading to slavery.

\egroup
\mainmatter
% promote sections for the main text
% (unlike in frontmatter and appendix)
\let\subsubsection\subsection
\let\subsection\section
\let\section\chapter
\let\chapter\part

\sphinxstepscope


\chapter{Part I: Psychopaths}
\label{\detokenize{psychopaths:part-i-psychopaths}}\label{\detokenize{psychopaths::doc}}

\section{Psychopaths}
\label{\detokenize{psychopaths:psychopaths}}
\sphinxAtStartPar
In the summer of 1974, Ram Dass had just finished teaching a six\sphinxhyphen{}week
course to 1,300 students at Naropa Institute in Boulder, Colorado. His
guru, Neem Karoli Baba Maharajji, had died a year before, and he was
feeling somewhat depressed because he had not found a new teacher.
Although many thousands of people looked up to him as their spiritual
teacher, he knew that he was not enlightened, and he longed for a
teacher he could trust. He had decided that he was going to go to India
that winter to search for a suitable teacher.

\sphinxAtStartPar
Hilda Charlton, an eclectic spiritual teacher in New York City, took Ram
Dass to see a woman she felt he should meet. He came into a room where a
woman was sitting, apparently in deep samadhi. He was invited to verify
her trance state by putting a mirror under her nose and trying various
means to get her attention. It appeared to him that the trance was real.
He then sat in the room for awhile meditating with her.

\sphinxAtStartPar
Abruptly she came out of her trance, and then appeared to channel
Maharajji to him. The information being channeled seemed to be kinds of
things Maharajji would discuss and things that Ram Dass had never
mentioned to anyone. He was impressed, but at the same time he felt
repelled by her excessive makeup, jewelry and vile language. It was
explained to him that the woman was an incarnation of the Hindu deity
Kali. Kali is a manifestation of the Divine Mother, who purifies people
by appearing very frightening. This seemed to be a legitimate
explanation for feeling uneasy about her.

\sphinxAtStartPar
After vacillating for several weeks, Ram Dass decided to commit himself
to taking teachings from this woman. Although the teacher and teachings
he was receiving were supposed to be a secret, word spread among the
Western devotees of Maharajji, and soon New York City became an unlikely
Mecca for dozens of them who came to attend secret classes with this
teacher.

\sphinxAtStartPar
Later, Ram Dass wrote an exposé of his two year involvement with this
teacher. His description of the complicated web of lies, deception,
sexual misconduct and drug use by his teacher portrays a classic example
of a psychopath pretending to be a saint.


\subsection{There are more psychopaths than saints}
\label{\detokenize{psychopaths:there-are-more-psychopaths-than-saints}}
\sphinxAtStartPar
Psychopaths pretending to be saints present a very serious problem for
all spiritual traditions. There are many more psychopaths pretending to
be saints than there are real saints. If you have a true saint for a
teacher, then you have a real possibility for spiritual attainments,
including enlightenment. If your teacher is a psychopath, then you may
become a programmed puppet, and you risk being , sexually or financially
abused. You also may lose your job, your family and possibly even your
sanity. Eventually you risk disillusionment in the pursuit of any
spiritual quests.

\sphinxAtStartPar
For the purposes of this book I define a saint as any true spiritual
seeker who, through a process of study, discipline, prayer, or
meditation has attained a purification of mind and true spiritual
understanding. In the Buddhist tradition a saint would be fully
enlightened, although a legitimate teacher would be one who has attained
at least the first of four levels of enlightenment.

\sphinxAtStartPar
A psychopath is someone who is morally defective and does not respect
the values of property, truth and proper consideration for the effect of
actions on self and others. Generally mental health professionals do not
regard psychopaths as mentally ill because they do not manifest obvious
dysfunctional behavior, but they appear to be rational. Most
professionals prefer the terms \sphinxstyleemphasis{sociopaths}, \sphinxstyleemphasis{borderline personalities},
or \sphinxstyleemphasis{antisocials.}

\sphinxAtStartPar
Perhaps it is because my degree in psychology dates back to 1959 that I
prefer the old fashioned term \sphinxstyleemphasis{psychopath}. I am doubtful that
changing the name for each current vogue in professional understanding
contributes to the public’s understanding of this very important issue.
Also, my direct personal experience with psychopaths has reinforced the
view that psychopaths are indeed mentally ill, even if the signs are not
immediately obvious.


\subsection{The origins of psychopathy}
\label{\detokenize{psychopaths:the-origins-of-psychopathy}}
\sphinxAtStartPar
The origins of psychopathy occur in early childhood when prolonged
periods of feeling the unbearable pain of being unloved are experienced.
Actually psychopaths may have been loved, but their parents’ problems
with marriage, career, health, drinking, drugs, travel, etc. may have
kept them from adequately expressing it. When the child had problems, he
or she felt that there was no one to whom to turn for support, guidance
and love. Most children who have this type of experience simply become
neurotic, but others experience a more sinister development: As the
stress builds, they feel that everyone and everything is the source of
their suffering. They reach a breaking point and make a conscious
decision to get even with the world. From that point on they feel that
any harm they cause others is justified revenge. They become juvenile
delinquents, and by the time they become young adults their pattern of
behavior becomes so deeply rooted that they are virtually incurable.

\sphinxAtStartPar
There are two general types of psychopaths. One type is overtly violent,
and most of them quickly end up in prison for murder or a series of
violent crimes. The other is a covert type that is actually much more
dangerous and can cause both violent and nonviolent suffering to large
numbers of people. There are no limits to the amount of damage a
psychopath can do. Hitler, Stalin, and Saddham Hussein are examples of
psychopaths who did great damage when they seized political control.

\sphinxAtStartPar
You should understand that being a psychopath is not a black or white
situation, but is measured on a gradient scale, and we all have some
element of a psychopath in our personality. It is only when this
characteristic is strong enough to dominate the personality that the
label \sphinxstyleemphasis{psychopath} should be used. Even when this occurs some people
are only slightly psychopathic and others are very psychopathic.

\sphinxAtStartPar
Very psychopathic people rarely remain long amid the byproducts of their
actions even if they are the covert type. They either end up in prison,
or are constantly on the move from place to place. They rarely acquire
an advanced education, establish themselves in a career, or become
recognized as useful persons in society.

\sphinxAtStartPar
However, slightly to moderately psychopathic persons can become
established in a community, and covert psychopaths can have
extraordinarily attractive and charismatic personalities. The
entertainment business and advertising seem to attract a
disproportionately large number of psychopaths perhaps because both
professions involve creating illusions. They may be doctors, perhaps
with fake degrees, who frequently are charged with malpractice. They may
be lawyers who become deeply , involved with criminals and scam artists.
They may be politicians who take bribes and abuse the powers of office.
They may be psychotherapists who seduce or enslave their clients. They
may be business people who sell shoddy merchandise, inflate repairs and
do not honor guarantees. They may be religious leaders pretending to be
saints.


\subsection{Distinguishing saints from psychopaths}
\label{\detokenize{psychopaths:distinguishing-saints-from-psychopaths}}
\sphinxAtStartPar
Distinguishing a saint from a psychopath presents a unique problem
because they have some common characteristics that seem at first to be
identical. Both saints and psychopaths can have the appearance of a
beautiful, radiant and attractive being. Both may tell you, ‘Be here
now, forget the past, forget the future; be spontaneous, heed your inner
voice, follow your bliss.” Both may advise you to not be bound by
traditional social values but by higher spiritual values. Both may have
messages from God or spiritual teachings tailored just for \sphinxstylestrong{you}.
Both may be homeless wanderers. Both may manifest fearless behavior
and may risk persecution. Saints and psychopaths can be intuitively
perceptive of people’s mood changes, new developments, and new
understandings. They may appear to manifest similar psychic powers,
healing, mind reading, and channeling from other realms.

\sphinxAtStartPar
Although the powers of a saint and a psychopath may seem the same at
first, they have different roots. Saints have a calm, clear, empowered
state of mind as a result of discipline, meditation, and introspection.
Psychopaths can develop \sphinxstyleemphasis{paranoid samadhi}, which is a concentrated
mind, because they have done so many unskillful things such as lying,
theft, injury, adultery, substance abuse, etc. Their powers come from
having to have a very sensitive awareness to perceive when someone is
coming after them. They are also gluttons for attention, and when they
have your attention they will start to feed on your spiritual energies
like a psychic vampire. They can sometimes read minds, tell the future,
do healings, see things which aren’t physically apparent and you may
become mesmerized and convinced of their divine power.

\sphinxAtStartPar
So how do we tell saints from psychopaths? My teacher, Sayadaw U
Pandita, says that he never makes up his mind about peoples
enlightenment until he has known them and observed them closely for a
year. It is in the nature of saints to respond to sincere requests for
help, and guidance. If you sincerely want help they will be there for
you. They may ask you to make commitments once you are training under
their guidance, but there is unlikely to be an initial urgent
commitment. Psychopaths, on the other hand, are more likely to come on
to you with an initial urgency, demanding that you make a commitment
immediately or lose your opportunity. Therefore, my first advice about
telling saints from psychopaths is to \sphinxstylestrong{take your time.}


\subsection{Amoral or immoral?}
\label{\detokenize{psychopaths:amoral-or-immoral}}
\sphinxAtStartPar
In time some very distinguishing differences between saints and
psychopaths become apparent. Saints have such a deeply rooted morality
from their own direct understanding that by normal social standards they
may be amoral. The Buddha clashed with his culture by disparaging rites
and rituals and not respecting caste. Christ, too, conflicted with his
culture.

\sphinxAtStartPar
Psychopaths, on the other hand, are simply immoral. Their divergence
from social standards involves self gratification and disregard for
doing harm. At first it may be difficult to discern whether a teacher is
amoral or immoral, but in time it may become apparent whether or not he
or she adheres to the standards of behavior being taught. The situation
that Ram Dass found himself in was that of having a teacher who insisted
that everyone tell the truth, but she herself constantly lied. She
forbade the use of drugs, but used them habitually. She insisted on
celibacy for her students, but practiced adultery.

\sphinxAtStartPar
I use a standard of evaluation I call SAY, MEAN, DO. Saints will say
what they mean and will do what they say. Psychopaths will mean
something other than what they say and what they do may have little
relationship to what they say and mean. For example, psychopaths may say
they love you or want to help you, when what they mean is that they want
attention or money. What they do in the long run is going to be a
disappointment. It takes a while for consistency or inconsistency of
SAY, MEAN, DO to come into focus. The more time you take in evaluating
this the more accurate your conclusion will be.


\subsection{Seeds of destruction}
\label{\detokenize{psychopaths:seeds-of-destruction}}
\sphinxAtStartPar
Psychopaths are constantly planting the seeds of their own destruction,
so it is good to look and listen carefully for this. When new
psychopaths arrive on the scene, they or their co\sphinxhyphen{}psychopathic entourage
will tell you many stories of how successful and well respected they
were at their previous locations. However, in time they or other people
will begin to tell stories of great conflict and discord at their
previous places. Listen carefully to these stories and you will hear
that they \sphinxstyleemphasis{were} at the center of these problems. Listen more
carefully and you will see that they were the cause of these problems.


\subsection{The Big Lie}
\label{\detokenize{psychopaths:the-big-lie}}
\sphinxAtStartPar
Psychopaths frequently make use of the Big Lie method, so you should
critically evaluate the plausibility of the claims people make. You
should be very suspicious when someone claims that 98\% of cancers were
cured, or 99\% of the marriages they arranged were successful, or 100\% of
their students become enlightened. Almost all of such claims are made by
psychopaths, especially if they repeat such claims over and over again.


\subsection{Spiritual scenes}
\label{\detokenize{psychopaths:spiritual-scenes}}
\sphinxAtStartPar
Spiritual scenes are fertile feeding grounds for psychopaths. There are
always new members and teachers arriving and there is a bias to welcome
and accept them as being wonderful. Psychopaths thrive on not having to
verify what they say. They frequently get away with claiming to be
enlightened, the reincarnation of some deity, delivering messages from
God, or having special spiritual powers.


\subsection{The chorus of psychopaths}
\label{\detokenize{psychopaths:the-chorus-of-psychopaths}}
\sphinxAtStartPar
It is common for psychopaths to pick up ideas from each other.
Channeling is a classic example. From time to time there is a chorus of
psychopaths proclaiming that California is about to fall into the ocean,
or that Earth will be hit by a comet next August. When the disaster
doesn’t happen they will say that their prayers prevented it. How do you
immediately check out the validity of channeled information? Generally
speaking you can’t, other than reflecting on the plausibility of their
claims and waiting to see what happens.


\subsection{More claims than comets}
\label{\detokenize{psychopaths:more-claims-than-comets}}
\sphinxAtStartPar
Of course, you may miss out on the excitement of being a close disciple
of the next avatar or messiah. You may drown when California sinks.
Remember that avatars come every thousand years or so and a comet hits
Earth every 100,000 years despite the proliferation of claims. In the
last two thousand years there have been tens of thousands of predictions
of drastic earth changes and not one has come to pass. You may feel
attracted to a radiant being who is making unlikely claims and offering
a shortcut to spiritual development. If you want a short cut to
spiritual development, consider Russian roulette. The odds are better
and the chance of suffering from a mistake is less.


\subsection{The pattern of avoiding punishment}
\label{\detokenize{psychopaths:the-pattern-of-avoiding-punishment}}
\sphinxAtStartPar
Psychopaths skillfully evade blame when they are confronted with having
done something wrong. Since they lack a true sense of guilt, they do not
respond the way you may expect a guilty person to behave. Psychopaths
have a very distinctive sequence of responses to dealing with
confrontations. If one method of stopping a confrontation does not work,
they will change strategies. When confronted with wrong\sphinxhyphen{}doing, a
psychopath will respond in this sequence:
\begin{enumerate}
\sphinxsetlistlabels{\arabic}{enumi}{enumii}{}{)}%
\item {} 
\sphinxAtStartPar
Ignore the issue.

\item {} 
\sphinxAtStartPar
Deny that they have done something wrong.

\item {} 
\sphinxAtStartPar
Attack the accuser, usually accusing the accuser of being the one
who has done wrong.

\item {} 
\sphinxAtStartPar
Threaten to harm the accuser, someone else, something, or self.

\item {} 
\sphinxAtStartPar
Apologize and admit that they have done wrong, then ask for a clean
slate or new start

\end{enumerate}

\sphinxAtStartPar
A saint, on the other hand, will either immediately admit that he or she
has made an error, or ask for clarification and seek reconciliation. An
example would be Christ’s advice that when someone asks for your coat
you should give your cloak also. Generally, saints will place a higher
value on harmonious relationships than on pride or possessions. We must
allow for cultural factors and personality characteristics, but when
confronted with wrong\sphinxhyphen{}doing, saints generally will follow this sequence:
\begin{enumerate}
\sphinxsetlistlabels{\arabic}{enumi}{enumii}{}{)}%
\item {} 
\sphinxAtStartPar
Acknowledge errors and misunderstandings

\item {} 
\sphinxAtStartPar
Admit that they have made an error

\item {} 
\sphinxAtStartPar
Apologize

\item {} 
\sphinxAtStartPar
Offer compensation or correction

\item {} 
\sphinxAtStartPar
Avoid that type of error in the future

\end{enumerate}

\sphinxAtStartPar
The first strategy of the saint is the last strategy of a psychopath.
But when psychopaths are finally forced to apologize they will outdo the
saints. Their previous belligerent attitudes will vanish. They will
apologize profusely and confess the error of their ways in great detail.
They may even list wrongdoings that you were unaware of, to impress you
with the depth of their change. Their transformation seems quite
impressive and even professionals who should know better are sometimes
taken in by their pretense. Judges have suspended sentences of repeat
bigamists and outrageous con artists who swore to devote the rest of
their lives to making restitution.

\sphinxAtStartPar
“Give me a clean slate”, is the refrain of psychopaths. They will proclaim
that they are a new person or that they have been born again. Sometimes
they insist that they should not be punished because the person who did
those things no longer exists. Indeed they may make drastic changes in
their behavior, from being rude and domineering to being humble and
submissive. It is, however, all a ruse to get off the hook. For awhile
after being caught psychopaths may go through a quiescent period, but in
time the same old patterns of behavior will reoccur. They are not bound
by conscience or true remorse. As soon as you walk out the door they may
revert to their old ways without skipping a beat.


\subsection{Psychopaths are self\sphinxhyphen{}destructive}
\label{\detokenize{psychopaths:psychopaths-are-self-destructive}}
\sphinxAtStartPar
You should remember that the essential characteristic of psychopaths is
that they are self\sphinxhyphen{}destructive and destructive of those around them.
Sooner or later things are going to turn out bad. Having a psychopath
around is like having a pet rattlesnake running loose in your house.
When you determine that someone is psychopathic, you should make an
immediate clean break with them.

\sphinxAtStartPar
Threatening a psychopath is like waving a red flag at a bull. Our
legislators should be aware of this when they pass laws. Laws which
motivate normal people to avoid crime may result in psychopaths
committing more crime. Psychopaths are essentially self\sphinxhyphen{}destructive and
so to threaten psychopaths with destruction if they break a law only
increases their motivation. All too often our system of punishment
results in sending psychopaths to prison which becomes a school for
learning psychopathic tricks. Associations made in prison result in the
establishment of a more dangerous network of psychopaths in society
after they leave prison.


\subsection{Psychopaths crave attention}
\label{\detokenize{psychopaths:psychopaths-crave-attention}}
\sphinxAtStartPar
As a result of not having received love and attention as a child,
psychopaths have an almost unlimited need for attention. One of the
signs of a psychopath is that wherever they go they tend to become the
center of attention. It doesn’t matter to them whether they do something
good or bad as long as they get attention. They can be benefactors as
easily as they can be dangerous and may steal things to give to someone
else. At parties they become the focal point of jokes, speeches, pranks,
story telling, arguments, fights, singing, dancing and intense activity
where they are the center of attention. They are skillful, talented,
entertaining, argumentative and accident prone. At lectures and meetings
they become the center of attention by asking many questions, making
statements about how they oppose something, or how they applaud and
approve of what is being said. If they can’t get attention in a group,
they will usually be doing lots of fidgeting in their seat, like O.J.
Simpson did at his trial. If they no longer can get your attention by
dominating you, then they can shift to seeking your assistance in
helping them rehabilitate themselves or recover from an accident.


\subsection{Motivating with guilt}
\label{\detokenize{psychopaths:motivating-with-guilt}}
\sphinxAtStartPar
Another sign to watch for is that psychopaths tend to motivate you with
guilt. Anything you do wrong becomes a lever for manipulating you. This
is particularly true if you break, or threaten to break a promise, even
though they usually have poor records in keeping promises.


\subsection{Anxiety attacks}
\label{\detokenize{psychopaths:anxiety-attacks}}
\sphinxAtStartPar
There is one sign of a psychopath that usually only a close associate
will have an opportunity to see. From time to time a psychopath will
have anxiety attacks. They hide alone, or with someone totally under
their control, when they become panicked about their health, fear of
being arrested, assassinated, or attacked by devils, spirits, etc.
Sometimes anxiety attacks last days or weeks, or sometimes only brief
moments, especially if they get their co\sphinxhyphen{}psychopath motivated to do
something for them as a result.


\subsection{Organizations}
\label{\detokenize{psychopaths:organizations}}
\sphinxAtStartPar
Most organizations, especially spiritual organizations, tend to make
rules of various types as a result of their encounters with psychopaths.
This can be useful in some circumstances, but usually the result is
complexity and inconvenience for the members of the organization. The
psychopaths will simply break the rule if they can get away with it, or
do something else which is equally bad but not against the rules.
Organizations would be better advised to develop an awareness of
psychopaths and establish a system for getting rid of them.

\sphinxAtStartPar
Too often the initial impression that a psychopath makes on people is
very positive. This is especially true in spiritual organizations.
Sometimes psychopaths have an attractive, radiant appearance. They stand
out in a group and people are likely to feel especially drawn to them.
An organization would be well advised to be extra cautious about
becoming involved with any unusually attractive or impressive newcomer.

\sphinxAtStartPar
Another thing to watch for in spotting psychopaths is that a group of
people is likely to have a polarized mix in their response to a
psychopath. Quite likely some people are going to feel extraordinarily
drawn to a particular newcomer, and others will have a strong negative
reaction to him or her. Typically, in time, everyone in the group is
either going to develop a strong liking or disliking for a psychopath.

\sphinxAtStartPar
Psychopaths are dangerous even in legitimate organizations with honest
leadership. If a psychopath comes on the scene bad things are bound to
happen. Businesses become inoperable, teams become disorganized,
families break up. Psychopaths are likely to be trouble makers,
embezzlers, drug dealers, or get the organization involved with illegal
dealings. The morale of the organization is likely to deteriorate, and
the staff is likely to become divided into warring camps.


\subsection{Buddhism}
\label{\detokenize{psychopaths:buddhism}}
\sphinxAtStartPar
Buddhism has fewer psychopaths than other major religious traditions.
This is partly because Buddhists have a clearer idea of what
enlightenment is, and leaders are more likely to spot someone who is
pretending to be enlightened. Also, Buddhism is outwardly comparatively
boring. Psychopaths are more likely to be attracted to singing, dancing,
love, light, miracles, and channeling. Usually psychopaths have a great
deal of trouble sitting quiet and still. I appreciate the boring facade
of Buddhism, as it is a great protection.

\sphinxAtStartPar
At the same time the comparative scarcity of psychopaths in Buddhism
leaves Buddhists more vulnerable to them. A psychopath may be sitting at
the back of a meditation hall reading a book while everyone else is
meditating. Occasionally even monks in monasteries get involved in
strictly forbidden activities, such as sex, drugs, lying and stealing
and get away with it because no one expects monks to do such things.

\sphinxAtStartPar
Generally you are safer choosing a Buddhist teacher who has been
authorized to teach by a widely recognized teacher or tradition. Of
course, psychopaths may claim authorization to teach, but they usually
do not maintain close association with their tradition or other
teachers. On the other hand, exceptionally good teachers frequently
develop their own styles of practice which are different from their
tradition. As a rule of thumb you would do well to avoid teachers who
proclaim their enlightenment and put down other teachers. Although some
of the most effective teachers in the West are laypeople, monks and nuns
make safer teachers than laypeople, especially if they are actively
associated with their tradition.


\subsection{Co\sphinxhyphen{}psychopaths}
\label{\detokenize{psychopaths:co-psychopaths}}
\sphinxAtStartPar
Although prospects for fundamental change in psychopaths is unlikely,
the prospects for change in co\sphinxhyphen{}psychopaths are very good. Co\sphinxhyphen{}psychopaths
are close associates of psychopaths who are caught up in their web of
control and deception. They may be a spouses, partners or disciples.
They, themselves are usually not morally defective, but they have
accepted the artificial reality that the psychopaths have created, and
they believe that the psychopaths’ behavior is acceptable because of
their divinity, illness, abused childhood, enlightenment, etc. They have
been programmed by the psychopath to lie, cheat, steal, or even murder.
It is usually the disparity between their own inner sense of morality
and the rationalized, programmed morality of the psychopath that causes
them to break away. Frequently the self destructive nature of
psychopaths will lead them to overplay their hand and cause their
co\sphinxhyphen{}psychopaths to break away. Sometimes \sphinxstyleemphasis{co}\sphinxhyphen{}psychopaths leave
repeatedly and then return to their psychopath.

\sphinxAtStartPar
Psychopaths can be a disaster for an organization if they succeed in
getting a manager or teacher into a co\sphinxhyphen{}psychopathic role. There are some
signs to watch for when a legitimate spiritual teacher becomes involved
with a psychopathic partner: Teachers will have much less time to devote
to their teaching. There will be a greater emphasis on the teachers
making money either directly or indirectly as a result of their
teaching. The teacher may leave to start a new organization, or many key
people will leave. A new emphasis or activity will develop in the
teachings. The teacher may get involved with some immoral or illegal
activity. These signs do not mean a teacher has become immoral per se,
but it is an indication of the power of their partner to create a web of
commitments and distort reality.


\subsection{Helping a co\sphinxhyphen{}psychopath}
\label{\detokenize{psychopaths:helping-a-co-psychopath}}
\sphinxAtStartPar
Even after co\sphinxhyphen{}psychopaths break away, the roots of their programming are
deep, and there is a possibility that they may return to the control and
influence of the psychopath. It is a pattern of behavior similar to that
of co\sphinxhyphen{}alcoholics and abused spouses. Even when co\sphinxhyphen{}psychopaths are
successful in breaking away, there are prolonged periods during which
they gradually realize the extent of the artificial reality that they
adopted. Some never figure out what happened, other than realizing that
the relationship was unwholesome for them. Co\sphinxhyphen{}psychopaths may need
psychotherapy and support groups to overcome the inner conflicts they
experience. Their chances for recovery are very good. It is highly
recommended that co\sphinxhyphen{}psychopaths seek relationships with people or groups
that have had similar experience, such as former members of religious
cults.

\sphinxAtStartPar
If you have extricated yourself from a group dominated by a psychopath,
you will have a strong desire to get your friends out of it too. Unless
your friends in the group are expressing some doubts or reservations, it
is usually fruitless to trying to talk them into leaving. Co\sphinxhyphen{}psychopaths
are too caught up in the artificial reality of the psychopath to hear
your advice. If you try too hard to convince them, you will succeed only
in breaking off communication. The best strategy for a friend is to wait
until they show signs of doubt and give them understanding support at
the right time.


\subsection{Purity of motives offers protection}
\label{\detokenize{psychopaths:purity-of-motives-offers-protection}}
\sphinxAtStartPar
Some of the spiritual teachings you need to learn concern impurities in
your seeking. To the extent that your seeking is motivated by desires
for power, prestige, sex, sense desires, etc., you are vulnerable to
being seduced by a psychopath. To the extent that you are motivated to
become enlightened or to purify your mind of defilements, then you are
on safe ground. Usually our motives are a mixture of good and bad. If
you are a true spiritual seeker, you will convert experiences with
psychopaths to a process of purification which decreases your lower
motives and increases higher motives.


\subsection{Psychopaths do not want to be cured}
\label{\detokenize{psychopaths:psychopaths-do-not-want-to-be-cured}}
\sphinxAtStartPar
One of the most tragic aspects of psychopaths is that even though they
may pretend to want to change, they really do not want to. Psychopaths
are rarely responsive to psychotherapy. They are difficult to cure
because they don’t want to be cured. On rare occasions, they come to
reflect on their lives during a mid\sphinxhyphen{}life crisis and may truly desire to
change. Even then, it may take extensive psychotherapy with expensive
specialists to induce a true transformation of behavior. These rare
occasions occur only when they come to the desire to change on their
own, and \sphinxstylestrong{not when they have been caught and are only pretending to
want to change.}

\sphinxAtStartPar
It is not advisable to try to reform or cure psychopaths. The
best advice is to learn to identify them and develop a strategy for
cutting off the relationship. Although a relationship with a psychopath
is painful, it is also an opportunity to enhance your spiritual growth
and purify your own motives. Be patient with friends who are under the
control of a psychopath, and be ready to help them when they express
doubts. Awareness of the problem of psychopaths is half of the solution.


\subsection{Conclusion}
\label{\detokenize{psychopaths:conclusion}}
\sphinxAtStartPar
Although there is a superficial similarity between saints and
psychopaths, we can in time distinguish between them by objectively
observing their behavior. We all have some psychopathic tendencies, as
well as tendencies to be saints, but when these tendencies predominate,
we use the label \sphinxstyleemphasis{psychopath or saint}. When we encounter some
extraordinarily impressive personality, we would be wise to ask
ourselves if this is the personality of a psychopath. Observe their
behavior carefully to determine if they say what they mean and do what
they say and mean. There is a \sphinxstyleemphasis{Checklist for Saints and Psychopaths}
on the next page, and there are extra copies which can be removed at
the end of this book. Keep this list handy and review it from time to
time.

\bgroup\footnotesize


\begin{savenotes}\sphinxattablestart
\sphinxthistablewithglobalstyle
\centering
\begin{tabulary}{\linewidth}[t]{TT}
\sphinxtoprule
\sphinxstartmulticolumn{2}%
\begin{varwidth}[t]{\sphinxcolwidth{2}{2}}
\sphinxstyletheadfamily \sphinxAtStartPar
\sphinxstylestrong{Checklist for Saints and Psychopaths}
\par
\vskip-\baselineskip\vbox{\hbox{\strut}}\end{varwidth}%
\sphinxstopmulticolumn
\\
\sphinxhline\sphinxstyletheadfamily 
\sphinxAtStartPar
\sphinxstylestrong{SAINTS}
&\sphinxstyletheadfamily 
\sphinxAtStartPar
\sphinxstylestrong{PSYCHOPATHS}
\\
\sphinxmidrule
\sphinxtableatstartofbodyhook
\sphinxAtStartPar
SAY MEAN DO consistency
&
\sphinxAtStartPar
SAY MEAN DO disparity
\\
\sphinxhline
\sphinxAtStartPar
Adhere to own moral standards
&
\sphinxAtStartPar
Breaks own rules
\\
\sphinxhline
\sphinxAtStartPar
Pay debts
&
\sphinxAtStartPar
Many bad debts, writes bad checks
\\
\sphinxhline
\sphinxAtStartPar
Keep promises
&
\sphinxAtStartPar
Break promises
\\
\sphinxhline
\sphinxAtStartPar
Truth is highest standard
&
\sphinxAtStartPar
No true regard for truth
\\
\sphinxhline
\sphinxAtStartPar
Insists dose associates tell the truth
&
\sphinxAtStartPar
Tell close associates to lie
\\
\sphinxhline
\sphinxAtStartPar
Un\sphinxhyphen{}aggressive philosophy
&
\sphinxAtStartPar
Push philosophy aggressively
\\
\sphinxhline
\sphinxAtStartPar
Attractive but not drawing
&
\sphinxAtStartPar
Attractive and drawing
\\
\sphinxhline
\sphinxAtStartPar
Waits for you to seek help
&
\sphinxAtStartPar
Comes on with unsolicited advice
\\
\sphinxhline
\sphinxAtStartPar
Good reputation endures \& improves
&
\sphinxAtStartPar
Good reputation fades in time
\\
\sphinxhline
\sphinxAtStartPar
Projects \& organization grow \& improve
&
\sphinxAtStartPar
Projects \& organization degenerate
\\
\sphinxhline
\sphinxAtStartPar
In the long run things turn out well
&
\sphinxAtStartPar
In the long run things turn out badly
\\
\sphinxhline
\sphinxAtStartPar
People have long term benefit from association
&
\sphinxAtStartPar
People are damaged by long term association
\\
\sphinxhline
\sphinxAtStartPar
Have concern for effect of actions on self and others
&
\sphinxAtStartPar
Are unconcerned for effect of actions on self and others
\\
\sphinxhline
\sphinxAtStartPar
Will immediately apologize for errors
&
\sphinxAtStartPar
Apologize as last resort
\\
\sphinxhline
\sphinxAtStartPar
Look for their own mistakes \& will apologize
&
\sphinxAtStartPar
Ignore their own mistakes and apologizes only if cornered
\\
\sphinxhline
\sphinxAtStartPar
If trapped will not renounce principles
&
\sphinxAtStartPar
If trapped will do or say anything to escape
\\
\sphinxhline
\sphinxAtStartPar
Typically have good health
&
\sphinxAtStartPar
Typically have variable exotic health problems
\\
\sphinxhline
\sphinxAtStartPar
Typically have few accidents \& injuries
&
\sphinxAtStartPar
Typically have many accidents and injuries
\\
\sphinxhline
\sphinxAtStartPar
Felt loved when a child
&
\sphinxAtStartPar
Felt unloved when a child
\\
\sphinxhline
\sphinxAtStartPar
Can sit very still
&
\sphinxAtStartPar
Can sit still only when center of attention
\\
\sphinxhline
\sphinxAtStartPar
Encourage associates to be self reliant
&
\sphinxAtStartPar
Enslave people around them
\\
\sphinxhline
\sphinxAtStartPar
Refrains from using mind\sphinxhyphen{}dulling substances
&
\sphinxAtStartPar
Substance abuse common
\\
\sphinxhline
\sphinxAtStartPar
Are comfortable being in the background
&
\sphinxAtStartPar
Compulsion to become the center of attention
\\
\sphinxhline
\sphinxAtStartPar
May adopt a spiritual name one time
&
\sphinxAtStartPar
Adopt many aliases
\\
\sphinxhline\sphinxstartmulticolumn{2}%
\begin{varwidth}[t]{\sphinxcolwidth{2}{2}}
\sphinxAtStartPar
\sphinxstyleemphasis{Any one psychopath or saint is unlikely to have all of the characteristics listed. Just because someone has
some of these characteristics does not mean he or she is a psychopath or saint.}
\par
\vskip-\baselineskip\vbox{\hbox{\strut}}\end{varwidth}%
\sphinxstopmulticolumn
\\
\sphinxbottomrule
\end{tabulary}
\sphinxtableafterendhook\par
\sphinxattableend\end{savenotes}

\egroup


\section{Screw U}
\label{\detokenize{psychopaths:screw-u}}
\sphinxAtStartPar
In 1986 I gave a talk on the subject of Saints \& Psychopaths at a
meditation retreat that I lead with Shinzen Young. Several years later,
a tape of this talk was played on a Los Angeles radio station. The next
day I received many phone calls from people who wanted copies of this
tape and more information on the subject. It turned out that about half
of the people calling were mental health professionals.

\sphinxAtStartPar
One of the callers asked me about my education. I spontaneously replied
that I had gone to Screw U. Although I have a BA. degree in Psychology
(Ohio State University 1959), my advanced studies have come from my
personal contact with psychopaths in daily life. I encountered several
psychopaths when I was a businessman, when I was involved with
intentional communities and also in religious organizations. One of the
most valuable courses I took at Screw U was with a teacher I shared with
Ram Dass.

\sphinxAtStartPar
While I was manager of the Hanuman Foundation Tape Library I received
teachings from a teacher with whom Ram Dass was also receiving
teachings. It was the same teacher described in the previous chapter. I
regarded Ram Dass as my teacher, so I automatically gave the status of
being a legitimate teacher to his teacher. It is unlikely that I would
have been attracted to this teacher had Ram Dass not been involved.

\sphinxAtStartPar
After I had been involved with this teacher for about a year, Ram Dass
announced to his students that he no longer felt that his new teacher
was a legitimate teacher. Even though he gave a very clear and honest
explanation for changing his mind, it took a couple of months for me to
fully see that he was correct. I learned a great deal from the process
of getting involved with and extricating myself from, association with
this teacher.


\subsection{The Hindu Tradition}
\label{\detokenize{psychopaths:the-hindu-tradition}}
\sphinxAtStartPar
It is helpful to understand some aspects of the Hindu and Buddhist
teachings which Ram Dass taught. In the Hindu tradition the ultimate
objective is to merge your consciousness with God. Enlightenment is an
implied part of this process. Since it is difficult to merge your
consciousness directly with God, it is recommended that you merge your
consciousness with enlightened beings who are capable of merging their
consciousness with God. Usually there is a hierarchy of beings involved
with this process of merging consciousness with God. Devotees should
strive to merge their consciousness with their guru who is enlightened.
The guru has merged consciousness with his/her guru and a lineage of
gurus who have Merged with some deity such as Krishna, or Shiva, who has
merged with God. If your guru is enlightened, then it is quite possible
that this arrangement may hasten your attainment of enlightenment.

\sphinxAtStartPar
Although this system offers a short cut to enlightenment, it also offers
an opportunity for infinite mischief for a psychopath. A devotee should
cultivate an attitude of total oneness, trust and obedience to a guru. A
legitimate guru will use this attitude only to expedite the devotee’s
spiritual growth and will not use it for personal benefit. If you choose
the path of devotion, you should be able to distinguish between a saint
and a psychopath.


\subsection{Buddhism and Hinduism}
\label{\detokenize{psychopaths:buddhism-and-hinduism}}
\sphinxAtStartPar
Just as Jesus Christ was a revolutionary in Jewish culture, the Buddha
was a radical revolutionary in Hindu culture. He encouraged questioning
of spiritual authority and on a number of occasions set down principles
for judging the validity of spiritual teachers and teachings, He
directly rejected the idea of giving someone high spiritual status
because of the status of their parents. He frequently challenged the
idea that rites and rituals such as animal sacrifices, or bathing in a
holy river could absolve sin. Some of his views were so contrary to
Hindu culture that this may be one of the reasons that Buddhism died out
in the land of its origin. The Mogul invaders formed alliances with some
Hindu kings and eliminated all Buddhist monasteries and temples.

\sphinxAtStartPar
The oldest lineage of Buddhism, the Theravada tradition, lacks the
devotion to many different deities that is characteristic of the
Mahayana tradition. However, there is some expeditious value to devotion
as a means of speeding the progress of a student’s development. The
Mahayana tradition evolved for a long period amid Hindu culture after
the Theravada tradition had gone to Sri Lanka. The Mahayana tradition
has many more characteristics similar to Hindu culture than the older
Theravada tradition. The Mahayana tradition places the greatest emphasis
on enlightenment, but makes use of devotional qualities similar to the
Hindu tradition by worshiping gurus and many deities. Ram Dass’ new
teacher was primarily a Hindu teacher, but made use of some Buddhist
Mahayana teachings.


\subsection{The mysterious teacher}
\label{\detokenize{psychopaths:the-mysterious-teacher}}
\sphinxAtStartPar
I heard the first rumors about this teacher in November of 1974 when I
attended a ten\sphinxhyphen{}day \sphinxstyleemphasis{Vipassana} retreat with Ram Dass and many devotees
of Neem Karoli Baba Maharajji lead by Joseph Goldstein, Jack Kornfield
and Richard Barsky. At the end of the retreat Ram Dass mentioned to me
that he had found a new teacher, but strangely he would say nothing else
about this teacher. Someone else told me that he was taking teachings
from a Brooklyn housewife who had become spontaneously enlightened in
her bathroom.

\sphinxAtStartPar
I decided to visit friends on Long Island, New York to see if I could
get information about this new teacher. It soon became clear that my
friends were involved with this teacher too, but they would say nothing
about her or what they did with her. Several of my friends would
mysteriously join a car pool and drive somewhere together, then return
several hours later. After a few days I decided that I was going to
learn nothing more, so I decided to return to Colorado.

\sphinxAtStartPar
At that time I was managing the Hanuman Foundation Tape Library in
Boulder, Colorado. Two of my employees had been close devotees of
Maharajji and spent many years in India with him. Soon after my return
they received letters, from friends in New York that said devotees who
had been in India with Maharajji were welcome to attend the secret
classes. My employees wanted to go to New York to see what was going on,
so we drove to New York together. I figured that my association with my
employees would get my foot in the door and it did.

\sphinxAtStartPar
My employees were immediately included in all of the secret classes. We
were staying with friends in New York City. As before, they would
mysteriously join a car pool and go off to classes for several hours
each day. I did get some more information on the nature of the classes.
Most of the classes were held on week days in large houses occupied only
by students, or in the home of the teacher while her husband was at
work. There were five classes a week and the largest was over 100 people
on Wednesdays in a rented class room. The classes were different sizes
and only about a dozen people including Ram Dass were permitted to go to
all of the secret classes. Of course, the secrecy and exclusivity of
these classes increased my desire to belong to them.


\subsection{A call from the teacher}
\label{\detokenize{psychopaths:a-call-from-the-teacher}}
\sphinxAtStartPar
After we had been in New York for a couple of days, I received a phone
call. It was the teacher and without telling me who she was, she
aggressively asked me many questions about my sex life. It seems that I
dealt with these questions satisfactorily as I was told that a car would
come immediately and take me for a face to face interview.

\sphinxAtStartPar
The interview was conducted in front of a class of about three dozen
people. The interview involved many questions about the type of
spiritual practices I had done, where I lived and my relationship to Ram
Dass. It was made clear in the interview that I wasn’t quite up to the
high standards of spiritual development that was required for her
students, but it was implied that I had some potential if I worked
hard. At the end of the interview she told to see Hilda Charlton who
would assign me special practices to, as she said, “Clean me up.”

\sphinxAtStartPar
Hilda Charlton was a devotee of Nityananda, who was one of the most
famous gurus in India before he died. After the death of Nityananda,
Hilda lived several years in the monasteries of Sai Baba. Sai Baba has
long been the most famous guru in India and his monasteries are large
and luxurious.

\sphinxAtStartPar
Sai Baba had authorized Hilda to teach a class in New York City as part
of his organization. Two or three hundred people would attend her
Thursday night classes at Saint John the Divine Cathedral on Manhattan
Island. Hilda was the one who had discovered our secret teacher and
introduced Ram Dass to her.

\sphinxAtStartPar
When I went to see Hilda, she gave me instructions to be celibate,
vegetarian, to meditate, to do breathing exercises, and pray to the
Virgin Mary every day to relieve my sins. She told me to return to
Colorado and do these practices for three months before returning to New
York. I then went to Ram Dass who told me that I could attend secret
Tuesday night classes that the teacher had told him to conduct.


\subsection{From Boulder to New York}
\label{\detokenize{psychopaths:from-boulder-to-new-york}}
\sphinxAtStartPar
Since my employees were being admitted to all classes, we decided to
move the Hanuman Foundation Tape Library from Boulder to New York. Ram
Dass instructed me to rent a house in Queens on Long Island for the Tape
Library and for six of his students. My employees were assigned a
different house to live in. At this time there were six other houses in
Queens which each had about a dozen students living in them, but this
was the only one with just Ram Dass students.

\sphinxAtStartPar
A week after we moved into the house we all found ourselves in the
kitchen at one time. It was clear that we needed to get organized as to
cleaning and other chores; we decided to have a house meeting. Within a
minute the phone rang. It was our mysterious teacher who promptly told
us to call off the meeting. She appointed the person who answered the
phone the house leader and said that she would send Ram Dass to come the
next morning at 4 A.M. to conduct the house meeting. We were all quite
impressed at the timing of the call and the special attention we had
received.


\subsection{Attending secret classes}
\label{\detokenize{psychopaths:attending-secret-classes}}
\sphinxAtStartPar
It was not long until every one in the house except myself was attending
secret classes. It was clear that I was the low man on the totem pole in
terms of spiritual development. Then about a month later, I received a
call from the teacher who yelled at me, “Why are you not here!” It
seemed that she was trying to make me feel guilty. But very clearly no
one had told me to attend. She told me to show up for the largest class
next Wednesday.

\sphinxAtStartPar
I felt honored that the diligent work on the various practices I had
been assigned had paid off, and that I did not have to wait the three
months that Hilda had mentioned. The Wednesday class was packed like
sardines. We all sat on the floor, knee to knee, for about 3 hours
without getting up. The teacher knew each of us by name, and during the
class she had a brief interchange with everyone in the room. I was quite
impressed with how appropriate her questions to me were ond how accurate
her observations were. Wednesday was the day set aside for new students
and visitors. On occasion she would exorcise a devil, or do a healing
for newcomers. It was an interesting and dramatic class.

\sphinxAtStartPar
About a half dozen very advanced students who, we were told had psychic
powers sat in front and there would be occasional exchanges with them.
The teacher might ask them what invisible deities were in the room and
then tell them that they had observed correctly. It seemed that this
must be one of the most important places in the world since so many
important gods dropped by for visits. The advanced students would praise
our teacher for her love and devotion she had for her students and the
wonderful results of her teachings. In turn the teacher would praise the
advanced students for their enlightenment, psychic powers and tell them
how important they were. Clearly we were a very special group of people
to be honored by receiving such teachings.

\sphinxAtStartPar
After about three hours our teacher and about twenty advanced students
would leave the main room for special advanced teachings; the rest of us
were told to remain and meditate. Usually Hilda would be left in charge
of leading the meditations of the main group. After a while the teacher
would leave the second group and tell them to meditate while she took a
few of them to another room for even more advanced special teachings.
Finally, she would leave that group with Ram Dass to give him very
advanced teachings.


\subsection{The Divine Mother}
\label{\detokenize{psychopaths:the-divine-mother}}
\sphinxAtStartPar
We were constantly reminded that our teacher was an incarnation of the
Hindu goddess Kali. Kali is the wrathful aspect of the Divine Mother,
who is regarded as the mother of the world or reality. The Kali aspect
of the Divine Mother is frightening, the way a mother who loves her
children frightens them with punishment if they misbehaved, Kali has
black hair, black skin, and blood dripping from her mouth. She cuts off
the heads of sinful beings and wears their heads in a necklace around
her neck. Kali banishes sin and impurity from her children. Only the
spiritually pure are safe from the wrath of Kali.

\sphinxAtStartPar
Our teacher had long jet black hair, and it sometimes seemed that her
normal dark complexion took on a decidedly black tone. We were
repeatedly told she would take on our karma and as a result, we were
told, she would bleed copious quantities of blood from her mouth. On
occasion, students who were her attendants in her home would call other
students and tell them that their teacher was bleeding so much that they
were worried about her health. They were sometimes told that the
bleeding was from taking on the karma of their sin of doubt, and they
were reminded what a great sacrifice she was making because of her love
for them.

\sphinxAtStartPar
Our teacher had enormous amounts of energy. On occasion we had marathon
classes which would last sixteen hours, and she would be full of divine
energy when everyone else was ready to drop. We were told that she did
not sleep, but there would be times that she would go into a deep state
of samadhi for hours. It was not unusual for people to receive phone
calls from her at two or four o’clock in the morning. Her style was to
challenge people and make them feel guilty. She used vile language and
usually brought up subjects which would embarrass them, such as their
sexual inclinations. The ultimate punishment was to be banned from
attending classes, although most who were banned would eventually be
permitted to return.


\subsection{Ram Dass leaves The Teachings}
\label{\detokenize{psychopaths:ram-dass-leaves-the-teachings}}
\sphinxAtStartPar
After I had been there for about six months, our teacher announced that
Ram Dass would be leaving what we called \sphinxstyleemphasis{The Teachings}. Ram Dass
continued to attend classes for another month, but there were big
changes after this announcement. There were greater melodramas after Ram
Dass departed. It was announced that someone was trying to assassinate
our teacher, and we were told to be on the lookout for strangers with
guns. We were told that some lamas had come from Tibet to persuade our
teacher to return to Tibet with them. They had determined that she was
an incarnation of a deity, and they wanted her to return to her temple
so that they could worship her. Our teacher disappeared for several
days, and we were told that she was experiencing an overload of
spiritual energy. All students were instructed to sit up all night in
meditation to psychically draw off the excess energy so she could return
home. Then it was announced that our teacher had left her husband, and
they were going to be divorced. Again the teacher disappeared, but this
time a few students started to get phone calls to join her in Florida.

\sphinxAtStartPar
A few people had chosen to leave The Teachings before Ram Dass departed,
but after he left many more began to leave. I noticed that for a while
after people left they maintained a respect for our teacher, but about a
month after they left they developed a strong dislike for both The
Teachings and the teacher. It was a puzzle that took me a few months to
figure out.

\sphinxAtStartPar
The departing students left vacancies in the advanced classes, and
suddenly I found myself promoted to all of the most advanced secret
classes. It didn’t occur to me that the special attention that I was
receiving was because I had a close association with Ram Dass. Our
teacher was constantly pumping me for information to discover what Ram
Dass was doing and saying. As time went on, it became apparent that
there were various intrigues going on between the teacher and Ram Dass
as gifts were returned and confidences were betrayed. The teacher and
Ram Dass would deliberately give me conflicting instructions, and it was
clear that they were both trying to get me to make a choice between
them. Other students were also placed in circumstances where they had do
choose which teacher to follow.


\subsection{Ram Dass confesses}
\label{\detokenize{psychopaths:ram-dass-confesses}}
\sphinxAtStartPar
Finally about three months after his departure, Ram Dass made a public
explanation as to why he left The Teachings. He announced to his Tuesday
night class that he had been deceived into believing that the teacher
was enlightened. His suspicions started when the teacher made sexual
advances to him under the guise of advanced tantric teachings. The
private sessions of advanced teachings that he was supposed to be having
became a cover for a sexual liaison with a married woman. She claimed to
be beyond desire, but it became clearer and clearer to him that she was
motivated by lust. The hypocrisy of having an adulterous sexual
relationship while pretending to be a celibate teacher of celibate
students was too much for him. Eventually he decided to leave, but since
he felt that The Teachings were benefitting others he did not want to
create a schism.

\sphinxAtStartPar
He had intended to quietly leave the teachings without conflict.
However, the teacher would not leave him alone; he was constantly
confronted with dirty tricks instigated by her. When other people left
the teachings, they would come to Ram Dass to share their experiences of
their teacher. Soon a vast web of deception came into focus. The
information that convinced Ram Dass that the teacher was channeling
Maharajji had come from a diary of a woman who had been with him in
India. People who claimed to have seen large amounts of blood come from
her mouth had been instructed to call Ram Dass, Hilda and others to say
they were worried about the amount of blood. Everyone in the inner
circle of students believed that others had seen blood, but when they
compared experiences none had seen it. Her enormous amount of energy
came from pills. Much of her supposed psychic information came from an
intelligence network of her students.

\sphinxAtStartPar
The reason that people who left The Teachings developed dislike for her
within a month after leaving was that they would exchange information
with the others who left. People who remained in the teachings would
rationalize what they heard, or not hear what was being said to them. I
too rationalized what I was being told, and the awareness I developed of
this capacity for rationalization was one of the most valuable lessons I
learned in 1976.

\sphinxAtStartPar
Ram Dass’ revelations about the teacher seemed to me to be
incomprehensible affairs of titans. Worship of the Divine Mother was to
see that all reality was a manifestation of the Divine Mother. The
highest teaching was that even horrible and unpleasant things were also
manifestations of the Divine Mother. To see the Divine Mother behind all
forms of reality was to see the true reality. It was a contradiction in
terms to say someone had deceived me into seeing anyone as the Divine
Mother.

\sphinxAtStartPar
I continued to regard our teacher as the Divine Mother. How could I be
wrong? Everyone and everything was the Divine Mother, and to see things
this way was an assurance that I was on the path to liberation. It
seemed to me that I had benefitted from The Teachings, and I had made a
great deal of progress in my spiritual development. This judgment was
largely based on having progressed from being a rejected outsider to a
key member of the inner circle. This is an example of the dubious types
of reasoning that a co\sphinxhyphen{}psychopath is likely to use.


\subsection{The inner circle}
\label{\detokenize{psychopaths:the-inner-circle}}
\sphinxAtStartPar
Being in the inner circle provided an opportunity to witness deception
and dishonesty that I had not been aware of before. I witnessed her
taking pills and saw the effect they had on her energy. She took some of
her students on shoplifting excursions, she charged meals to other
people’s hotel rooms. She conspired with students to fake injuries for
insurance claims, and she would habitually tell people things which I
knew were not true. Seeing all of this as the \sphinxstylestrong{play} of the Divine
Mother was wearing thin. These things were clearly wrong.

\sphinxAtStartPar
A couple of months after Ram Dass made his revelations, I finally
announced my departure from The Teachings. It was a gradual process like
a balance scale slowly swinging to the other side. I left when the
balance was 51\% doubt and 49\% faith. In the ensuing weeks and months,
the balance continued to shift to increasing doubt.

\sphinxAtStartPar
Over the years I would occasionally encounter friends who never left The
Teachings. Knowing how their minds would rationalize anything I would
say, I would make little attempt to encourage them to leave. If I had
vigorously tried to get them to leave, they would have cut off
communication with me.


\subsection{Special interpretations of reality}
\label{\detokenize{psychopaths:special-interpretations-of-reality}}
\sphinxAtStartPar
Although the method of viewing everything as a manifestation of the
Divine Mother has some value in cultivating a transcendent awareness, it
is a method which can be easily abused. In fact, all methods which
require a special interpretation of reality are potentially dangerous,
especially if a psychopath is involved. The worship of deities, gurus
and teachers can be helpful in attaining liberation, but they all
involve adopting a special view of reality.


\subsection{Teachers should be guides}
\label{\detokenize{psychopaths:teachers-should-be-guides}}
\sphinxAtStartPar
My personal preference is now to be a light unto myself. My taste in
teachers these days inclines to boring Buddhists who focus directly on
the objective of enlightenment. The Buddha encouraged the attitude that
teachers are guides who are familiar with a path. The guides do not make
the journey for us, but only advise us. From \sphinxstyleemphasis{The Dhammapada:}

\begin{DUlineblock}{0em}
\item[] \sphinxstylestrong{273}
\item[] The best one is one who sees:
\item[] The best of paths is The Eightfold Path.
\item[] The best of truths
\item[] The Four Noble Truths.
\item[] Transcendence is
\item[] The best of states.
\end{DUlineblock}

\begin{DUlineblock}{0em}
\item[] \sphinxstylestrong{274}
\item[] This is the best way
\item[] There is none better
\item[] To the purifying insight.
\item[] Follow This Path
\item[] It puts an end to temptation.
\end{DUlineblock}

\begin{DUlineblock}{0em}
\item[] \sphinxstylestrong{275}
\item[] Enter upon The Path
\item[] Which leads to the end of pain.
\item[] Having learned to end mine
\item[] I have shown you The Way.
\end{DUlineblock}

\begin{DUlineblock}{0em}
\item[] \sphinxstylestrong{276}
\item[] You must do the work
\item[] Buddhas are only guides
\item[] Meditators who follow This Path
\item[] Transcend the bonds of temptation.
\end{DUlineblock}


\section{Mukti}
\label{\detokenize{psychopaths:mukti}}
\sphinxAtStartPar
Mukti Ma Deva Walla was the name which Neem Kara Baba Maharajji gave to
Jane. Mukti was traveling in India visiting all the big time gurus when
she came to Maharajji. He was famous in the West as the guru of Ram
Dass. Maharajji asked her what she did in America, and she replied that
she and her then current husband, Charles Berner, conducted weekend
Enlightenment Intensives. Maharajji replied to this by giving her the
spiritual name which means Goddess Who Sells Enlightenment.

\sphinxAtStartPar
Mukti came into my life in the summer of 1977. At that time I was living
by the yacht harbor in Santa Cruz, California. I had an income of
\$25,000 to \$35,000 a year from investments and was managing the Hanuman
Foundation Tape Library as an unpaid volunteer, distributing Ram Dass
tape recordings. Mukti, had been traveling with a friend of mine until
they had decided to go their separate ways, happened to be staying
nearby, so when I heard that there was going to be a gathering of
Maharajji devotees in the Bay Area I left a message at the home where
she was a guest, telling her about it.

\sphinxAtStartPar
Instead of returning the call, she arrived at my house in the evening.
She was on her way to Berkeley, and since it was late she spent the
night. One thing led to another, and we ended up being together for the
next two years. Because many of my friends were devotees of Maharajji,
she started to use the name Mukti.


\subsection{Feeling spiritually desolate}
\label{\detokenize{psychopaths:feeling-spiritually-desolate}}
\sphinxAtStartPar
I had been feeling spiritually desolate because I had never traveled to
the East and met the great gurus. I had heard stories of miracles,
insight and wonder from Ram Dass and many of my friends who had spent
years traveling and meditating in India, but I had been unable to go on
a spiritual quest because I was responsible for my mother, who was
having a progressive series of strokes. Also, I had founded the Hanuman
Foundation Tape Library four years earlier, and it was in a stage of
development that made a prolonged trip for me impossible.

\sphinxAtStartPar
It did not occur to me that the quality of my meditation practice, the
merit of my work and caring for a parent were more valuable spiritual
activities than an exotic journey. I was fascinated by the stories of
Mukti’s travels around the world and by her acquaintance with many
famous gurus, saints, and teachers. She was adept at getting their
personal attention, and she attributed this to their recognition of her
spiritual development.

\sphinxAtStartPar
At age thirty\sphinxhyphen{}three, beautiful and talented, she had given up the
comforts and luxury of a super\sphinxhyphen{}rich family to pursue her spiritual
quest, and could tell endless stories of her adventures. She spoke
fluent Greek, Spanish, Hindi and had a working vocabulary of many other
languages. Mukti could sing, dance and draw. She was very intelligent, a
great talker and she could get the attention of anyone she wished.

\sphinxAtStartPar
After we met, we were together constantly for three weeks during which
we told each other the stories of our lives. She was intimately
acquainted with many major gurus and spiritual teachers including Ram
Dass, Ken Keyes, Chinmayananda, Sai Baba, Anandamayi Ma, Swami
Satchitananda, Muktananda, Yogaswar Muni, Al Drecker, Amarit Desai to
mention just a few. She also told me many stories about her mother who
had been married to several different movie stars and millionaires. A
particular set of stories concerned her mother’s yacht in Coca Beach,
Florida; a 71 ft. converted PT boat with a cabin that covered the entire
deck. Mukti said she was receiving \$5,000 a month from a trust fund and
was supporting sixteen hippies who lived on the yacht. She had many
adventures with sex and drugs during this time. She would open the yacht
to crowds of people Sunday afternoons when they would play Ram Dass
lectures on the speaker system. I had previously heard stories before
about a yacht at Coca Beach where Ram Dass tapes were played and now
concluded that this was Mukti’s mothers yacht.

\sphinxAtStartPar
Mukti said that it was when she was living on the yacht that she met
Charles Berner, who developed the weekend Enlightenment Intensive and
fell in love with him. She said that he had tried to get her mother to
give them a large amount of money, but her mother was angered by that
and threatened to disown her and cut off her trust fund if she married
Berner. She married Berner anyway, and had not received any trust fund
payments since then.

\sphinxAtStartPar
Although the marriage ended in divorce, her mother had not forgiven her
disobedience, and she still was not receiving her trust fund payments.
Mukti said that eventually she would have to get all the delayed
payments in a lump sum, but she did not want to legally challenge her
mother who was trustee of her fund.

\sphinxAtStartPar
After being together with Mukti constantly for three weeks, I realized
that my office work was piling up as well as other urgent matters
demanding attention. The tape library office was in the house, a
normally convenient arrangement, but when I attempted to get some work
done I gradually became aware that there was something wrong with Mukti.

\sphinxAtStartPar
She could not leave me alone. It seemed that every five minutes she
would come down to sit on my lap, ask a question and sit on my lap, or
start an argument and end up sitting on my lap. I could not get any work
done. We would have a long talk in which I would explain the pressing
importance of my need for unbroken concentration. A few minutes later
she would cut her finger, or need help upstairs lifting something, or
want to go to bed. It just went on and on and, finally I would give up
and we would go to the beach. The next day, when I tried to work, the
pattern repeated itself.

\sphinxAtStartPar
I had planned to attend a three\sphinxhyphen{}month retreat at the Insight Meditation
Society in Barre, Massachusetts that fall. One of the things that
attracted me to Mukti was that, when I mentioned this, she said that
she, too, had been planning to attend that retreat. I had to adjust the
business to operate in my absence anyway, and after a couple of weeks of
effort, I finally managed to do so.

\sphinxAtStartPar
During the next two years we traveled around the world twice. Mukti
convinced me that traveling to India was more important than being
available if my mother had a serious stroke. Mukti had a thick address
book of names of wealthy devotees she had met while visiting various
gurus, and she was a master at using her association with the gurus as a
way to get invitations to stay in luxurious homes. My association with
Ram Dass also proved useful in that respect, and she taught me how to
capitalize on it. Before long I fell into her lifestyle of a
professional house guest living out of a suitcase.

\sphinxAtStartPar
Because of our transient lifestyle it was difficult to focus on the
quality of life we were leading. We were constantly entertaining or
being entertained as guests, and moving from place to place. We had no
particular responsibilities and promises did not have to be kept, as we
were always on the move. Mukti and I did no productive work during our
travels.


\subsection{The three\sphinxhyphen{}month retreat}
\label{\detokenize{psychopaths:the-three-month-retreat}}
\sphinxAtStartPar
Somehow we managed to arrive a month late for the three\sphinxhyphen{}month retreat.
The rules of the retreat were simple and strict: Total silence was to be
observed, except during interviews and talks with the teachers; there
was to be no communication with other retreatants, no sex, no drugs, no
stealing and no sentient being was to be harmed. The schedule of sitting
and walking meditation retreat began at 4:30 A.M. and ended at 10:45
P.M, There were to be no days off this schedule until the end of the
retreat in December. Mukti had told me impressive stories of her
previous meditation practice and advanced initiations, so I expected
that she would take to an intensive retreat like a duck to water.

\sphinxAtStartPar
She took to it more like a chicken takes to water. At the end of almost
every hour of sitting meditation she would be lying in wait for me. She
would grab my arm and lead me off to the boiler room in the basement
with an urgent need to discuss something. She was having some medical
problem, some psychic experience in her meditations, or plans had to be
made for our travels, and on and on. Those discussions in the boiler
room were strictly against the rules, and I kept reminding Mukti of
that, but in the next hour she would have a new emergency.

\sphinxAtStartPar
She had several exotic medical problems and was on a very special diet.
The diet was difficult for her to follow, so she got me to promise to
support her morally by following her diet. Some of our meetings in the
boiler room involved discussing the need for changes in our diet.

\sphinxAtStartPar
Once I had a serving of yellow tofu on my plate. Mukti mistook the tofu
for scrambled eggs that were forbidden in our diet. She was enraged at
my apparent betrayal, so in the middle of the meal she came over and
punched me very hard in the ribs. That was a rather extraordinary thing
to do at a meditation retreat in the presence of many sensitive, quiet
people.

\sphinxAtStartPar
It soon became apparent to the teachers that Mukti and I were having
major problems sticking to the rules of the retreat. Jack Kornfield had
us in almost every day, together or individually, to discuss our
problems, but after three weeks of trying to settle into the retreat,
the incident in the dining room was the last straw. Despite her talent,
beauty and being in line to inherit a hundred million dollars, I wanted
to end the relationship.

\sphinxAtStartPar
Mukti was quick to see that I had truly resolved to bail out, and for
two or three days she became a model meditator. I maintained my resolve
that our relationship was over, but seeing her settle into the retreat
opened my heart.


\subsection{The clean slate}
\label{\detokenize{psychopaths:the-clean-slate}}
\sphinxAtStartPar
Finally she left a note for me to meet her briefly in the boiler room
after lunch. Her manner changed from demanding insistence to being soft
and open. She confessed at great length that she now saw how out of line
she had been. She even told me about rule\sphinxhyphen{}breaking of which I had not
been aware. She told me that our experience and the meditation practice
had resulted in her making a complete transformation, and that she
wanted a clean slate and a fresh start on a new relationship. She had
been married four times already and now she felt that she had found a
relationship that might really work. She painted a vision of what
wonderful things we could do with her money, then she said that it was
difficult for her to meditate until we could resolve our relationship.
It was becoming difficult for me, too, so the next day we left the
retreat and checked into a hotel.


\subsection{Our trip around the world}
\label{\detokenize{psychopaths:our-trip-around-the-world}}
\sphinxAtStartPar
After visiting friends in Boston, we went to Texas to spend the
Christmas season with my relatives, then left the country on our first
trip around the world. We had been in Greece for about a month, visiting
Mukti’s relatives, when we received word that my mother had had a severe
stroke. We immediately returned to the United States to be with her.

\sphinxAtStartPar
Although it appeared that she would make a good recovery from paralysis,
that stroke seemed to affect my mother’s mind more than previous
strokes. She was convinced that Mukti and I were going to get married,
and she had been on the telephone with her friends, planning a wedding
for us.


\subsection{Getting married}
\label{\detokenize{psychopaths:getting-married}}
\sphinxAtStartPar
That brought up an issue that we had not resolved: Should we get
married? Although there had been a great improvement since our
experience at the meditation retreat, I could see that there were still
potential problems with our relationship. On the other hand, being
married would make traveling together easier, especially when we were
staying at monasteries in India. Mukti also said that I was the kind of
person of whom her mother would approve. If we got married and showed up
at her house with a baby in arms, her mother’s heart would open, and all
our problems would be over. From time to time Mukti had attempted to
call her mother. Sometimes her mother would talk with her\sphinxhyphen{}for a few
minutes, but usually she would hang up immediately when she realized it
was Mukti. I thought that was strange, but Mukti explained that her
mother was an unusually strong\sphinxhyphen{}willed woman. However, her plan to arrive
home with a baby seemed likely to break the ice.

\sphinxAtStartPar
I still could not decide about getting married, so Mukti and I called
Ram Dass and explained the situation. Ram Dass strongly recommended that
we get married, and said that he would take a 40\% interest in our
marriage, Maharajji would take a 10\% interest, and Mukti and I would
each have a 25\% interest.

\sphinxAtStartPar
With assurance from my teacher that it was my spiritual work, I no
longer hesitated to get married. We quickly found a judge to marry us,
and that quieted Mother’s plans for our marriage ceremony. Then, when it
was clear that Mother had settled down and was making a good recovery,
we resumed our trip around the world.


\subsection{India}
\label{\detokenize{psychopaths:india}}
\sphinxAtStartPar
India was indeed an awesome and spiritual experience. We spent three
months traveling to pilgrimage cities and visiting famous gurus. I was
ill with something or another most of the time I was there and quickly
caught malaria, so when we arrived in the pilgrimage city of Hardwar I
was taken immediately to the hospital. While I was there Mukti would
stop by and visit me for a few minutes each day as she was going to see
Anandamayi Ma, one of the most famous gurus in India. I, on the other
hand, was too ill to be entertaining.

\sphinxAtStartPar
After a few days I was well enough to join Mukti in our hotel, I slowly
continued to recover from malaria, but I had other problems, such as
diarrhea and boils. Mukti was insistent that I should be using
traditional Indian Ayurvedic medicine for my complaints and I agreed. At
one point we were staying at the Neem Kara Baba Ashram in Brindaban when
I began to develop two huge boils on my leg. I had just recovered from a
boil the size of a grapefruit that had a core with the dimensions of an
apple core. The boils were not only painful but made me feel ill from
all the toxins they created. I was becoming quite impatient with the
Ayurvedic treatments that seemed totally ineffective.

\sphinxAtStartPar
When it seemed that the two new boils were going to be as huge as the
first, I panicked. I finally decided that the Ayurvedic approach was not
working for me and that I was going to use Western medicine, so I began
scraping off the herbal Ayurvedic poultice with the intention of
applying antibiotics. When Mukti saw what I was doing, she became
enraged that I would break my agreement and she shouted louder and
louder at me as I proceeded. As I was about to apply the antibiotic, she
grabbed the medicine from my hand and ran outside with it. I chased
after her, caught her and wrenched the medicine from her hand. She was
screaming and crying at the top of her voice as she followed me back
into the room, trying to grab the medicine. By this time the entire
staff of the ashram had gathered outside our window. Mukti continued to
cry and then screamed that I had been beating her. I told the staff that
I had not harmed her, but she continued to insist that I had been
beating her. The staff finally left after they were satisfied that, at
least, I was not beating her then. It was especially embarrassing for me
considering my relationship with Ram Dass and the American devotees of
Maharajji.


\subsection{Another clean slate}
\label{\detokenize{psychopaths:another-clean-slate}}
\sphinxAtStartPar
Again I decided to end my relationship with Mukti, for it had become
obvious to me that it was more important to Mukti to control and
manipulate me than for me to be healthy. But as soon as I resolved to
end our relationship she abruptly changed. She cried for forgiveness and
admitted that she was wrong. Again she described the error of her ways
in great detail, became soft, open and asked for a clean slate to start
our relationship again. I had seen this change before, but I really
wanted to believe her. I went to the hospital for a shot of penicillin
and my boils immediately began to subside. So did the turmoil in our
relationship.

\sphinxAtStartPar
All went well until we returned to Santa Cruz, where Mukti wrote to her
mother to tell her that she was married again and was going to settle
down. She tried to call her, too, but had no more luck than before.


\subsection{The financial crisis}
\label{\detokenize{psychopaths:the-financial-crisis}}
\sphinxAtStartPar
Meanwhile, my financial situation was getting critical. I had been
spending money freely on the premise that there was a trust fund
somewhere accumulating \$5,000 a month, and that it was going to be only
a matter of time before the money would be available to us. By then we
had accumulated a total of \$60,000 in one year bank loans. I had been
borrowing from Peter to pay Paul, but it was getting difficult to play
that game when I owed both Peter and Paul.

\sphinxAtStartPar
Something had to be done immediately as we could wait no longer for
Mukti’s mother to come around. We had purchased a wide variety of items
on our trip with the idea that we could find out what kinds of things
would be best to import. I had purchased some new credit card sized
calculators in Hong Kong, thinking that it would be easy to reproduce
and print business cards on the back. The small calculators were quite a
novelty in 1979 and I thought they would make an excellent business gift
premium, so we decided to go into the business of custom imprinting
small calculators.


\subsection{The other strategy}
\label{\detokenize{psychopaths:the-other-strategy}}
\sphinxAtStartPar
Another strategy, as I mentioned earlier, was for Mukti to arrive on her
mother’s doorstep with a baby in her arms, but there was one problem
with that: I had had a vasectomy. Casting about, we located the nation’s
top specialist in reversing vasectomies, and he assured us that I had
excellent prospects, so we scheduled an operation.

\sphinxAtStartPar
There is much work to do when setting up a new business, but when I
tried to do it I discovered that Mukti’s impulse to pester me had not
subsided. She had been having increasing problems with stomach pain and
frequent vomiting, so we decided that it might be best for her to go to
the Hippocrates Institute, which specialized in a regimen of raw food
and wheat grass juice. It was one of Mukti’s favorite places. When she
left for the Hippocrates Institute I flew down to Los Angeles for my
operation.

\sphinxAtStartPar
The two weeks Mukti was at the Hippocrates Institute was a very
productive time for me. I established a source for small calculators,
bought and set up equipment for silk screening business cards on them,
developed direct mail advertising, and rented a mailing list. The
business was ready to go, and all that was left to do were routine
things that Mukti and I could work on together.

\sphinxAtStartPar
When Mukti returned from the Hippocrates Institute, she was unusually
eager for us to have sex immediately. We did, once, and immediately
after that she received a message from God saying that we were not to
have sex. This was one was frustrating, but I felt confident that this
situation would pass. In the Hindu tradition celibacy is an ideal even
for married people and since we would be in India with many Hindu
friends it seemed appropriate at that time.


\subsection{Importing from India}
\label{\detokenize{psychopaths:importing-from-india}}
\sphinxAtStartPar
We settled into developing our new business. Although we had a promising
start with the calculators, Mukti was not happy about it, saying that
the business was interfering with our spiritual development. What she
really wanted was to import things from India, as that would give us an
opportunity to visit the gurus when we went on buying trips.

\sphinxAtStartPar
We decided that the best item to import would be hand made silk rugs. I
had a good location for a rug business near the River Oaks section of
Houston, and Mukti knew many rich people all over the world. Mukti
would be the sales force, and we were confident that she would do well
selling investment quality silk rugs, so we abandoned the calculator
business and set out for a buying trip to India.

\sphinxAtStartPar
I knew that Mukti would want to stay on to play in India, but our
financial situation was critical. I made her promise faithfully that our
trip would be a quick one\sphinxhyphen{}week buying trip, and that under no
circumstances would we be gone for more than a month. Our plan was to
buy \$13,000 of silk carpets on my American Express card, then we would
return and sell the rugs before lapse of the three months required for
the charges to reach us from India.

\sphinxAtStartPar
On our way we stopped in Texas to visit my mother, and while there Mukti
suffered a miscarriage. The doctor mentioned that the fetus was six
weeks old.

\sphinxAtStartPar
We continued on to India, and there we ran into a former girlfriend of
mine with whom Mukti quickly developed a close friendship. We traveled
together to Kashmir to buy carpets and then went down to Bombay, where
we learned that Anandamayi Ma and Sai Baba would be in the same city in
southern India on Anandamayr s birthday. They were the two most
prominent gurus in India, and it seemed certain that those two favorites
of Mukti’s would come together on that day. The auspicious occasion was
only a week away, but our allotted week had expired.


\subsection{Mukti stays in India}
\label{\detokenize{psychopaths:mukti-stays-in-india}}
\sphinxAtStartPar
We finally decided that Mukti would stay on for another week with our
friend, but before I went ahead to Houston to prepare for the big rug
sale I reminded her of her solemn pledge not to stay more than one
month, and urged her to return to the United States as soon as possible.

\sphinxAtStartPar
A week went by and there was no word from Mukti. A month went by and
there was no Mukti or word from Mukti. Then, finally, after six weeks, I
received a telegram from Mukti in northern India saying that she had
typhoid and was too ill to travel.

\sphinxAtStartPar
I felt trapped in an impossible predicament. I wanted to go to India,
but I couldn’t. Mother’s health had deteriorated and her medical bills
were now \$1,700 a month more than her income, and I had to make up the
difference. I had to make over \$1,000 a month in interest payments, and
bills from our travels were coming in at the rate of about \$2,000 a
month. I had to make over \$5,000 a month before I even could pay for
food and shelter. I could not possibly go to India, or our finances
would fall like a house of cards—and Mother was dying.

\sphinxAtStartPar
I contacted everyone I knew who might be going to India. Starting with
Ram Dass I sent a string of people to Mukti’s aid, but it was another
six weeks before a report arrived.

\sphinxAtStartPar
Meanwhile, two weeks after the telegram, I finally received a telephone
call from Mukti in New Delhi. She said that she could not return because
she had lost her plane ticket, and she was still too ill to travel. She
begged me to come to India to be with her. I explained that it was
impossible for me to go to India, but that I had sent many people to
take care of her, I told her to use a credit card to buy another ticket
and return immediately, but she continued to beg me to come to India to
be with her, and then ended the call, saying that she was too ill to
continue. That was the last I heard from her for a few months.


\subsection{Reports from India}
\label{\detokenize{psychopaths:reports-from-india}}
\sphinxAtStartPar
Then the reports from India started to come in. Ram Dass’ letter was the
first, reporting that Mukti had returned to Hardwar and seemed in good
health and spirits. The journey from New Delhi to Hardwar is more
arduous than taking a plane from New Delhi to Houston, and that cast
doubt upon her claim that she was too ill to travel. Other reports came
in from time to time saying that she was doing well and did not need
help. One report included a relayed message that the plane ticket she
left with an agent had been sold and that she should come by and pick up
her money. So much for her excuse about not being able to return because
she had lost her ticket.

\sphinxAtStartPar
And then another kind of report started to arrive. Back in 1979 it took
credit charges made in India about three months to appear on statements
in America. It soon became clear that at the time Mukti claimed to be on
her death bed she was traveling almost continually to every major city
in India. The true magnitude of her lies and betrayal was starting to
dawn on me. How could someone I had totally trusted do such things to
me?


\subsection{The American Express card}
\label{\detokenize{psychopaths:the-american-express-card}}
\sphinxAtStartPar
I attempted to have her American Express card cancelled, but I had
signed a contract with American Express that legally obligated me to pay
any charges she made on her card. It was the same as if I had asked
American Express to cancel my card and then continued to make charges on
it. I had to pay the bill.

\sphinxAtStartPar
By the end of 1979, I had been forced to sell all of my income producing
assets, and I had ten thousand dollars in the bank When Mukti called me
at Christmas, while I was visiting my mother, I said some things to her
in anger. Her typical strategy when caught doing something wrong was to
counter with accusations of her own. True to form, she became angry with
me for not going to India to help her when she was ill. I made an
obvious reply to that and she hung up in a fury.


\subsection{\$20,000 a month in India}
\label{\detokenize{psychopaths:a-month-in-india}}
\sphinxAtStartPar
That was the beginning of what was probably a record for credit card
abuse in India. For the next five months Mukti went on a charging
rampage that averaged \$20,000 a month on my American Express card. India
was not accustomed to dealing with fraud of that kind. Indian merchants
rarely bothered getting authorization for charges because the telephone
system was so poor and authorization was never refused anyway. During
that time Mukti went to each of the American Express offices in India
and got the maximum of \$1,000 in travelers checks despite the fact that
I was trying to get American Express to cancel my card. I told them that
I had no chance of paying her bills, and advised them where she was and
what she was doing. The charges were stopped at last, but only after she
reached Hong Kong and tried to use the card there.

\sphinxAtStartPar
People are amazed that it is possible to spend \$20,000 a month in India.
Later Mukti admitted that it was hard work, but she proved that she had
a talent for spending money. She started in New Delhi by taking a suite
of rooms for herself and another for her boyfriend in one of the best
hotels. Then she bought plane tickets for a guru and his entourage of
twenty people and they all flew down to Bombay. There she checked into
the Taj Mahal hotel and took a suite for herself, one for her boyfriend
and a third for the guru. Meanwhile, she left the meter running on the
rooms in New Delhi. From Bombay she went to Madras and repeated her
performance. After all, she had to return to each major city every three
weeks to draw her limit of \$1,000 in travelers checks from American
Express.

\sphinxAtStartPar
She was quite adept at drawing attention with her credit charges. For
example, she charged a \$1,500 shawl that she gave to Anandamayi Ma,
which really impressed her Indian devotees. Again, when Indira Gandhi’s
son was killed in a plane crash, and she went to Anandamayi Ma for
consolation, Mukti saw a good attention\sphinxhyphen{}getting scene, so she rented a
white, air\sphinxhyphen{}conditioned Mercedes Benz, with a chauffeur and arrived at
the ashram while Indira Gandhi was there.

\sphinxAtStartPar
Throughout that period I was at my mother’s bedside, as it appeared that
she was close to death. But after three months her condition stabilized,
so I made a trip to California to take care of some of my possessions
there. On the way I stopped in Santa Fe, New Mexico, to visit Ram Dass
for a few days, and mentioned to him that his share of the current
month’s American Express bill was \$8,000. He was silent for a few
minutes, and then said, “Is that a forty percent or fifty percent
interest I have in your marriage?” He never paid.


\subsection{Back after fifteen months}
\label{\detokenize{psychopaths:back-after-fifteen-months}}
\sphinxAtStartPar
I also visited another friend who was a devotee of Anandamayi Ma, and he
suggested that I write a letter to the abbot of one of her monasteries
and tell him exactly what Mukti had been doing. I did, and it worked.
Anandamayi ordered Mukti to return to me immediately, and within a month
she was back in Houston — fifteen months after she had faithfully
promised to return in one month.

\sphinxAtStartPar
I was prepared for her tactic of attacking when she was wrong, so when I
picked her up at the airport and she met me with an angry “Why didn’t
you come to India to be with me?” I replied with accusations of my own.
Immediately she switched from being angry to being soft and loving: “You
are right and I was very wrong. I love you. Forgive me. Give me a clean
slate and let’s start over.” She was a master and put on a good show,
but this time she had gone too far. I was determined to extricate myself
from my co\sphinxhyphen{}psychopathic role.

\sphinxAtStartPar
Mukti called her mother, and it appeared that at last she was giving in.
Her mother said she was purchasing a new house in Palm Springs,
California, and she invited Mukti to come out and see it in the
following month. At that time Mukti was staying with a friend in Carmel,
California, and was so overjoyed at the news that she put a deposit on a
million\sphinxhyphen{}dollar house in Carmel.

\sphinxAtStartPar
We agreed to a divorce and a settlement Mukti agreed to pay me \$150,000,
which represented fifty percent of the decline of my net worth while we
were married, and she also agreed to pay the \$150,000 in charges
remaining on my American Express account. The settlement was to be paid
on September 1, 1980, shortly before I was to start my long\sphinxhyphen{}delayed
three\sphinxhyphen{}month retreat at the Insight Meditation Society.


\subsection{“I stole that”}
\label{\detokenize{psychopaths:i-stole-that}}
\sphinxAtStartPar
I joined Mukti in California to separate our personal possessions that
were in storage there. First we separated our belongings in the usual
way, but when deciding who should take possession of items acquired
together Mukti would frequently say, “I stole that,” to establish her
claim to them. I was amazed at how frequently she would say this.

\sphinxAtStartPar
I had been vaguely aware that Mukti would pick up some merchandise from
stores and be in too much of a hurry to pay. Occasionally I would say
something to her about it and she would invariably reply, “God told me
to take it.” What appalled me was not only how frequently she said “I
stole that,” but that some items were from homes where we had stayed as
house guests. That was a sobering insight into how the mind of a
co\sphinxhyphen{}psychopath can rationalize and not see things that should have been
seen.


\subsection{Another delayed insight}
\label{\detokenize{psychopaths:another-delayed-insight}}
\sphinxAtStartPar
I had another delayed insight while we were in Santa Cruz, during an
amicable period when Mukti had agreed to make restitution in our divorce
settlement. After we had separated our personal possessions, Mukti took
me to attend a concert given by a friend whom she admired for living a
life of luxury by receiving multiple welfare payments. She said that
they had met when she was at the Hippocrates Institute. A couple of
weeks later it finally dawned on me that the last and only time I had
sex after my vasectomy reversal was one month before her miscarriage.
The fetus she miscarried was six weeks old, so she must have gotten
pregnant shortly after arriving at the Hippocrates Institute.


\subsection{There are no limits to the evil that a liar can do}
\label{\detokenize{psychopaths:there-are-no-limits-to-the-evil-that-a-liar-can-do}}
\sphinxAtStartPar
I had resigned as manager of the Hanuman Foundation Tape Library
when Mukti and I began our travels, but I still received some mail there
and stayed there whenever I was in California. One morning I had gone
out for breakfast. When I returned I discovered that Mukti had arrived
early for our meeting and had persuaded an employee to lend her a key to
get into the office. When I walked in, she was sitting at a desk,
talking on the telephone, with a stack of my unopened mail in front of
her.

\sphinxAtStartPar
A few months later I began to receive charges on a new Master Card that
must have been in my unopened mail, and in a few months they totaled
over \$20,000. That was after Mukti returned from India and apologized
profusely for what she had done with my American Express card, It was
after she asked for a clean slate to start over. It was after we had
agreed to our divorce settlement. That reminded me of a saying
attributed to the Buddha, “There are no limits to the evil that a liar
can do.” That saying has come to mind often since we went our separate
ways.

\sphinxAtStartPar
Fortunately, I did not have to pay for any of the charges on the stolen
credit card because I never actually received it. The charges on the
American Express card were only a civil matter involving a bad debt, but
the use of a stolen card constitutes theft, a felony. Meanwhile,
American Express would not issue me a new card, and that blew my last
good line of credit.


\subsection{The divorce settlement}
\label{\detokenize{psychopaths:the-divorce-settlement}}
\sphinxAtStartPar
The September 1, 1980 date for Mukti’s divorce settlement payment
arrived, but the payment did not. I called a mutual friend in California
who told me Mukti had gone to India for one month. (We know about her
one\sphinxhyphen{}month trips to India.) Mukti knew that my three\sphinxhyphen{}month retreat
started in September, so it was especially inconsiderate of her not to
let me know what was happening. I would not know until the end of the
retreat whether I had all my bills paid and \$150,000 in the bank, or no
money and \$150,000 in bills that I could not pay. Meditation is
difficult enough without a situation like that.

\sphinxAtStartPar
My mind was in turmoil for the first ten days of the retreat, but since
I had done everything I could, I felt morally clean, and that allowed me
to break through the confusion and settle into a good retreat
experience.

\sphinxAtStartPar
Though I would not know my fate until Mukti returned from India, I was
not at all surprised at the end of the three months to find that she had
not returned and had left no hint of when she might do so. With my
options now severely curtailed, I chose to begin a two\sphinxhyphen{}year retreat as a
hermit in a remote area on the island of Maui, Hawaii.


\subsection{Maybe there wasn’t a trust fund}
\label{\detokenize{psychopaths:maybe-there-wasn-t-a-trust-fund}}
\sphinxAtStartPar
In the spring of 1981, while I was secluded on Maui, Mukti came through
Hawaii on her way back from India, learned of my location from friends,
and came to visit me. She told me that when she called her mother in
Palm Springs, her mother said that it would not be a good time for a
visit, and abruptly hung up. Apparently, she had reflected upon the many
clean slates and new starts that she had given Mukti, and renewed her
resolve to have nothing to do with her. I told Mukti that my attorney
and American Express could find no trace of a trust fund in her name.
“Well,” Mukti replied, “maybe there wasn’t a trust fund. All I know is
that the last check that I received from my mother was for five thousand
dollars.”

\sphinxAtStartPar
It turned out that the woman whom Mukti called her mother was actually
her adoptive parent. Mukti’s real parents had immigrated from Greece to
New York City when she was ten years old. They had been wealthy, but
lost their businesses. They had difficulty adjusting to a new culture
and having to work and had put a heavy strain on their marriage. Mukti
felt ignored and unloved in the new situation, and had become a problem
child. She began to spend more and more time with an old friend of the
family who had a home in New York City, a wealthy but childless lady who
proposed to adopt Mukti. Because her real parents did not know what to
do with her, and their friend offered a good opportunity for their
daughter, they were quick to accept the offer.


\subsection{Cocaine Jane}
\label{\detokenize{psychopaths:cocaine-jane}}
\sphinxAtStartPar
Unfortunately, Mukti continued to be a problem child with her adoptive
parents. Her new mother divorced and remarried. Perhaps her adoptive
mother, giving attention to a new husband, made Mukti feel insecure and
resentful, but in any case she did not like her new stepfather, and by
the time she turned eighteen she had left home.

\sphinxAtStartPar
Living off a temporary allowance, Mukti became a rock band groupie and
had a rock song written for her entitled \sphinxstyleemphasis{Cocaine Jane}. Mukti told me
many stories about how her adopted mother would try to make her change
her behavior and lifestyle. Some things she did were quite embarrassing
to her family. After several major but futile efforts to get her to
straighten out her life, she was warned that if there were not a basic
change in her behavior, she would be cut off. She didn’t, and she was.


\subsection{There are no limits….}
\label{\detokenize{psychopaths:there-are-no-limits}}
\sphinxAtStartPar
After my two\sphinxhyphen{}year retreat in Hawaii I left my truck with a friend to
sell for me, having signed the title to make the sale easy. Mukti was a
houseguest of my friend shortly after I left, and she took the
opportunity to steal the signed title, sell the truck, and keep the
money. “There are no limits to the evil that a liar can do.”

\sphinxAtStartPar
Ken Keyes wrote extensively about his affair with Mukti in his book,
\sphinxstyleemphasis{Discovering the Secrets of Happiness}. The yacht which she described
as her mother’s in fact belonged to Ken Keyes. This is but one example
of how she changes names and circumstances to weave a lie that has some
basis in fact. From time to time I hear reports of Mukti’s trail of
devastation and betrayal. She has changed her name several times, and
for a while lived with a plastic surgeon who changed her appearance, but
her behavior has not changed. She stole \$40,000 from a lady who had
befriended and helped her often; she was caught selling food stamps and
has forged checks, but somehow managed to avoid jail; and I have heard
several rumors that people have invested in her schemes and lost their
money.

\sphinxAtStartPar
In 1992 I ran into Mukti at the Neem Karoli Baba Ashram in Taos, New
Mexico. She told me that the person I knew as Mukti was dead and that
she was a new person. The only difference I could detect was that the
old Mukti would have asked for a clean slate. True to form, I have heard
stories since then of the destruction and duplicity that the new Mukti
has left in her wake.


\subsection{Saint Mukti}
\label{\detokenize{psychopaths:saint-mukti}}
\sphinxAtStartPar
My experience with Mukti is particularly relevant to the subject of
saints and psychopaths. Mukti represented herself as being enlightened,
and indeed many people believed that she was. One of her previous
husbands was a guru, with a substantial following in the United States.
When Mukti came into contact with followers of that tradition, they
treated her with awe.

\sphinxAtStartPar
One of the techniques of this tradition was to be able to give
\sphinxstyleemphasis{shaktipat}. \sphinxstyleemphasis{Shaktipat} is a psychic energy a guru can administer which
has a profound effect on the recipient. \sphinxstyleemphasis{Shaktipat} can be
administered directly by a touch, or remotely in a large room, or even
in another country when the recipient is meditating. There are a variety
of experiences a recipient can have, such as falling into a deep trance,
sudden rapid, erratic breathing, energy sensations in the body, etc.
Once Mukti touched a man (from her former husband’s organization) who
was visiting us. He fell over in a brief trance and slowly recovered in
a confused, blissful state. He felt that he had been cleansed and
transformed by his experience, and was quite impressed by Mukti’s power.

\sphinxAtStartPar
I invited Mukti to give me \sphinxstyleemphasis{shaktipat,} but she claimed to have been
told that after having been given this power she should never use it
frivolously. She said that it should be administered only when someone
indicated certain signs of readiness for the experience, but she never
clearly explained what those signs were.

\sphinxAtStartPar
Although \sphinxstyleemphasis{shaktipat} may very well be a legitimate, beneficial
phenomenon used by saints and gurus, the ability to give \sphinxstyleemphasis{shaktipat}
should not be regarded as proof of anything. The experience of
\sphinxstyleemphasis{shaktipat} is similar to experiences that dubious Christian faith
healers are able to give some people. There have been some studies with
Kirlian photography which indicates that there is some kind of
transferal of energy from a healer to recipient, but there is no
explanation of what this energy is. Certainly we would not declare that
the electric eel is a divine fish because it is able to transmit a
profoundly effective energy. My view is that any strange or psychic
phenomenon should be taken at face value only and that no conclusions
should be drawn about anyone’s spiritual attainments because of their
powers. Because people usually regard the manifestation of powers as
proof of divine authority, some legitimate spiritual teachers manifest
powers to get people’s attention, but demonstrations of miracles and
powers may only prove that someone is a good illusionist or magician.

\sphinxAtStartPar
Mukti could also manifest psychic powers other than \sphinxstyleemphasis{shaktipat.} One
example occurred when we were unpacking after our first trip to India. I
was in the bedroom with the door closed, and I unpacked a souvenir from
a previous relationship that Mukti had asked me to throw away. I hid it
under some things in a drawer. About ten minutes later Mukti entered and
was walking across the room toward me when she suddenly stopped, turned
ninety degrees, crossed the room, opened the drawer, reached in,
directly pulled the object from the bottom, and said, “What are you
doing with this”?

\sphinxAtStartPar
Mukti frequently received messages from God. Sometimes the messages were
dubious, and at other times they seemed profoundly right and
appropriate. I never was convinced that they were indeed messages from
God, but while we were together I never entirely dismissed them as
fraudulent. I mentally filed those messages under the category of \sphinxstyleemphasis{Maybe},
but they tended to manifest a self\sphinxhyphen{}serving pattern from Mukti’s
viewpoint.


\subsection{Need for attention}
\label{\detokenize{psychopaths:need-for-attention}}
\sphinxAtStartPar
Mukti was motivated by a need for attention. I suppose it was rooted in
her childhood, because her parents paid her less attention as their
family and their life of luxury and comfort fell apart. Then the change
to a new life in a new country brought her even less attention. Whatever
the cause, it seemed that she would do anything, good or bad, to get the
attention she craved.

\sphinxAtStartPar
As we traveled around, constantly meeting new people, I had a ring\sphinxhyphen{}side
seat for observing whatever she did to people. Whether a maid in a
hotel, a waitress in a restaurant, a customs official, a policeman, a
stewardess, or someone sitting next to her on a plane, they would notice
and remember her. She seemed equally inclined to do something for which
they would hate her or love her, but they would remember their encounter
with her. She would never order something from a menu, but had to
exchange or substitute items, or have something especially prepared. It
was usually easier to get attention by complaining bitterly and
excessively about the quality of food but, on occasion, she would have
ringing words of praise for food and service.


\subsection{Trust and distrust}
\label{\detokenize{psychopaths:trust-and-distrust}}
\sphinxAtStartPar
It seems that Mukti was involved in an endless cycle of trust and
distrust. It was very important to her that people keep their promises
to her. She was constantly insisting that everyone around her make all
sorts of promises to her, but became enraged whenever she felt that
anyone had not kept a promise. At the same time she seemed incapable of
keeping her own promises.

\sphinxAtStartPar
Mukti’s constant lies added a dimension to her cycle of trust and
distrust. One of the first things she told me about herself was that she
was a reformed pathological liar. That was her first lie to me. At times
it was a good cover for things she had done in the past that involved
lying. As with promises, she would become enraged if she suspected
anyone told her a lie, and yet she seemed incapable of telling the
truth. Many times she told me stories of events in her life, changing
the names and circumstances slightly to create a false image of what
really happened. It was puzzling to me why she would do that, as it
would have been just as easy and interesting had she told the truth.

\sphinxAtStartPar
On occasion I was aware that she was telling someone something that was
not true or not entirely true, yet, when confronted with the fact, she
never failed to manufacture a rationalization or explanation. It is the
nature of a co\sphinxhyphen{}psychopath to readily agree to improbable distortions of
reality. One of the lessons I learned as I extricated myself from my
co\sphinxhyphen{}psychopath role was that she had told me countless lies which I had
been unable to detect, for even though I knew she would lie to other
people, I thought I was exempt. It wasn’t until after she failed to
return from India that I became aware that she had told me many lies.

\sphinxstepscope


\chapter{Part II: Saints}
\label{\detokenize{saints:part-ii-saints}}\label{\detokenize{saints::doc}}

\section{Saints}
\label{\detokenize{saints:saints}}

\subsection{What is a saint?}
\label{\detokenize{saints:what-is-a-saint}}
\sphinxAtStartPar
I previously defined a saint as a spiritual seeker who, through a
process of study, discipline, prayer, or meditation has attained a
purification of mind and true spiritual understanding. Now I am refining
this definition to say that saints are enlightened. In this chapter I
will be defining enlightenment from the viewpoint of different religious
traditions, scientific research, psychological theory and from the
viewpoint of the Buddhist tradition.

\sphinxAtStartPar
By defining saints as enlightened, I am excluding some saints from
different traditions whose sanctity is based on miracles, or as being
special authorities representing God. For example, I am being more
restrictive than the definition of a saint by the Catholic church. A
purpose of certification of saints by the Catholic church is to assure
people that their prayers to saints are received by angels in heaven who
can intercede on their behalf. The bases for canonization
(certification) include such things as three verified miracles and
review by church scholars, etc., but the standard does not include
enlightenment. Certainly it seems that some canonized Catholic saints are
enlightened, but there is no guarantee that all are. On the other hand,
traditional Buddhists regard Buddhism as having a monopoly on
enlightenment, and this does not seem to be true.

\sphinxAtStartPar
All major religious traditions have some saints, prophets, founders, or
authorities who seem to have been enlightened. Most religions have some
literature which touches on this subject, but the Buddhists, by far,
have the best technical understanding of enlightenment. They not only
have a clear systematic analysis of what enlightenment is, but also they
have a comprehensive system for attaining it that works. At the same
time, understanding the views of other traditions gives a richer
understanding than Buddhism alone provides. It is the contemplatives
from different religious traditions who can provide us with this link of
understanding.


\subsection{The Snowmass Contemplative Group}
\label{\detokenize{saints:the-snowmass-contemplative-group}}
\sphinxAtStartPar
In the early 1980’s Father Thomas Keating, a Catholic priest, sponsored
a meeting of contemplatives from many different religions. The group
represented a few Christian denominations as well as Zen, Tibetan,
Islam, Judaism, Native American \& Nonaligned. They found the meeting
very productive and decided to have annual meetings. Each year they have
a meeting at a monastery of a different tradition, and share the daily
practice of that tradition as a part of the meetings. The purpose of the
meetings was to establish what common understandings they had achieved
as a result of their diverse practices. The group has become known as
the Snowmass Contemplative Group because the first of these meetings was
held in the Trappist monastery in Snowmass, Colorado.

\sphinxAtStartPar
When scholars from different religious traditions meet, they argue
endlessly about their different beliefs. When contemplatives from
different religious traditions meet, they celebrate their common
understandings. Because of their direct personal understanding, they
were able to comprehend experiences which in words are described in many
different ways. The Snowmass Contemplative Group has established seven
Points of Agreement that they have been refining over the years:
\begin{enumerate}
\sphinxsetlistlabels{\arabic}{enumi}{enumii}{}{.}%
\item {} 
\sphinxAtStartPar
The potential for enlightenment is in every person.

\item {} 
\sphinxAtStartPar
The human mind cannot comprehend ultimate reality, but ultimate
reality can be experienced.

\item {} 
\sphinxAtStartPar
The ultimate reality is the source of all existence.

\item {} 
\sphinxAtStartPar
Faith is opening, accepting \& responding to ultimate reality.

\item {} 
\sphinxAtStartPar
Confidence in oneself as rooted in the ultimate reality is the
necessary corollary to faith in the ultimate reality.

\item {} 
\sphinxAtStartPar
As long as the human experience is experienced as separate from the
ultimate realty it is subject to ignorance, illusion, weakness and
suffering.

\item {} 
\sphinxAtStartPar
Disciplined practice is essential to the spiritual journey, yet
spiritual attainment is not the result of one’s effort but the
experience of oneness with ultimate reality.

\end{enumerate}


\subsection{Contemplatives and enlightenment}
\label{\detokenize{saints:contemplatives-and-enlightenment}}
\sphinxAtStartPar
Contemplatives from different traditions generally agree that there is a
transforming experience they agree to call enlightenment. They agree
that enlightenment is attained as a result of controlling the mind with
various forms of practice. Usually these forms of practice are done in a
simplified protected environment where practitioners are freed from
worldly concerns to direct their attention inward. The practices may
involve body motions or body sensations, sight or focusing the vision on
particular objects, an awareness of certain outer or inner sounds,
focusing on the sense of taste or smell, observing the processes of the
mind or controlling the processes of the mind with prayer, mantra,
reflection or meditation. The common denominator of these practices is
that they focus consciousness on a sense door (Buddhists include the
mind as a sense), and the result is a profound examination of the
present moment.

\sphinxAtStartPar
It is generally agreed that enlightenment is a progressive series of
experiences or understandings with sudden dramatic breakthroughs or peak
experiences. The methods used to induce enlightenment have a great
effect on the type of objective experiences contemplatives have. Even
within particular traditions using identical techniques, the objective
experience individuals have vary greatly. For example, at a particular
stage of development some people have visions, and others using the same
method do not. Despite the wide variety of objective experiences that
people report, teachers with extensive experience can identify the
essential common denominators. Regardless of the tradition, method and
individual experience, the result of enlightenment, in terms of wisdom
and relief from personal suffering, are identical. The wisdom and
reduced suffering are the result of a change in perceptual thresholds
which allow access to previously unconscious mental processes.


\subsection{Perceptual thresholds}
\label{\detokenize{saints:perceptual-thresholds}}
\sphinxAtStartPar
By focusing the mind in a profound examination of the present moment,
processes of the mind which were not accessible to normal consciousness
become conscious. Unconscious processes become conscious processes.
Enlightenment is a particularly good term for this process. It is like
turning a light on in a dark room so that which was unseen becomes seen.
Regardless of which sense object is used—sight, sound, smell, taste,
touch, or processes of the mind itself—the object becomes a projection
screen for observing the fundamental processes of consciousness.

\sphinxAtStartPar
Research done by Dr. Daniel Brown and Dr. Jack Engler of Harvard
University gives us an understanding of one of the factors involved in
enlightenment. Their work is reported in the book \sphinxstyleemphasis{Transformations of
Consciousness} by Wilber, Engler, \& Brown. Brown and Engler did
scientific studies of psychological characteristics before and after
enlightenment. Part of their studies involved changes in perceptual
thresholds as a result of \sphinxstyleemphasis{vipassana} meditation. \sphinxstyleemphasis{Vipassana}
meditation is a Buddhist practice, and their studies were done
primarily at the Insight Meditation Society in Barre, Mass.

\sphinxAtStartPar
The perceptual thresholds are levels where subtle or fast processes can
be observed. Below the threshold the process is not observed, and above
the threshold the process is observed. A tachistoscope or T\sphinxhyphen{}scope is an
instrument that can present visual displays at rates of thousandths of a
second. The T\sphinxhyphen{}scope has been used to determine what humans are capable
of becoming aware of at the level prior to conscious attention. Brown’s
experiments involved determining how slow objects needed to be flashed,
before the subjects were able to perceive them as two separate events.
The smallest gap of time between the two events an individual is capable
of perceiving the change is that individual’s threshold. Just as IQ will
vary among different people, perceptual thresholds vary. Scientists had
concluded in 40 years of research before Brown’s work that a threshold
for any particular person did not change in a lifetime.

\sphinxAtStartPar
However, Brown’s research produced a startling new finding. After
3\sphinxhyphen{}months of \sphinxstyleemphasis{vipassana} meditation his subjects had significantly
lower perceptual thresholds. They were able to perceive much faster and
subtler events than before the retreat. The changes were not small
changes but big changes. Changes were frequently 100\%, 200\%, 500\%. One
friend of mine had an increase of 1,500\%.

\sphinxAtStartPar
The results of Brown’s research give a scientific basis for
understanding the results of meditation practice. By focusing the mind
in a profound examination of the present moment, processes of the mind
which were not accessible to normal consciousness become conscious.
These processes are beyond the perceptual threshold of the normal
person.

\sphinxAtStartPar
People who have accessed these unconscious levels will have a great deal
of difficulty describing the essence of their experiences. It is
analogous to the difficulty of describing the essence of seeing a sunset
to someone who has been blind from birth. There is not only the
difficulty in describing an unfamiliar experience, but the problem of
differentiating one sunset from another. If one has seen only one
sunset, then it would be difficult to describe the essence of what a
sunset is. People who have seen many sunsets and think the world is flat
may describe the essence of their experience differently than people who
think the world is round. These problems are even simpler than the
problem of describing what happens when the unconscious is first
accessed.


\subsection{The difficulty of describing enlightenment}
\label{\detokenize{saints:the-difficulty-of-describing-enlightenment}}
\sphinxAtStartPar
Each contemplative tradition has evolved one or more systems for
describing the sequence of experiences that people will have as they
access unconscious processes and penetrate to deeper and deeper levels.
Usually this information is restricted to teachers and scholars. Of
course, the scholars cannot understand what they are describing because
they lack direct personal experience, but they often think they
understand. This is one reason why scholars from different traditions
will argue endlessly.

\sphinxAtStartPar
Even when people have had the experience of accessing their unconscious
processes, there are many problems in relating their experiences to the
descriptions in the texts. There are problems of relating the
experiences to specific beliefs of the religion they belong to. There
are problems in relating experiences for which there are no words. There
are problems in describing experiences in such a way that does not
encourage other meditators to have too much in the way of expectation.
If meditators have too much expectation, they will have difficulty in
settling into a profound examination of the present moment. There are
problems in describing things in such a way that other meditators may
imagine they are having experiences they have not had. The textual
descriptions tend to describe experiences of one person as though
everyone has very much the same type of experiences. It is also said of
attempts to describe these experiences that, “Those who know need no
explanation.” “To those who do not know, no explanation is possible.”

\sphinxAtStartPar
Even for scholars in one tradition who have accessed their own
unconscious processes, there is difficulty in relating one system of
descriptions to another. For example, all Theravada Buddhist meditation
masters are familiar with Buddhaghosa’s Progress of Insight. Rarely, if
ever, are they aware that both the Mahamudra and the Buddha Families of
the Mahayana Buddhist tradition are descriptions of the same sequence of
experiences.


\subsection{Mystical Experiences}
\label{\detokenize{saints:mystical-experiences}}
\sphinxAtStartPar
Some people would insist that what I am referring to as \sphinxstyleemphasis{accessing
unconscious processes of the mind} is much more than a mundane
psychological insight. When I say this is a psychological insight, I do
not intend to limit it to being just psychological. When Saint Teresa of
Ávila said that an angel penetrated her heart with a spear and left her
slumped in ecstasy, this was a mystical as well as psychological
experience. An experience of this type could very well be an example of
the first time someone accessed his/her unconscious processes. It is
very common for the mind to spontaneously project all sorts of visionary
experiences the first time this happens. Typically, Buddhists have
visions in Buddhist metaphors, Christians have visions in Christian
metaphors, Native Americans have visions in Native American metaphors
and Scientists have visions in scientific metaphors.

\sphinxAtStartPar
\sphinxstyleemphasis{Accessing the processes of the unconscious mind} means to observe the
process which are creating the reality of consciousness. People who have
accessed these processes may not be able to explain how reality is
created. They do have at least an intuitive understanding which serves
them very well. The understanding of reality is a central issue and a
common denominator for all religious traditions. Whether or not this is
a mystical experience is a question of semantics.

\sphinxAtStartPar
Most people who have accessed their unconscious processes would rate
their first access as one of the most profound experiences of their
lives. Even people who regard themselves as atheists report experiences
of transcending time and space. They might even say things like, “Space
and consciousness are the same thing. There is only one space and one
consciousness. The experience of self is only an illusion which
temporarily seems to create a separation from the one
space\sphinxhyphen{}consciousness. Ultimately, your greatest pleasure will be to let
go of your separation and merge your consciousness with the one
consciousness.” It is possible for people to make reports like this and
still regard themselves as atheists. Some people who report they have
had an experience of and understanding of the laws of karma regard this
as a profound mystical process. Others who have had essentially the same
experience regard it as becoming aware of a natural law such as the law
of gravity.


\subsection{Moral paragons}
\label{\detokenize{saints:moral-paragons}}
\sphinxAtStartPar
Whether or not these profound experiences that people have are mystical
or psychological, the result is a new or deeper understanding of
morality. In Buddhism, the traditional commentary texts state that one
who has attained the first level of enlightenment is a moral paragon who
would not break five basic precepts. The precepts are: 1) Not to kill
any sentient being; 2) Speak only the truth and never lie; 3) Not to
steal or take anything which is not freely offered; 4) Not to engage in
sexual misconduct; 5) Not to take substances which dull the
consciousness. Of course, the Commentaries were written by monks who
follow very strict rules.

\sphinxAtStartPar
In the real world, it is questionable to say that an enlightened
layperson, especially a Westerner, would never kill a mosquito, or
absolutely keep any other precept. Habits and cultural differences are
deep conditionings which may create variations in the standards of the
Commentaries. However, it would be safe to say that those who are
enlightened would be naturally inclined to follow the precepts. They
would be moral paragons.

\sphinxAtStartPar
Among the factors which contribute to this increased sense of morality
is a greater clarity of mind and greater awareness of what is happening
in the present moment. Also, as deeper levels of the unconscious
processes are accessed, meditators will spontaneously experience this
sequence of mental states: unconditional loving\sphinxhyphen{}kindness, caring for the
suffering of others, sympathetic joy in the happiness of others and a
deep equanimity. These mental states will occur in this order and
meditation masters can recognize and judge the progress of their
students based on which state is manifesting. In addition to the
spontaneous manifestation, Buddhists have specific practices where these
four mental states are cultivated as a concentration meditation
practice. The carry over into daily life after intensive meditation
practice is variable and depends on personality and other factors, but
increased clarity, wisdom and compassion are hallmarks of the
enlightened mind.


\subsection{Degrees of Enlightenment}
\label{\detokenize{saints:degrees-of-enlightenment}}
\sphinxAtStartPar
In the Buddhist view, people gradually become saints. One who has
attained the first level of enlightenment is regarded as being ¼ of a
saint or is partly enlightened; the second level is ½ of a saint and an
Arahant is a full saint. The Buddha frequently mentioned four levels of
enlightenment which he referred to as: Stream\sphinxhyphen{}winner, Once\sphinxhyphen{}returners,
Non\sphinxhyphen{}returners and Stream\sphinxhyphen{}crossers. The Buddha used many other names for
Stream\sphinxhyphen{}crosser. The most familiar is Arahant which means one worthy of
praise. He also called them Brahman which is the highest Hindu cast.
Buddhist scholars have evolved more technical terms which are more
commonly used by teachers today, which are: 1) \sphinxstyleemphasis{Sotapatti,} 2)
\sphinxstyleemphasis{Sakadagami,} 3) \sphinxstyleemphasis{Anagami,} 4) \sphinxstyleemphasis{Arahant.}

\sphinxAtStartPar
The first level of enlightenment provides access to the grossest level
where consciousness is formed. The processes of cause and effect can be
perceived as the basis for an intuitive sense of morality. When this
level is first accessed in meditation, people frequently experience a
revolution in their perception of the nature of reality. Frequently they
say that they can now see that the reality they experience in the
present moment is the result of the effect they have intended to have on
other people. I am referring to a direct perception of processes and not
an intellectual or philosophical understanding. Sometimes people report
that they have contacted previous incarnations at this phase of
meditation, and they see the relationship of this in their current life.
Whether or not the reports of past lives are true contacts, or a way of
the mind illustrating a new understanding, the essential wisdom is the
same.

\sphinxAtStartPar
If you do actions with intentions of love, caring, sympathetic joy and
equanimity, this is the type of experience you will have from other
people. If you do actions with intentions of hate, disregard,
selfishness and craving, this is the type of experience you will have as
a result. This process is seen as a natural law—the law of karma. The
Buddhist view is that no entity enforces this law any more than an
entity enforces the law of gravity. The intuitive understanding that
comes from this insight is basically the same for each level but grows
broader and deeper with each level of enlightenment.


\subsection{The stream of consciousness}
\label{\detokenize{saints:the-stream-of-consciousness}}
\sphinxAtStartPar
Sigmund Freud attributed the discovery of both the unconscious mind and
the concept that consciousness was a stream, to the Buddha. The Buddha
frequently used the analogy of a stream in his discourses. He would
admonish people to use mindfulness to become like an island in the
stream of consciousness. We should not become swept away in the stream
by sense desires.

\sphinxAtStartPar
The following are some verses from the Dhammapada which refer to the
concept of the stream of consciousness:

\begin{DUlineblock}{0em}
\item[] \sphinxstylestrong{25}
\item[] With sustained effort and earnestness
\item[] Discipline and self\sphinxhyphen{}control
\item[] The wise make islands
\item[] Which no flood overwhelms.
\end{DUlineblock}

\begin{DUlineblock}{0em}
\item[] \sphinxstylestrong{236}
\item[] Make an island of yourself
\item[] Strive quickly: Get insight!
\item[] Purged of impurity and passion
\item[] You shall enter the Buddha Realms.
\end{DUlineblock}

\begin{DUlineblock}{0em}
\item[] \sphinxstylestrong{238}
\item[] Make an island of yourself
\item[] Strive without delay: Get insight!
\item[] Purged of impurity and passion
\item[] To birth and age
\item[] You will come no more.
\end{DUlineblock}

\begin{DUlineblock}{0em}
\item[] \sphinxstylestrong{362}
\item[] One who is mindful in hand and foot
\item[] In speech and thought
\item[] One who delights in meditation
\item[] And is composed
\item[] One who is like an island
\item[] And is contented I call a monk.
\end{DUlineblock}

\begin{DUlineblock}{0em}
\item[] \sphinxstylestrong{34}
\item[] Like a fish out of water
\item[] If in anyone
\item[] The six streams of craving are strong
\item[] The flood of thoughts to pleasure
\item[] Will carry them off.
\end{DUlineblock}

\begin{DUlineblock}{0em}
\item[] \sphinxstylestrong{340}
\item[] The streams of craving flow everywhere
\item[] Like the creeper sends out strands.
\item[] Seeing the creeper that has sprung up
\item[] Cut off the root with wisdom.
\end{DUlineblock}

\begin{DUlineblock}{0em}
\item[] \sphinxstylestrong{341}
\item[] As pleasures arise that rush
\item[] Beings are flooded with craving
\item[] Wanting happiness they seek happiness
\item[] But come to birth and death.
\end{DUlineblock}

\begin{DUlineblock}{0em}
\item[] \sphinxstylestrong{348}
\item[] Let go of the past.
\item[] Let go of the future.
\item[] Let go of the present.
\item[] Crossing beyond the shore of existence
\item[] With mind released from everything
\item[] Go beyond birth and death.
\end{DUlineblock}

\begin{DUlineblock}{0em}
\item[] \sphinxstylestrong{414}
\item[] He who has passed:
\item[] Beyond this quagmire
\item[] This difficult path
\item[] The ocean of life
\item[] And delusion
\item[] He has crossed \& Gone Beyond
\item[] He is meditative
\item[] Free of craving and doubts
\item[] Who clings to naught
\item[] Has attained Nirvana
\item[] I call an Arahant
\end{DUlineblock}


\subsection{Stream\sphinxhyphen{}winner}
\label{\detokenize{saints:stream-winner}}
\sphinxAtStartPar
Stream\sphinxhyphen{}winner is the first level of enlightenment. I have never been
satisfied by any of the descriptions of what changes occur as a result
of attaining it. I doubt that I ever will because it is sort of like
describing shifts in shades of grey. Also there is a wide variation in
changes from individual to individual as the result of attaining any
level of enlightenment. This makes clear definitive statements very
difficult.

\sphinxAtStartPar
The Buddha most often talked about the number of incarnations people
would have after attaining certain levels of enlightenment. A
stream\sphinxhyphen{}winner will not reincarnate in this world more than seven times.
A Once\sphinxhyphen{}returner will be born in this world only one more time. A
Non\sphinxhyphen{}returner will reincarnate only in a high heavenly realm. An Arahant
will go into Nirvana just before death and never return. This was a big
selling point in a Hindu dominated culture. People in the time of the
Buddha not only believed in reincarnation, but they were concerned about
having to endure an almost infinite number of lifetimes.

\sphinxAtStartPar
The number of incarnations before attaining enlightenment is not a hot
topic for most Westerners, but suffering is. Westerners want clearly
defined and quantified results or objectives. I could say that you would
have less suffering, but I cannot say precisely how you would suffer
less. I could say you would be happier, but I cannot say exactly how you
would be happier.

\sphinxAtStartPar
There are some definitive descriptions of changes at different levels of
enlightenment in the classic Commentaries by Buddhist scholars.
According to the Commentaries, with the attainment of each level certain
defilements of mind are uprooted, but more deeply rooted defilements
remain. At the first level, the defilements uprooted are: doubt that the
Eightfold Path will lead to total purification of the mind, belief that
rites and rituals will result in enlightenment, belief the self exists
and defilements which would cause rebirth in the realms of hell, hungry
ghosts and animals. The belief that only direct knowledge and perception
of one’s own mind can result in enlightenment becomes firmly
established, as well as an inclination to continue the process of
purification of one’s mind. At the second level, defilements of lust and
aversion are weakened. At the third level the defilements of lust and
aversion are fully uprooted. The Fourth is the highest path that can be
attained and still remain in a separate individual body. The final
defilements uprooted are: a. the last veil of unknowing, b.
restlessness, c. craving for the subtle material realms, d, craving for
the subtle immaterial realms, e. conceit.


\subsection{Belief in self}
\label{\detokenize{saints:belief-in-self}}
\sphinxAtStartPar
The classic descriptions in the Commentaries don’t hold up very well to
Western standards of scientific quantifiable definition, for example,
the uprooting of the belief of self. I have yet to meet anyone who
believes that they do not exist as a result of becoming enlightened.
However, there is a change in the definition of self which occurs when
the first level of enlightenment is attained. Before enlightenment,
there is a tendency to define oneself as being in control of one’s
destiny, and after, one sees oneself as a natural process which must
follow natural laws. Before enlightenment, one tends to see oneself as a
solid fixed pattern, and after enlightenment, as a fluid pattern of
energy. These changes can be measured by psychological tests, but they
are projective tests which are regarded as \sphinxstyleemphasis{soft science} because they
involve subjective evaluations of the tester.

\sphinxAtStartPar
The Buddha had a negative attitude toward the Hindu belief in \sphinxstyleemphasis{atta}.
Westerners may believe in a soul, but it is a very fuzzy and poorly
defined concept compared to the Hindu concept. The Hindu concept is that
the core of your consciousness is an unchanging, unalterable object
which seems separated from the consciousness of God. The \sphinxstyleemphasis{atta} cannot
be affected by fire or cold, cut by a knife, smashed by a rock, shocked
by electricity, shot by a bullet, nor is it affected when you die.
Traditional Eastern Buddhist teachers spend a considerable amount of
time demolishing this concept of self that Westerners do not even have.
There is great value in this discussion in that it turns the attention
of the students to the processes of their consciousness.

\sphinxAtStartPar
I suspect that the Buddha had two reasons for his negative attitude
toward the concept of \sphinxstyleemphasis{‘atta’:} First is the unskillful beliefs
Hindu’s tend to have had about \sphinxstyleemphasis{atta;} Second is that \sphinxstyleemphasis{atta} types
of meditations should not be mixed with the \sphinxstyleemphasis{vipassana} type of
meditation the Buddha advocated. The Buddha was specifically against the
idea that the ritual breaking of the father’s skull on the funeral pyre
by the eldest son would release the \sphinxstyleemphasis{atta} and the father would attain
enlightenment, despite any sins. The Buddha likened this to expecting
rocks in a clay pot on the bottom of a pool of water, to float to the
surface if the pot were broken.


\subsection{Wave Theory vs Particle Theory}
\label{\detokenize{saints:wave-theory-vs-particle-theory}}
\sphinxAtStartPar
Dr. Daniel Brown has pointed out that the theoretical model that the
Hindus use is analogous to the theoretical physicist’s wave theory. The
theoretical model Buddhists use is analogous to the particle theory. A
hundred years ago physicists were divided into camps of disagreement as
to whether matter and energy were particles or waves. Present day
physicists use both concepts and sometimes use the wave theory for
observations and sometimes use the particle theory. However, they follow
the Uncertainty Principle that says that waves and particles can never
be measured at the same time. This advice would serve meditators well,
in that they should never try to mix \sphinxstyleemphasis{atta} theories with Buddhist
practices that emphasize viewing objects as being particles.

\sphinxAtStartPar
I think it improves the Buddhist concept to understand the Hindu concept.
The analogy that I use comes from an experience that I had during my
two\sphinxhyphen{}year retreat in Hawaii. My retreat camp was located at the
three\sphinxhyphen{}thousand foot level on Mt. Haleakala, and I had a vast view of the
ocean at the base of the mountain. I was on the south side of the
island, so my view was of the edge of the wave crests moving by the
island. As a form of meditation, I would pick an approaching wave and
follow it as it passed my location and watch it as long as I could see
it in the distance. It was very clear that an individual wave had a
separate distinct existence, and at the same time, it never separated
from the ocean. The wave very clearly moved across the ocean, and I had
no question that it existed.

\sphinxAtStartPar
If I went down and got into the ocean and observed the water, I would
have a fundamentally different experience of the same phenomenon. Being
in the water I could directly experience that there was no wall of water
moving across the ocean. I would find myself drawn forward as a wave
approached and buoyed up by a wave and carried backwards by the wave and
I would fall and move forward again as the wave passed. It was very
clear that a wall of water was not moving across the ocean. It was
arising and falling in a circular motion. From this view, the wave does
not exist, and it is just an illusion that results from not seeing the
true nature of the water. Of course, the Buddhist view is to carefully
observe the water from a close perspective, and the Hindu view is to
carefully observe the water from a broad perspective.

\sphinxAtStartPar
In the Hindu macroscopic view, the origin of the wave of self cannot be
perceived, but it is understood that the origin is God. The destination
of the wave cannot be perceived, but it is understood that the
destination of the wave is God. At the same time, the wave seems to be a
separate individual entity, and paradoxically it is never separated from
the ocean which is God. The Hindu meditation practice develops an
awareness of the continuous wave of consciousness unfolding and
infolding into experiences. The Hindu practice focuses on the present
moment as does the Buddhist practice, but the perspective is different.

\sphinxAtStartPar
In the Buddhist microscopic view, the wave is disregarded and the focus
is on process of experience. Phenomena arise and pass away, but there is
no concern for where they come from or where they are going. Attempting
to establish a continuity of self or concern for the origin and
destination of phenomena would block the type of insight that Buddhist
practice is trying to develop. In successful Buddhist meditation
practice, objects become discontinuous and less real.


\subsection{Deep insight}
\label{\detokenize{saints:deep-insight}}
\sphinxAtStartPar
Just before meditators access the flow of unconscious processes, they
develop a very profound awareness of what Buddhists call \sphinxstyleemphasis{nama\sphinxhyphen{}rupa},
or mind and matter. This is a state of awareness where the meditator
becomes vividly aware that there is the mind {[}nama{]} observing objects
{[}rupa{]}. This is fertile ground for what Buddhists refer to as \sphinxstyleemphasis{deep
insight.}

\sphinxAtStartPar
The term \sphinxstyleemphasis{deep insight} is an abbreviation of \sphinxstyleemphasis{Deep Insight into
Arising and Passing of Phenomenon} described in the Commentaries. When
concentration and the ability to observe change are developed to the
point that thought processes can be observed to arise and pass in the
mind, the nature of objects of consciousness appears to change. After
deep insight, insight into faster and subtler components of the thought
process is realized, and the meditator acquires an intuitive wisdom
about the nature of consciousness and reality.

\sphinxAtStartPar
Enlightenment is the most important attainment in meditation practice,
but \sphinxstyleemphasis{deep insight} is almost as important. It is said that once \sphinxstyleemphasis{deep
insight} has been reached, it is certain that enlightenment will be
attained. However, describing \sphinxstyleemphasis{deep insight} is a very delicate issue
for two basic reasons. One is that there are three basically different
ways that \sphinxstyleemphasis{deep insight can} develop, and there are innumerable
variations of responses individuals have to this experience. The other
is that if the descriptions of what happens are too explicit, then other
meditators will mistake experiences they are having as being \sphinxstyleemphasis{deep
insight} experiences.

\sphinxAtStartPar
Assume for a moment that an objective description of \sphinxstyleemphasis{deep insight}
involved an intense itching of the left earlobe and a teacher
mentioned this at a retreat. It is fairly certain that meditators’
interviews the next day would include numerous reports of itching all
over the body, and some would report mild to intense itching of the left
earlobe. This would be confusing for the students and the teacher.
Teachers need to be very guarded in their descriptions of phenomena to
avoid spurious reports. Sometimes teachers use analogies for \sphinxstyleemphasis{deep
insight} rather than objective descriptions.

\sphinxAtStartPar
My favorite analogous description of \sphinxstyleemphasis{deep insight} and ensuing
developments comes from my teacher Sayadaw U Pandita. Suppose you are
walking toward a wall and you notice that there is a line on it. As you
look at the line, you notice that the line seems to be sort of moving.
You look at it closer and find that it is not only moving, but that it
is not even a solid line. It is actually composed of moving clumps. You
examine the clumps closely and discover that the clumps are composed of
even smaller clumps. You look at these clumps and find that these clumps
are actually composed of even smaller clumps. You come close to the wall
and find that the smaller clumps are made up of even smaller clumps. You
come right up to the wall and examine it very closely and discover that
the smallest clumps are actually groups of ants moving together.

\sphinxAtStartPar
The line of ants is a safe analogy for describing the changes in
perception of reality as one begins the process of entering the stream
of consciousness. \sphinxstyleemphasis{Nama\sphinxhyphen{}rupa} is when you become aware of the line on
the wall. \sphinxstyleemphasis{Deep insight} occurs when you discover your consciousness
is not a solid continuous event and you actually begin to see that
phenomena appear as discontinuous events that arise and pass away. This
is not an intellectual understanding that your consciousness is
changing, but a direct perception of the process. The phenomena you see
become fundamentally different. As concentration improves, you penetrate
the new phenomenon, and it again undergoes a fundamental change over and
over again until you see the ultimate reality of phenomena. After seeing
the briefest endurance of reality, the mindfulness shifts to Nirvana,
which is the ultimate reality. Nirvana is an experience of the
Unconditioned which defies any description. Any description of Nirvana
is not a description of Nirvana, and that is the most that can be said
about Nirvana. There are no reference points in Nirvana on which to base
a description.


\subsection{Nirvana}
\label{\detokenize{saints:nirvana}}
\sphinxAtStartPar
A teacher determines that a student has experienced Nirvana partly on
the student’s inability to describe it. The student’s experience just
before and just after this experience will have certain distinctive
characteristics. Then a masterful teacher will give students specific
instructions to direct their practice in certain ways and evaluate their
proficiency in doing these tasks. After this, the teacher may directly
or indirectly let their students know they have attained the first level
of enlightenment.

\sphinxAtStartPar
The first experience of Nirvana is just a brief glimpse, but it is at
this point that one has truly entered the stream of consciousness. It is
like putting your foot on the bottom of the stream. During the course of
practice, the meditator experiences level after level of what were
unconscious strata of the mind. If practice is discontinued before the
experience of Nirvana and concentration is lost, then access to these
levels will be lost. It is as if the foot will eventually be returned to
the shore if the bottom of the stream is not touched. However, after the
first experience of Nirvana the first level of enlightenment is
attained. The first stratum of the former unconscious processes are then
permanently accessible to the consciousness.


\subsection{Parinirvana}
\label{\detokenize{saints:parinirvana}}
\sphinxAtStartPar
There is a final level of enlightenment the Buddha called parinirvana.
Once the final level of enlightenment is attained, Nirvana is never left
and individual existence ceases. Mahayana Buddhists take the Bodhisattva
Vow to delay their enlightenment until all sentient beings attain
enlightenment. The purpose is to remain in existence to be of help to
all other sentient beings in attaining enlightenment. Mahayana Buddhists
have different interpretations of what this vow means, but many regard
this as not entering \sphinxstyleemphasis{Parinirvana.}


\subsection{Finding a teacher}
\label{\detokenize{saints:finding-a-teacher}}
\sphinxAtStartPar
There are a number of factors to consider in determining who would be
the best teacher for you. Factors you might consider are: How
enlightened is the teacher? Do you have good chemistry with the teacher?
and is the teaching environment suitable? The more enlightened teachers
are, the deeper their intuitive understanding is. But the most
enlightened teacher may not be suitable for you, and it is not easy to
establish how enlightened a teacher is. Ultimately, you are looking for
a teacher that you think may be at least partly enlightened whose
teaching is in a context that you are comfortable with.

\sphinxAtStartPar
Although a more enlightened teacher is generally a better teacher, it is
possible that a Western Stream\sphinxhyphen{}winner who thinks and speaks in your
language might be a better teacher for you than an Arahant who does not.
However, you would be well advised to make the effort to transcend the
difficulties of culture and translation, and take advantage of working
with an Arahant if you have the opportunity.

\sphinxAtStartPar
The Buddhists have a clear set of standards in terms of enlightenment
for teachers. It is always appropriate for anyone to share what they
know about meditation. If you have no level of attainment and know basic
meditation instructions you can share the instructions. But you should
never advise anyone beyond your level of attainment. If possible, even
giving basic instructions should be deferred to a qualified teacher.
According to the Buddha the minimum level for a qualified teacher is
having attained \sphinxstyleemphasis{deep insight.} Generally in Burma and Thailand the
second level of enlightenment is reached before someone is given
training and recognition for being a teacher because there are so many
who have attained this level. Of course, in the West, standards are much
lower because of the lack of teachers and comparatively few have
attained higher levels.

\sphinxAtStartPar
There is a major problem in finding a very enlightened teacher, or even
an enlightened teacher. As I will explain in greater detail in the
chapter on the Embarrassment of Enlightenment, attainment of
enlightenment is almost always a secret. Generally, you would be well
advised to avoid a teacher who claims to be enlightened or very
enlightened. However, there are ways of figuring out who is likely to be
enlightened and how enlightened they are if you use time, effort,
patience and a skillful approach. I will be saying more on that in the
chapter on \sphinxstyleemphasis{The Embarrassment of Enlightenment.}


\subsection{Practice in Asia}
\label{\detokenize{saints:practice-in-asia}}
\sphinxAtStartPar
The degree of enlightenment alone should not be the only standard for
finding a suitable teacher. It is very possible to journey to Asia,
determine that someone is an Arahant, only to find out that he/she may
be an unsuitable teacher for you. The most likely obstacle is that
she/he may not speak English and no suitable translator is available.
You may have difficulties relating to the culture and personality of the
Arahant. Asian monasteries are notoriously noisy, hot and uncomfortable,
and there is a great chance you would be very ill some or all of the
time you are there. Some people thrive in these situations, but most
will find the difficulties too great.

\sphinxAtStartPar
There are some positive aspects to practicing in Thailand and Burma. It
is not necessary to ordain as a monk or a nun to practice in an Asian
monastery. My teacher, Sayadaw U Pandita discourages Westerners, in
Asia, from ordaining until after they have attained the first level of
enlightenment. The duties of monks and nuns take time from intensive
practice, plus dealing with awkward robes does not help the practice.
Foreign laypeople are usually very well cared for in Southeast Asian
monasteries, and in Burma there is no charge of any type, even for
laypeople, to stay in a monastery. Any monastery of a famous meditation
teacher in Thailand or Burma would be wealthy from strong public
support. If you are interested in a retreat longer than a month, the
plane ticket to Asia might be less than the cost of a retreat in the
West.

\sphinxAtStartPar
Even if you find an Arahant in a most suitable setting for you, there is
the question of what I call chemistry. Not everyone liked the Buddha. As
people become more enlightened, their personalties become more like
caricatures of their former personalities. (More on this in the next
chapter). Arahants typically have unusual and sometimes unpredictable
personalities. You are going to have good chemistry with some very
enlightened teachers, and with others you may not. You should consider
your limitations with dealing with certain types of situations and
personalities when looking for a suitable teacher.

\sphinxAtStartPar
Usually the best teaching situations involve a strong authoritarian
teacher who makes the rules and regulations. If there isn’t a strong
authoritarian teacher, then the staff and sometimes the meditators
become involved in spending enormous amounts of time and energy making
decisions.

\sphinxAtStartPar
In Asia it is not unusual for a monastery to become dysfunctional after
the death of a strong teacher. The reason is that the teaching of the
dharma tends to become an extension of the personality of the teacher.
Exceptionally good teachers teach intuitively and are likely to develop
a unique form or style of practice. Generally, they will want to avoid
situations where their teaching might clash with other strong teachers.
The use of different meditation objects, techniques, emphasis on study
etc. become a source of confusion for students. This is a good reason
why strong teachers usually prefer to have their own centers. In
Southeast Asia the monasteries are owned by a board of laymen and
occupied by monks. The board of directors will follow suggestions from
the founding teacher, but usually not from a successor.

\sphinxAtStartPar
Assuming that you are able to accept being in the authoritarian
environment of a strong teacher, there are other factors to consider.
Culture clash is not unusual. Some styles of practice involve many rites
and rituals that you may want to avoid. It is rare for Eastern teachers
to have scientific and logical views which are compatible with Western
culture. Western teachers are more likely to explain the dharma in
scientific terms or with personal stories, but generally they are not as
advanced in their personal practice as Eastern teachers. Eastern
teachers are more likely to be very traditional in their views and
methods of teaching.

\sphinxAtStartPar
Then there is the etiquette of relating to monks and nuns which is
uncomfortable to many Westerners. This repels some people and attracts
others. The bottom line is that you should feel that you have a good
chemistry with a teacher and the teaching situation, as one of your most
important standards.


\subsection{The search}
\label{\detokenize{saints:the-search}}
\sphinxAtStartPar
Your search for a teacher might include reading books, or listening to
tapes from many different teachers. When I was managing Insight
Recordings, my policy was to include free sample talks of a variety of
teachers with orders so people could make appropriate connections.
Although it is wise to do some shopping around for the right teacher, it
is even wiser to settle down with one teacher or style of practice when
you get into serious meditation practice.

\sphinxAtStartPar
There is a great danger in becoming superficially sophisticated. Some
people get into the habit of making the rounds of different teachers and
traditions. They become quite knowledgeable about personalities, gossip,
facilities, various techniques and practices, but never experience the
essence of the dharma. One teacher suggested that you should come to a
teacher like an empty glass of water. If you come like a full glass, you
will get little benefit.


\subsection{Conclusion}
\label{\detokenize{saints:conclusion}}
\sphinxAtStartPar
In this chapter I described what a saint is from an eclectic view and
extended it with the Buddhist concept of enlightenment. I explained
Buddhist concepts in ways that relate to Western psychological theories
and scientific studies. For those who are interested in finding a
teacher I have suggested some guidelines for finding a suitable one. In
the next chapter I will explore the possibilities that everyone has for
attaining enlightenment.


\section{The Possibility of Enlightenment}
\label{\detokenize{saints:the-possibility-of-enlightenment}}
\begin{DUlineblock}{0em}
\item[] \sphinxstylestrong{153}
\item[] Through many lives I wandered
\item[] Vainly seeking the builder
\item[] Of my house of suffering.
\item[] Sorrowful it is to be born again
\item[] And again.
\item[] 
\item[] But now I see you O builder!
\item[] You will build no house again.
\item[] The rafters are broken
\item[] The ridgepole is shattered
\item[] My mind has attained the Unconditioned
\item[] Achieved is the end of craving.
\end{DUlineblock}

\sphinxAtStartPar
This is a poem from the \sphinxstyleemphasis{Dhammapada} that is believed to have been
uttered by the Buddha immediately after his enlightenment. Since that
time, enlightenment has been passed from teacher to student like the
flame of one candle being passed from one to many.


\subsection{Spiritual decline}
\label{\detokenize{saints:spiritual-decline}}
\sphinxAtStartPar
The traditional belief in Asian countries is, that in the time of the
Buddha, it was easy to become enlightened. This is in line with both
Hindu and Buddhist scriptures, which describe the state of this world in
the past as being far superior to the present in both material and
spiritual standards. Many Buddhists believe that it is impossible to
attain enlightenment until the next Buddha comes. This is especially
true for Pureland Buddhists, who believe we must strive to be born in
the Pureland where enlightenment is possible. Even in Southeast Asia,
the majority of Buddhists make gifts to monks, attend rites and rituals
and even ordain as monks and nuns for the purpose of attaining merit to
be born in the time of the next Buddha.


\subsection{The revival of enlightenment}
\label{\detokenize{saints:the-revival-of-enlightenment}}
\sphinxAtStartPar
About 100 years ago, a revolution in Buddhist practice began in
Southeast Asia. Some prominent monks in Thailand \& Burma discovered that
if people started with \sphinxstyleemphasis{vipassana} meditation, that it was possible to
become enlightened. For hundreds of years, it was believed that \sphinxstyleemphasis{samatha}
meditation must be mastered before a meditator should switch to
\sphinxstyleemphasis{vipassana.} In \sphinxstyleemphasis{samatha} meditation the emphasis is focusing the
mind on an object in a steady manner. In \sphinxstyleemphasis{vipassana} meditation the
object is investigated to observe its true nature and change. \sphinxstyleemphasis{Samatha}
is essential for \sphinxstyleemphasis{vipassana,} but the new discovery was that
\sphinxstyleemphasis{samatha} can be developed simultaneously with \sphinxstyleemphasis{vipassana}. \sphinxstyleemphasis{Samatha} is
focusing the mind on an object, and \sphinxstyleemphasis{vipassana} is a way of looking
at the object. It really doesn’t matter what meditation object is used
in \sphinxstyleemphasis{vipassana.} Ultimately, the object becomes a screen for observing
the processes of the mind.

\sphinxAtStartPar
Although \sphinxstyleemphasis{samatha} meditation can result in developing great power and
bliss, these effects wear off quickly after meditation. Successful
\sphinxstyleemphasis{samatha} meditators frequently develop almost an addictive attachment
to their meditation practice becoming dependent on meditation for their
happiness, and they become irritable if their meditation is interrupted.
Without proper instruction, a \sphinxstyleemphasis{samatha} meditator is likely to develop
a fixed focus on an object that precludes observing the changes which
are essential for \sphinxstyleemphasis{vipassana.}

\sphinxAtStartPar
Until about 100 years ago, the meditation practice done in Theravada
monasteries was almost exclusively \sphinxstyleemphasis{samatha.} When \sphinxstyleemphasis{vipassana}
meditation monasteries were first built, there was considerable
criticism from conservative monks. This controversy continues to this
day as some prominent Buddhist monks still cling to the idea that
enlightenment is unlikely until the next Buddha comes. However, the
enthusiasm of many thousands of people who have attained some level of
enlightenment has resulted in a dynamic growth of the meditation
monasteries in Thailand, Burma and Sri Lanka. This trend is spreading to
all parts of Southeast Asia.


\subsection{Mainstream\sphinxhyphen{}entry}
\label{\detokenize{saints:mainstream-entry}}
\sphinxAtStartPar
Although emigrants from Asia have brought Buddhism with them, it has had
little tendency to enter the main stream of Western culture. However,
the practice traditions which result in enlightenment are now slowly
entering Western culture. The strongest influence is from Tibetan
Buddhism and the Zen traditions through Japan and Korea, and the least
prominent is \sphinxstyleemphasis{vipassana} practice from Southeast Asia.

\sphinxAtStartPar
The success of different traditions in transplanting to the West seems
to depend on contact with impressive meditation masters more than the
tradition’s intrinsic compatibility with Western culture. Tibetan
Buddhism with its nondemocratic lineages, rites, rituals, magic and
innumerable deities to worship, would seem the least compatible with
Western culture. The significant success of Tibetan Buddhism is due to
the very impressive high lamas who were forced out of Tibet in the
Exodus.

\sphinxAtStartPar
Likewise, the occupation of Japan after World War II and troops
stationed in Korea brought American troops into contact with some very
impressive Zen masters. The method of Zen emphasizes exacerbating
confusion and discourages clear analytical thinking. There is a very
strong characteristic in Western scientific thinking to analyze,
quantify and qualify which is in direct opposition to Zen. Japanese Zen,
in particular, has a strong Samurai militaristic characteristic which
gives their practice a boot camp type character. Americans regard boot
camp as a necessary hardship, but the true American ideal is to
celebrate individualism.

\sphinxAtStartPar
It would seem that Theravada Buddhism with its emphasis on scholarship,
logic and systematic analysis would be more compatible with Western
culture. Also, the practice of the simple technique of \sphinxstyleemphasis{vipassana}
meditation, almost devoid of rites, rituals and mystical mumbo jumbo
is intrinsically much more appealing to Westerners. It seems that the
limited success of Theravadans in the West is a result of the limited
contact Westerners have had with advanced meditation practitioners. Even
in Southeast Asia, \sphinxstyleemphasis{vipassana} meditation monasteries are still only a
rapidly growing minority. The contacts which have brought this tradition
to the West have primarily been a small group of Peace Corps workers and
hippies on spiritual quests in the 1960’s and 1970’s.

\sphinxAtStartPar
The introduction of Buddhism into Western culture is a very significant
event. Arnold Toynbee is regarded as one of the greatest historians of
modern times. Back in the 1930’s, he wrote the following, “When
historians look back at the 20th century, they won’t have much interest
in things like communism or capitalism: those will be ripples in the
great historical picture. What will really be significant is the impact
of Buddhism as it enters the West, because Buddhism has transformed
every culture as it has entered, and Buddhism has been transformed by
its entry into that culture.”


\subsection{Buddhism encounters Buddhism}
\label{\detokenize{saints:buddhism-encounters-buddhism}}
\sphinxAtStartPar
Buddhism is not only encountering another culture in the West, but
unique in history, Buddhism is encountering Buddhism. Historically,
Buddhism has been very provincial in its view. Each village in Asia
considers the teachings in its area to be the one true teaching of the
Buddha’s \sphinxstyleemphasis{dharma.} They see all others as more or less in error. The
Burmese think they have a better and purer Buddhism than Thailand. Thais
and Burmese look down on Sri Lankans as having a degenerated Buddhism.
However, all the Southeast Asians agree that the Mahayana Buddhists are
way off track. When Buddhists were isolated, it did not matter because
they were talking to themselves. With modern communication and
transportation, they are missing out on the potential of sharing the
special benefits that each tradition has to offer, because of their
provincial attitudes.

\sphinxAtStartPar
Buddhist leaders who have come to the West have shown a remarkable
tendency to maintain their provincial views. However, the Western
students have been much more open to exploring different forms of
Buddhism. Buddhism has become so integrated with Asian cultures that it
has become almost impossible for Asians to distinguish between the
essence of Buddhism and their culture. Westerners exploring different
forms of Buddhism are much more able to spot the common denominators and
see the essence of the Buddha’s teaching.

\sphinxAtStartPar
This is not entirely an intellectual process. What the Buddha did was to
describe natural laws that he came to understand from exploring his own
consciousness. He developed a philosophy and psychology which had the
ultimate purpose of attaining enlightenment. As Buddhism integrates with
and transforms a culture, the transformed elements of the culture become
part of Buddhism. By doing the practice successfully, anyone can have
the experiences which were the basis for the Buddha’s natural laws and
psychology. Having this experience plus studying different Buddhist
traditions makes it possible to identify the essence of the Buddha’s
teachings within different cultures.


\subsection{The way Buddha taught}
\label{\detokenize{saints:the-way-buddha-taught}}
\sphinxAtStartPar
The way Buddha taught exceeded what could be written down. It seems
reasonable that the Buddha taught in a similar way that meditation
masters do today. As meditation masters talk about Buddhist philosophy
and psychology, they will intuitively access the levels of their
consciousness where they came to understand the things they talk about.
They will spontaneously resonate on a psychic level to the unconscious
of their listeners. This resonance will make their students’ unconscious
processes more accessible to them, and some of them will begin to access
the unconscious stream of their consciousness. This access might be made
during a discourse, or later during meditation. Once this process is
started in people, they simply need to meditate until they attain
enlightenment. It is much like tending a fire until all the wood in a
pile is consumed. To do this properly, they should separate themselves
from their normal daily life, and practice in a protected
environment where proper guidance is available until this process is
completed.

\sphinxAtStartPar
The actual techniques early Buddhist teachers taught were various ways
of focusing attention, and unique instructions were given to balance
characteristics they saw in their students. It really does not matter
what technique is used, as long as it results in a profound examination
of the present moment. Those who believe that the Buddha had one true
technique have missed this essential point.


\subsection{Formless practice}
\label{\detokenize{saints:formless-practice}}
\sphinxAtStartPar
Most Buddhist meditation teachers today are taught to apply a
standardized technique to their students. This is not necessarily bad as
an enlightened teacher will intuitively modify the instructions to their
students needs, and the standardized technique provides some structure
of uniformity that reduces confusion, at least within the tradition.
However, trying to intellectually understand details of the instruction
frequently leads to problems in the practice. Some people are more
adversely affected by this than others. I suspect that the recent
popularity of Dzogchen, non\sphinxhyphen{}dual and formless practice from both the
Hindu and Tibetan traditions is in part an indication of reaction to
overemphasis on \sphinxstyleemphasis{exact technique.} Formless practice which, involves
dialogue with a charismatic teacher and self inquiry, has decimated some
\sphinxstyleemphasis{vipassana} groups around the country. Some people who practiced
\sphinxstyleemphasis{vipassana} unsuccessfully for years have been successful with this
method.

\sphinxAtStartPar
In Dzogchen and other formless practices, enlightenment is attained by
dialogue with the teacher and listening to the teacher having dialogues
with other students in the group. Unfortunately, some of these teachers
tend to cultivate a negative attitude in their students toward any
particular method or tradition. Followers of formless practice tend to
be dependent on the presence of their teacher to progress, but they
waste less concentration on trying to understand theory or subtle points
about technique.

\sphinxAtStartPar
The essential point that I want to make is that any one technique that
is used is not critically important The way the consciousness is
directed is important, whether a technique is used or not used. Any
method or non\sphinxhyphen{}method which results in a profound examination of the
present moment can, under the guidance of an enlightened teacher, result
in enlightenment.


\subsection{Mahasatipatthana Sutta}
\label{\detokenize{saints:mahasatipatthana-sutta}}
\sphinxAtStartPar
In the \sphinxstyleemphasis{Mahasatipatthana Sutta,} the Buddha advises people to go sit
“at the root of a tree” or in a “chamber”. Early Buddhist monasteries
were very simple compared with present day monasteries. Records of the
Buddha’s discourses indicate that the first monasteries were meditation
camps with a scattering of very small huts (made of branches, leaves and
grass) sometimes caves, and an assembly area which was usually outdoors.
In later sutras, the meeting area was sometimes described as a large
room with a sand floor. Monks and nuns would assemble when a bell was
rung. There would be no kitchen as the custom was to take a begging bowl
to a nearby village, or in some cases laypeople would bring food to
them. It wasn’t until about two hundred years after the time of the
Buddha that Buddha statues, elaborate temples, rites and rituals became
common.


\subsection{The oral tradition}
\label{\detokenize{saints:the-oral-tradition}}
\sphinxAtStartPar
Early Buddhism involved no reading or writing. As was the tradition for
other religions of India at that time, spiritual teachings were given
orally, even though devotees knew how to write. One form of teaching
involved chanting stylized summaries of the Buddha’s discourses. This
occupied the monks and nuns, developed concentration and provided a
vehicle for the teacher’s transmission of the \sphinxstyleemphasis{dharma.} Monks and nuns
would also have to learn to recite the list of rules of behavior that
the Buddha prescribed. The rules are methods for maintaining restraint
but, they also have the purpose of cultivating continuous mindfulness.


\subsection{The wide variety of methods}
\label{\detokenize{saints:the-wide-variety-of-methods}}
\sphinxAtStartPar
The records of discourses briefly mention the Buddha giving people a
wide variety of methods of meditation, and except for the
\sphinxstyleemphasis{Mahasatipatthana Sutta,} there is little description of technique. It
is possible that the Buddha expected teachers to rely on their intuitive
powers in developing a teaching style of their own, and modifying
methods to unique needs of individuals. This is what most of the best
meditation teachers do today. There is no conclusive proof that early
Buddhists followed any of the elaborately prescribed rigid forms of
practice typical of modern traditions. Some Buddhists insist that they
are following the only true method taught by the Buddha, and others will
explain that the degenerated times require regimented forms of practice.

\sphinxAtStartPar
The Buddha seems to have had an extraordinary psychic power to ignite
the process of insight in people. Even allowing for exaggeration and
glorification which comes with the development of all religions, he must
have been an extraordinary teacher. There are many reports, in the
discourses, about people attaining the first level of enlightenment, and
sometimes multiple levels of enlightenment while listening to the
Buddha.


\subsection{The teacher is more important than the technique}
\label{\detokenize{saints:the-teacher-is-more-important-than-the-technique}}
\sphinxAtStartPar
The attainment of enlightenment is very dependent upon the quality of
teaching and teachers. For example, in 1984 Sayadaw U Pandita led a
three\sphinxhyphen{}month retreat for teachers at IMS in Barre, Massachusetts. During
that retreat, some teachers attained the second level of enlightenment
or insight to that level. Since that retreat, the number of people
attaining \sphinxstyleemphasis{deep insight} during regular retreats has doubled. Also, it
was unknown for people to attain higher levels of enlightenment at IMS
before then, and since then a few have attained the third level.


\subsection{The decline of the Buddha’s teaching}
\label{\detokenize{saints:the-decline-of-the-buddha-s-teaching}}
\sphinxAtStartPar
It is probably correct that enlightenment was easier to attain in the
time of the Buddha. After all, the Buddha was an extraordinary teacher,
and there were many men and women who attained the level of Arahant who
were also extraordinary teachers. The Buddha taught in an area of India
which was relatively peaceful because it was dominated by the kingdom of
Magadha. Meditation and spiritual practices were very highly respected
in Hindu culture, and there was a strong tradition of public support for
holy men. There is fertile ground for enlightenment during times when
there is extra food to support meditators, times of peace when young men
are not conscripted, and there is broad public support for spiritual
practice. When these conditions are combined with the high quality
teaching of the Buddha, there are going to be more people getting
enlightened. It is just as logical that at times of war, plagues and
famine there are going to be fewer people enlightened.

\sphinxAtStartPar
The Buddhists of Southeast Asia have a stylized view of the general
spiritual decline since the time of the Buddha. Their view is that the
operant factor is the length of time since the passing of the last
Buddha. This view does not include the effect of cycles of war, plague
and famine on the practice of meditation, although they are regarded as
signs of decline. On the other hand, the Tibetan Buddhists regard their
teaching of Buddhism as better and more advanced than earlier forms of
Buddhism. Tibet was relatively peaceful and was able to produce an
abundance of food over hundreds of years before the Chinese invasion.
Tibet was uniquely successful as a Buddhist country, largely because of
its comparative isolation from foreign influence and domination. Before
the Chinese invasion, the population of Tibet had fallen from
six\sphinxhyphen{}million to two\sphinxhyphen{}million, over several hundred years primarily because
so many people ordained as monks and nuns. Southeast Asia still is not
up to that level of success.

\sphinxAtStartPar
Historically, Buddhism has been eradicated from many countries because
of war and politics, including the land of its origin, India. Even in
the 20th century, Buddhism has almost disappeared from several countries
for this reason. For the first two\sphinxhyphen{}hundred years, Buddhism was purely an
oral tradition. It was not until King Asoka sponsored writing down of
all of the Buddha’s discourses that were known two hundred years after
the time of the Buddha that Buddhism became focused on written texts.
The Abhidhamma (Buddhist psychology) wasn’t written down for
five\sphinxhyphen{}hundred years after the death of the Buddha. Modern communication,
transportation and media have made the prospect for the reestablishment
of Buddhism after calamities comparatively quick and easy. In ancient
times there must have been a greater gap, and much of the vitality of
early Buddhism was lost.


\subsection{Transmission of the dharma}
\label{\detokenize{saints:transmission-of-the-dharma}}
\sphinxAtStartPar
There are approximately one\sphinxhyphen{}hundred\sphinxhyphen{}thousand pages of the Buddha’s
discourses, and yet they contain almost no specific instruction on how
to meditate. The \sphinxstyleemphasis{Mahasatipatthana Sutta} is the richest source of
information on meditation practice. However, it is unlikely that someone
could use these instructions alone to meditate correctly. It is the
nature of enlightenment that many important teachings cannot be given
until a student’s practice has progressed to specific levels, and this
may be one reason written instructions may be inadequate. Monks in Sri
Lanka had these instructions for hundreds of years after calamities and
could not attain enlightenment with them. It is quite possible that the
Buddha intended that instructions could only be obtained orally, direct
from a teacher.

\sphinxAtStartPar
Without the personal guidance of an enlightened teacher, it is very
unlikely that the average person could successfully follow even the most
explicit instructions. Some authors of books on meditation have reported
that occasionally readers were able to practice successfully without
personal guidance. These cases are rare.

\sphinxAtStartPar
It seems that when a Buddhist culture goes through a time of disaster,
it is usually unable to support unproductive monastics. Even if the gap
is long enough so that there are no enlightened teachers, the culture
will retain its Buddhist religion. Temples and monasteries will be
rebuilt and scholars will copy texts and write Commentaries. However,
the vitality of transmission from enlightened teachers may be lost.
Mistakes in understanding the written text can be made, such as the
incorrect use of \sphinxstyleemphasis{samatha} practice in Southeast Asia. Although
Buddhism may appear to be revived, it may be only an empty religion of
moral code, rites, rituals and no enlightenment.

\sphinxAtStartPar
The point I want to make is that it is not necessarily true that
enlightenment is intrinsically more difficult today than at the time of
the Buddha. I think that if people are motivated to practice, have the
proper environment in which to practice and have proper guidance in the
right technique with an enlightened teacher, then they have the basic
ingredients for success. Of course, better teachers will get better
results.


\subsection{Follow instructions exactly}
\label{\detokenize{saints:follow-instructions-exactly}}
\sphinxAtStartPar
The majority of the people who make little progress in their practice
are not following meditation instructions as given. Only a meditation
master should mix techniques from different traditions. You should avoid
making any modification in the instructions without discussing it with
your teacher. One of my friends, who had done many years of intensive
\sphinxstyleemphasis{vipassana} meditation, never mentioned to her teachers that she was
also doing a mantra along with the regular practice. As soon as she
stopped doing the mantra she began to make progress. A slight
modification to the instructions that makes the practice more pleasant
or easier is very likely to sabotage it.

\sphinxAtStartPar
The most common variation on meditation instructions is to
deliberately indulge in psychotherapy. After the mind becomes calm and a
little concentrated from following the instructions, all sorts of
psychological insights may occur. Memories from childhood may start
bubbling up, or you might start to understand addictive behavior, or
work out problems with relationships. This process can become like
pulling feathers out of a down pillow. One insight leads to another and
it just goes on and on indefinitely. Although there might be
psychological benefit from this, it is a poor trade for the benefits of
true meditation practice. Many people have chosen psychotherapy over
enlightenment. Someone asked Anagarika Munindra, a great Buddhist
meditation master in India, why it was easier for Asians to attain
enlightenment. His reply was that, “Westerners are doing psychotherapy.”

\sphinxAtStartPar
Another variation on meditation instruction is to indulge in daydreams.
After the mind becomes calm and concentrated from doing a little
practice, images in the mind may become clear and colorful. It is
pleasant for the mind to wander into pleasurable daydreams. Daydreams
occur naturally from time to time in the practice, but they should not
be cultivated.

\sphinxAtStartPar
\sphinxstyleemphasis{Vipassana} meditation makes the mind very creative. You are very
likely to get all sorts of insights, ideas for talks or books,
inventions, artistic designs or community projects, and it will be very
tempting to stop the practice and develop them. During a two month
period of my practice, I deliberately spent one\sphinxhyphen{}third of my sittings
doing some mathematical calculations and developing theoretical models
in my head. I thought that I was only losing one\sphinxhyphen{}third of my practice
time. Later I came to understand how much momentum of progress I was
losing in my practice. Typically, meditators cover the area of progress
that I was making in my practice at that time in only a few hours.
Instead of losing one\sphinxhyphen{}third of my progress in practice, I had lost
almost all of what I could have accomplished in two months.


\subsection{How Long to Enlightenment?}
\label{\detokenize{saints:how-long-to-enlightenment}}
\sphinxAtStartPar
I haven’t seen any good statistics on how long it takes to become
enlightened. In any event, there are many variables to consider such as
how enlightened and suitable your teacher is, the proper environment,
the state of your practice, understanding and following the
instructions, just to mention a few. The Buddha said that if people
practiced continuously it would take between seven years and seven days
to become an Arahant. Those figures seem reasonable if your teacher is
the Buddha. One of my teachers, Tungpulu Sayadaw, practiced in a cave
for thirty\sphinxhyphen{}nine years without lying down, and he emerged an Arahant.
Another of my teachers, Dipa Ma, did three consecutive three\sphinxhyphen{}month
retreats, and attained a new level of enlightenment during each. If you
have a strong, correct daily practice, then a reasonable expectation
could be to attain \sphinxstyleemphasis{deep insight} in a ten\sphinxhyphen{}day retreat. If you have
attained \sphinxstyleemphasis{deep insight} in a ten\sphinxhyphen{}day retreat, then you have excellent
chances of attaining enlightenment on a three\sphinxhyphen{}month retreat.

\sphinxAtStartPar
Once \sphinxstyleemphasis{deep insight} is attained, the time it takes to reach the first
level of enlightenment depends on being able to observe precisely what
is happening without trying to manipulate what is happening. For
example, at certain phases of the practice people are likely to
experience visions. Some develop an attachment to these visions and even
learn how to produce them, It is possible to dwell in this phase of the
practice for prolonged periods, but the normal course of these visions
should be only a few hours. A similar attachment can develop to pleasant
energy sensations that flow through the body. Not all temptations to
manipulate what is happening are based on pleasant feelings. Sometimes
there are unusual pains, or feelings that the body is rotting. These
phenomena should be observed precisely, without trying to make them go
away. How long it will take to be able to profoundly examine pleasant
and unpleasant phenomena with equanimity is a key factor in determining
how long it will take for enlightenment to be attained.

\sphinxAtStartPar
Again, I haven’t seen any good statistics on how long it takes to get
from \sphinxstyleemphasis{deep insight} to Stream\sphinxhyphen{}winning. One teacher said it takes an
average of seven weeks for the average person. I have heard of people
doing it in two weeks, and one friend was not able to finish after 10
years of doing at least one three\sphinxhyphen{}month retreat every year. My sense of
it is that the seven week figure is probably fairly close to the
average.

\sphinxAtStartPar
It is an interesting phenomenon that the higher the level of
enlightenment, the more rapidly people progress from \sphinxstyleemphasis{deep insight}
to enlightenment. The key to the higher levels of enlightenment is
attaining \sphinxstyleemphasis{deep insight} to the higher path. After the first level has
been attained, most of the time spent doing intensive practice is
preparation for being able to attain \sphinxstyleemphasis{deep insight} again. One study
of how long it took to finish the second path after \sphinxstyleemphasis{deep insight}
ranges between two weeks and ten hours. One of my teachers went from
the third level of enlightenment to attainment of the fourth in one
sitting. The experience of each path is essentially the same, except
that the final phase of equanimity gets much deeper in each path. The
reason that the higher paths go so rapidly seems to be that after the
first experience, meditators intuitively know not to manipulate their
experiences.


\subsection{Some misconceptions about the possibility of enlightenment}
\label{\detokenize{saints:some-misconceptions-about-the-possibility-of-enlightenment}}
\sphinxAtStartPar
Frequently people have misconceptions about the possibility of
enlightenment. Some believe that enlightenment depends on grace from God
or a guru. Other misconceptions involve the importance of past lives,
vegetarianism, sex, race, education, and intelligence.


\subsection{Grace}
\label{\detokenize{saints:grace}}
\sphinxAtStartPar
The Snowmass Contemplative Group attempted to see if all traditions
would agree that there is such a thing as \sphinxstyleemphasis{grace.} All but the
Buddhists agreed that there was grace. Although many Mahayana Buddhists
pray to various deities to intercede and help them, the view of the
Buddha was that our fate is determined by the natural law of karma. From
the particle theory type view of the Buddhists, there would be no
exceptions to the law of karma any more than there would be an exception
to the law of gravity. What we experience is the result of our
volitional intent towards other sentient beings. The Buddha challenged
the idea that we could change our karma with rites, rituals and prayers.
He explicitly spoke against the idea that enlightenment could be
attained this way. Although teachers or gurus can help as guides, each
of us has to take the journey ourselves. This is in line with the
conclusions of the Snowmass Contemplative Group.


\subsection{Past lives}
\label{\detokenize{saints:past-lives}}
\sphinxAtStartPar
Another unprofitable view is that the spiritual work and merit from our
past lives will determine whether or not we attain enlightenment.
Whatever we have done in our past lives cannot be changed. In any event,
the basic factor in attaining enlightenment is settling into a profound
examination of the present moment. Concern for past lives can prevent
this from happening.


\subsection{Avoid getting too complicated in moral reasoning}
\label{\detokenize{saints:avoid-getting-too-complicated-in-moral-reasoning}}
\sphinxAtStartPar
Concern for the five basic precepts of not telling lies, not harming,
not stealing, avoiding sexual misconduct and avoiding intoxicants is
helpful. Basic morality is a necessary prerequisite for a calm
concentrated mind. However, we should avoid getting too complicated in
our moral reasoning. Buddhist morality is a very simple situation of how
your actions directly affect other sentient beings. A hamburger is not a
sentient being. The Buddhist vow for monks and nuns is very explicit in
saying that they may not eat meat if they asked for the animal to be
killed, or if they knew the animal was specifically killed for them. A
common erroneous belief is that Buddhists are vegetarians. This simply
is not true for all Buddhists.


\subsection{Diet}
\label{\detokenize{saints:diet}}
\sphinxAtStartPar
Some Mahayana traditions do have a custom of eating only vegetarian food
on retreats, but this is not true for Theravada traditions. Curiously,
it has become a tradition for Theravada retreats in the West to be
vegetarian. One theory is that Goenka, a vegetarian and former Hindu,
set the pattern for non\sphinxhyphen{}monastic lay retreats in the West. The first
well known Theravada lay teachers in the West had attended Goenka
retreats in India. They patterned their retreats after Goenka retreats.
Another curiosity is that most of the key people responsible for
establishing the vegetarian tradition at \sphinxstyleemphasis{vipassana} retreats are not
vegetarians.

\sphinxAtStartPar
When Sayadaw U Pandita first came to the West in 1984, he was quite
concerned about the vegetarian diet served at the Insight Meditation
Society. He conceded that concern for the welfare of animals was
commendable, and that eating meat was unhealthy, as well as ecologically
questionable. However, he was very strongly against his students
becoming vegetarians for the first time when they go on retreat. He
observed that it takes time for the body to become adapted to a
vegetarian diet, and an initial effect would be a loss of energy. High
energy is a key factor of enlightenment and is very important in his
style of practice. He was so concerned about this that he arranged for
some of his students, who were having problems with the IMS diet, to be
served meat prepared by volunteer cooks outside the IMS kitchen.
Vegetarianism is so firmly entrenched at IMS that despite U Pandita’s
public suggestions the Board of Directors never even considered changing
the diet.


\subsection{Celibacy}
\label{\detokenize{saints:celibacy}}
\sphinxAtStartPar
Another misconception, that probably comes from the Hindu tradition, is
that celibacy is essential for enlightenment. Celibacy is required for
monks and nuns primarily to simplify their lives and focus on their
practice. Also, during intensive meditation retreats, celibacy is
required for laypeople for the same reason. However, Buddhists in daily
life situations are not required to be celibate. Buddhists are required
to avoid sexual misconduct, but it is not clear what this means in
California. Many Buddhist teachers suggest that people should not engage
in sexual activities which result in anyone suffering. This is a vague
standard that could be logically argued to be a total prohibition of
sex, or an invitation to free love. Perhaps it is sufficient to simply
practice continuous mindfulness in situations where the issue of sex
arises.

\sphinxAtStartPar
The purpose of Buddhist precepts is pragmatic in that they are directed
at achieving a quiet mind. If your sex life is causing mental agitation
in your meditations, you should change your behavior. While on retreat,
you should avoid letting your senses wander to sexually stimulating
objects and direct to your attention to meditation objects. In the daily
life situation this may result in becoming a horny, neurotic celibate
which may not be good for your practice. It is better to be simple and
direct in your views, instead of getting involved in elaborate logical
reasoning about right and wrong.


\subsection{Asians and enlightenment}
\label{\detokenize{saints:asians-and-enlightenment}}
\sphinxAtStartPar
You would also be well advised to ignore rumors that enlightenment is
easy for Asians and difficult or impossible for Westerners. I have not
seen any good statistics on this, but it is generally conceded that
Asians are more likely to be successful in their practice. There is a
definite advantage in being raised in a Buddhist culture simply because
faith is a very positive contribution to the practice. Also a positive,
receptive attitude for taking instructions from monks has a beneficial
effect on the practice.

\sphinxAtStartPar
Despite these advantages some Westerners get enlightened quickly, and
some Asians take a long time. Your situation is your situation, and
there is no benefit in comparing yourself to anyone else. All you need
to do is to follow the instructions diligently and cultivate a profound
examination of the present moment.


\subsection{Sitting on the floor}
\label{\detokenize{saints:sitting-on-the-floor}}
\sphinxAtStartPar
Asians who have adapted their muscles and bones to sitting on the floor
can sit erect and cross\sphinxhyphen{}legged much more comfortably than Westerners.
There is some advantage in maintaining a better energy level in
meditation by sitting cross legged on the floor. You would be well
advised to try this out and progressively develop a greater ability to
do this. However, most Western teachers do not believe it is worth going
through a great deal of pain and suffering stretching out your body just
to do this. There are many people who have attained enlightenment who
could only sit in chairs. However, most teachers advise against resting
the back against the chair or attempting the lying meditation until your
practice becomes quite advanced.

\sphinxAtStartPar
There are many Asian teachers who are unaware how difficult it is for
Westerners to adapt their bodies. They can be very insistent that you
sit on the floor. Just do not worry too much about this. If you can’t
sit reasonably comfortably on the floor, practice in Asia may not be for
you.

\sphinxAtStartPar
One time Ram Dass went to Burma with the intention of doing a
three\sphinxhyphen{}month \sphinxstyleemphasis{vipassana} retreat. He was determined to follow his
teacher’s instruction to sit on the floor, and although a cushion would
have been permissible, he did not feel that sitting on one would have
been appropriate for his image. After about a month, he was on the verge
of reattaining the experience of \sphinxstyleemphasis{deep insight} that he had reached on
psychedelics. His back had gone out, and he was experiencing great pain.
Pain in this phase of the practice is usually exaggerated. He continued
to sit on the floor without a cushion enduring great pain. Then a
message came that his stepmother was ill, and he took this excuse to
escape the retreat. His teacher told him to continue his retreat, as he
was on the verge of a major breakthrough in his practice. The pain he
was experiencing was so great that he declined his teacher’s advice.
Since then, he has never had the courage to again undertake such a
rigorous commitment.


\subsection{Intelligence and enlightenment}
\label{\detokenize{saints:intelligence-and-enlightenment}}
\sphinxAtStartPar
Other things you need not worry about are lacking intelligence and
education. One of Dr. Jack Engler’s research subjects was a moron.
Although it took this lady a very long time to retain the meditation
instructions, once she understood them she very quickly became
enlightened. She was still a moron, but at least she was an enlightened
moron.

\sphinxAtStartPar
There is one negative aspect to being intelligent or highly educated.
The problem is that discursive and analytical thinking has a very
negative effect on being able to develop a calm and concentrated mind.
Intelligent and educated people are very prone to thinking about things.
The essence of \sphinxstyleemphasis{vipassana} meditation is to develop a profound
examination of objects of consciousness. Thinking is not observation.
The more advanced your practice becomes, the less logical or reasonable
your objects of perception become. It is essential to develop the
ability to profoundly examine and penetrate into objects of meditation
without thinking about them. Intellectuals, as a group, progress in
meditation more slowly and are less likely to be successful. However, I
know several people who have a Ph D, MD or other highly educated status
and have become enlightened.

\sphinxAtStartPar
Intelligence and education can have some advantages for a meditator such
as understanding the potential of meditation and cultivating
philosophical understanding, but you must develop the ability to suspend
intellectual processes during the practice. The Buddha always invited
people to come and see for themselves and challenge the things he said.
Buddhism places a very strong emphasis on developing a very logical
philosophy to support the practice.

\sphinxAtStartPar
Arthur C. Clarke is a scientifically oriented science fiction writer,
and he has had significant success in predicting future trends. He has
said that Buddhism will be the only one of the worlds major religions
which will survive in the distant future. The reason is that the basic
authority for other religions is faith. The basic authority for Buddhism
is a logical process that can be verified by anyone who takes the time
to investigate it.


\subsection{Conclusion}
\label{\detokenize{saints:id1}}
\sphinxAtStartPar
The traditional belief is that the longer the time it has been since the
time of the Buddha, the more difficult it is to attain enlightenment.
However, the recent revival of practice in Southeast Asia indicates that
the operant factors are qualified, enlightened teachers, a safe secluded
environment to practice and following instructions exactly. Being overly
obsessed with precise methods, grace, past lives, complicated morality,
sitting on the floor, or intellectual activities can be very detrimental
to success in the practice of meditation. Times of peace, prosperity and
availability of the teachings of the dharma are historically rare, and
anyone would be wise to take full advantage of this opportunity.


\section{The Embarrassment of Enlightenment}
\label{\detokenize{saints:the-embarrassment-of-enlightenment}}
\sphinxAtStartPar
Enlightenment has always been an embarrassing subject for Buddhism.
Enlightenment is embarrassing because it is the most important status in
Buddhism, and it is difficult to determine if someone is enlightened. At
one time or another, almost everyone in Buddhism will be embarrassed
about enlightenment. Some are embarrassed because they are not
enlightened, some because they are enlightened, and others are
embarrassed because they do not know whether or not they are
enlightened. A few reasons why people become embarrassed about
enlightenment are:
\begin{itemize}
\item {} 
\sphinxAtStartPar
They think they cannot become enlightened.

\item {} 
\sphinxAtStartPar
They assume that the enlightened will conform to their arbitrary
models.

\item {} 
\sphinxAtStartPar
They assume that the enlightened will seem reasonable and rational.

\item {} 
\sphinxAtStartPar
They assume that the enlightened will conform to standards of the
Theravada Commentaries.

\item {} 
\sphinxAtStartPar
They want to be enlightened and are not.

\item {} 
\sphinxAtStartPar
They think they are enlightened and are not.

\item {} 
\sphinxAtStartPar
They do not know whether or not they are enlightened.

\item {} 
\sphinxAtStartPar
They are enlightened, but do not want to say that they are
enlightened.

\item {} 
\sphinxAtStartPar
They are partly enlightened, but not fully enlightened.

\item {} 
\sphinxAtStartPar
They think certain people must be or cannot be enlightened because of
the amount of time they have meditated.

\end{itemize}


\subsection{They think they cannot become enlightened.}
\label{\detokenize{saints:they-think-they-cannot-become-enlightened}}
\sphinxAtStartPar
After becoming aware of the possibility of enlightenment, most people
will go through a period of thinking that they cannot become
enlightened. Some may have only brief moments of doubt, but others
unfortunately, may become obsessed with it. I think it is helpful to
make a realistic appraisal about the possibility of enlightenment.

\sphinxAtStartPar
The Buddha said that there are four basic types concerning the amount of
time and effort it takes to become enlightened. Some will find it easy
and will quickly become enlightened. Some will find the practice easy,
but it will take them a long time to become enlightened. Some will find
the practice very difficult, but they become enlightened quickly. Some
will find the practice difficult and it will take them a long time. The
Buddha also mentioned that there are some people who could practice all
their lives and have the Buddha for a teacher, but will not become
enlightened. Fortunately this latter group is very small.

\sphinxAtStartPar
People raised in a Buddhist culture take great comfort in many
assurances that even if they do not attain enlightenment in this
lifetime, there are things they can do to assure enlightenment in future
incarnations. Ordaining as a monk or a nun, even for a brief period, is
said to assure attainment of enlightenment. Supporting monasteries,
monks and nuns results in merit which brings enlightenment. Virtuous
actions such as keeping the five precepts, supporting parents and
meditation contribute to attainment of enlightenment. Many Buddhists have
no expectation of attaining enlightenment in this lifetime, but
diligently do things that contribute to future enlightenment.

\sphinxAtStartPar
It is doubtful that expectation of enlightenment in future lifetimes is
going to be a major motivation for most Westerners, and most will want
to know what their chances are for this lifetime. I have not seen any
studies which would answer this question, but my sense of it is that
most people who are properly motivated will be successful. Even so,
almost everyone has some degree of doubt at one time or another.

\sphinxAtStartPar
The problem with doubt is that if it becomes obsessive, it can be
self\sphinxhyphen{}fulfilling. The most effective method for dealing with doubt is
simply to become mindful of your doubt. When you observe that you are
experiencing doubt, you simply say to yourself, “I am having doubt”, and
then go on with your meditation practice, study or whatever you were
doing. Be careful not to cultivate excessive doubt or you will create a
mind state where you will not be able to objectively evaluate your
potential. The Buddha always encouraged questions and investigation, but
he warned about the danger of excessive doubt. If you ever find yourself
having obsessive doubt, it is probably better to discuss it with someone
rather than to trying to suppress it.


\subsection{Arbitrary models}
\label{\detokenize{saints:arbitrary-models}}
\sphinxAtStartPar
One of the things that can contribute to excessive doubt is failing to
see that there are people who have been successful in becoming
enlightened. There have been many successes, not only in the time of the
Buddha and Asia, but in the present, here in the West. Usually, people
fail to see this because they cling to false models of enlightenment.

\sphinxAtStartPar
Misunderstandings about what the enlightened can and cannot do, or
should and should not do, have existed since the time of the Buddha.
About ten percent of the Dhammapada is comprised of poems that the
Buddha composed when explaining or commenting on the actions of
Arahants. People would come to the Buddha complaining that a blind
Arahant was stepping on ants, and the Buddha explained that the
Arahant’s mind was free of the intent to kill. A nun was raped and not
upset, so some assumed she must have enjoyed it, but the Buddha
explained that she was an Arahant and was beyond sexual craving. A monk
had an odd way of jumping over streams, but the Buddha said that
although he was an Arahant, his style of jumping was the result of
previous lifetimes.


\subsection{Conforming to ideals}
\label{\detokenize{saints:conforming-to-ideals}}
\sphinxAtStartPar
A natural tendency is to assume that since enlightenment is an ideal,
then people who are enlightened will live up to your ideals. You might
expect that the enlightened would have haloes, be beautiful, graceful,
pleasing, healthy, intelligent, neat, reasonable and have psychic
powers, but none of these are standards for enlightenment. Now, as in
the time of the Buddha, people fail to recognize the enlightened around
them because they have artificial expectations. The Buddha spoke of this
in the Dhammapada:

\begin{DUlineblock}{0em}
\item[] \sphinxstylestrong{64}
\item[] Fools may live all their lives
\item[] In the company of the enlightened
\item[] And still not see the dharma.
\item[] As a spoon does not taste soup.
\end{DUlineblock}

\begin{DUlineblock}{0em}
\item[] \sphinxstylestrong{65}
\item[] The wise may know the enlightened
\item[] For only a moment
\item[] And still see the dharma
\item[] As the tongue tastes soup.
\end{DUlineblock}

\sphinxAtStartPar
During the years that I was doing intensive practice at the Insight
Meditation Society, I was surprised that a number of people thought that
none of the teachers were enlightened, or that no one ever gets
enlightened at IMS. I should explain that although retreats are
conducted in silence, people are allowed to talk for half a day at the
end of short retreats. I would take advantage of these brief periods to
find out all sorts of things.

\sphinxAtStartPar
My discussions with people indicated that many were using all sorts of
standards for evaluating enlightenment that had nothing to do with
enlightenment. The most common assumption would be that their
personality would fade away. One of the surprising results of Brown \&
Engler’s research into changes after the first level of enlightenment
was that personality profiles became exaggerated. They gave a series of
personality tests before and after three months of intensive meditation.
The subjects that the teachers suggested may have attained the first
level of enlightenment had higher scores after the retreat on many
characteristics measured.


\subsection{Belief in self}
\label{\detokenize{saints:id2}}
\sphinxAtStartPar
Perhaps people expect personality to fade away because the belief
in self is supposedly lost after the first level of enlightenment. It
might be more accurate to say that the old belief in self is replaced
with a new belief in self. Self definition tends to change from being
someone who is in control of the situation to seeing self as being a
process of cause and effect. One way of describing this is to say that
the view of self is a collection of natural, selfless processes which
obey natural laws. Another insight into how the definition of self
changes is a marked tendency to define self as an energy pattern in some
psychological tests.


\subsection{Psychic powers}
\label{\detokenize{saints:psychic-powers}}
\sphinxAtStartPar
A common expectation that people have is that the enlightened will have
psychic powers. The Buddha’s discourses contain many stories of the
Buddha, Arahants and Non\sphinxhyphen{}returners manifesting psychic powers such as
healing, mind reading, knowledge of the past \& future, teleportation,
identic memory flying and other phenomena. There are even many stories
in modern times of people who have manifested psychic powers.

\sphinxAtStartPar
One of my teachers, Dipa Ma Barua, was trained to develop psychic powers
by Munindra while she was at the Mahasi Sayadaw monastery in Burma.
Munindra had a book which described the methods to develop all the
standard psychic powers. After Dipa Ma had attained the third level of
enlightenment, he instructed her to develop all the powers in the book
She very quickly was able to master and demonstrate all of them.

\sphinxAtStartPar
Dipa Ma was following the instructions of her teacher to develop psychic
powers, but she thought that it was not appropriate to make use of them.
The manifestation of psychic powers can very easily take on the air of a
circus performance, and the importance of enlightenment can get lost.
After mastering all of the powers, she renounced all of them except the
power of \sphinxstyleemphasis{metta. Metta} is the development of a profound feeling of
unconditional love. One of the reasons that Dipa Ma was such an
outstanding teacher was that coming into her presence was like entering
a tangible field of love. Her students were inspired to work very hard
on their practice for her.

\sphinxAtStartPar
Although Dipa Ma had ostensively renounced psychic powers, she would
spontaneously use some of them in the course of teaching. From time to
time she would ask me if I had a specific experience in my practice that
day, and then give me specific instructions regarding it. Sometimes she
would give me special instructions regarding phenomena which would occur
within a day after she gave them. On a few occasions, I was aware that
she was doing healings for people that came to see her, but it seemed
that there was an effort being made not to advertise that this was
happening.

\sphinxAtStartPar
Several of my teachers would occasionally manifest a psychic awareness
of things that I had been thinking about, or that happened to me that
related to my meditation practice. On many occasions, I have been given
special instructions and comments concerning my practice that were both
unusual and exceptionally appropriate. These occasions are not arranged,
and they arise and pass very quickly. Most of the times they are done in
such a subtle way that even an interpreter may be unaware that anything
unusual is happening. Many of my friends have had similar experiences,
so the manifestation of teaching powers is very common, but not public.

\sphinxAtStartPar
Psychic powers are generally regarded as a dangerous sidetrack, and most
Buddhist teachers will actively discourage their students from trying to
develop them. The time and effort to develop psychic powers could be
better used to develop enlightenment. Also, it is very easy for people
to have a natural tendency to be entranced with powers. A teaching
situation could very quickly degenerate into a sideshow if too much
emphasis were placed on manifesting powers.

\sphinxAtStartPar
An occasional random psychic experience can add interest and authority
to a teacher if it is done appropriately. There is some danger that
people will choose teachers on the basis of manifested powers and ignore
even better teachers who are more discreet. People who will not
recognize someone as being enlightened unless she/he manifests psychic
powers are making a mistake. There is nothing in the list of
characteristics of enlightenment that requires the possession of psychic
powers.


\subsection{Higher levels of enlightenment}
\label{\detokenize{saints:higher-levels-of-enlightenment}}
\sphinxAtStartPar
I am not aware of any psychological tests that might support this, but
the mind becomes more sensitive and changeable with each level of
enlightenment. It would seem that this is partly a result of being able
to let go of a mind state more quickly. If you are able to let go of one
mind state, then almost instantly another mind state will arise. If this
ability is combined with the ability to consciously perceive processes
which were previously unconscious, then the mind becomes more volatile.

\sphinxAtStartPar
It is this increase in sensitivity and volatility which makes each level
of enlightenment more difficult to attain. It takes essentially the same
degree of concentration to attain \sphinxstyleemphasis{deep insight} into each new level
of enlightenment and to progress to attainment. Because the mind is more
sensitive, there is a greater probability that a mind object will arise
which will disrupt the concentration. A similar phenomenon occurs when
in the final phases of equanimity just before the attainment of Nirvana,
when new deep areas of unconscious processes are encountered. Usually
meditators working on higher paths progress very rapidly, a few days or
hours, from \sphinxstyleemphasis{deep insight} to the attainment of Nirvana and the higher
level of enlightenment. However, it is not uncommon for some to progress
very rapidly to final phases of the path in equanimity, and then spend
long periods, even years, in the final phase of equanimity just before
experiencing Nirvana. The higher the level of enlightenment being worked
on, the more likely this problem will occur. If they stop intensive
practice before attaining Nirvana, they will most likely lose the
progress they made and have to develop \sphinxstyleemphasis{deep insight} again on their
next retreat. They will then have to progress to where they left off on
their previous retreat. I suppose we can call this one of the
embarrassments of enlightenment.


\subsection{Advanced Training}
\label{\detokenize{saints:advanced-training}}
\sphinxAtStartPar
The proper development of advanced forms of practice is the key to
attaining higher levels of enlightenment. The availability of this
training might be a factor to consider if you are thinking of doing long
term intensive practice. For example, the structure of the schedule at
IMS almost precludes proper preparation for higher levels of
enlightenment by closing for two weeks shortly after the annual
three\sphinxhyphen{}month retreat.

\sphinxAtStartPar
In advanced forms of practice, the teacher should observe carefully the
student’s practice after the first level of enlightenment is reached.
There should be signs that the student is spontaneously having repeated
experiences of Nirvana. This requires teachers with considerable skill
and experience as there are seven types of meditation phenomena which
can be mistaken for Nirvana. Once it is established that the student is
having experiences of Nirvana, then the teacher should begin a series of
exercises to develop a great proficiency in controlling this experience.
It can take several months of continuous intensive practice to develop
an acceptable minimum level of proficiency to maintain this ability in
daily life after leaving intensive practice. The best time to begin
developing this ability is as soon as possible after the first level of
enlightenment has been obtained. When adequate proficiency has been
developed, the teacher instructs the student to renounce having the
experience of Nirvana. \sphinxstyleemphasis{Deep insight} into a higher level of
enlightenment will soon occur.


\subsection{The probability of attaining higher levels}
\label{\detokenize{saints:the-probability-of-attaining-higher-levels}}
\sphinxAtStartPar
I haven’t seen any statistics on the attainment of higher levels of
enlightenment, and structuring an accurate estimate is difficult. Since
adequate advanced training is not generally available in the West, it
makes it difficult to estimate the probability of attaining higher
levels of enlightenment I have had little success in getting even rough
statistical estimates from a number of Asian meditation masters. The
Southeast Asian mind just does not seem to think statistically. Usually,
no matter how carefully I structure my questions, the answers boil down
to either yes or no.

\sphinxAtStartPar
Having said this I can say that it seems that the majority of people who
attain the first level of enlightenment and continue their practice can
attain the second level of enlightenment. Some people who attain the
first level of enlightenment are satisfied, and simply do not pursue
their practice further. Some people who have not had advanced training
will spontaneously develop \sphinxstyleemphasis{deep insight} to a higher path.
However, only a small percentage of people who attain the second level
will attain the third level. There also seems to be a big drop between
the number who get to the fourth level from the third. The best we can
determine from studying the Buddha’s discourses is that there seemed to
be a similar distribution of attainment in the Buddha’s time. The main
difference is that a larger percentage of people in the time of the
Buddha seemed to be attaining the first level. Although Arahants were
more plentiful in the time of the Buddha, it seems that there were not
proportionally more of them among those who attained the first level.


\subsection{Reasonable and rational?}
\label{\detokenize{saints:reasonable-and-rational}}
\sphinxAtStartPar
The most frequent reason that people would come to the Buddha with
complaints about Arahants was that they felt that the Arahants were not
being reasonable or rational. This is a variation on the subject of
expecting the enlightened to live up to arbitrary models, but I feel
this subject needs some special attention. Meditation is
psychotherapeutic, but not all neuroses will be eliminated until the
final phase of enlightenment, and some characteristics of personality
may become exaggerated by enlightenment.

\sphinxAtStartPar
One of the most appealing prospects to Westerners is that Buddhist
meditation offers an effective and inexpensive form of psychotherapy.
Buddhism is primarily entering the mainstream of Western culture through
psychotherapists. Usually, psychotherapists are the largest professional
group at Buddhist meetings, meditation retreats or events. A poll at one
three\sphinxhyphen{}month retreat that I attended found that half of the retreatants
were psychotherapists.

\sphinxAtStartPar
Buddhist meditation is certainly psychotherapeutic, but there are some
problems with regarding it as a form of psychotherapy. The first is that
too much emphasis on psychological phenomena will derail the process of
simply observing phenomena without trying to change anything. The second
is that in terms of psychotherapy, results are random, and it is unwise
to expect specific psychotherapeutic results from the practice. The
third is that the level of the unconscious where mental illness
originates is the deepest level. This level is not accessed until the
final level of enlightenment.

\sphinxAtStartPar
Although there are many reports of neuroses, phobias and obsessions
which have been spontaneously cured during meditation, usually there is
no way to predict in advance if and when they will be cured. Even
general personality changes as a result of attaining different levels of
enlightenment are erratic. Some people have a very profound change in
personality as a result of attaining the first level, and others do not.
In terms of therapeutic cures, you can expect that the first level would
result in a significant reduction in irrational behavior which is
immoral. It is unlikely for people who have attained the first level of
enlightenment to be murderers, liars, thieves, adulterers, or
alcoholics. A common report is that although neuroses can exist at the
first level of enlightenment, there is a sense of space behind the
neuroses. Although neurotic anger may be expressed, there is an
impression that the anger can be observed as an impersonal process, and
stream\sphinxhyphen{}winners are not 100\% lost in it. This is difficult to quantify or
qualify, but my guess is that a study of outbursts of anger would show
that they are shorter and that amends for outbursts would more likely be
made.

\sphinxAtStartPar
A paradox here is that a stream\sphinxhyphen{}winner is likely to be more sensitive,
volatile and spontaneously expressive. The classic Commentaries of
Buddhism say that both lust and aversion are not reduced until the
second level of enlightenment, and not eliminated until the third level.
There may be some reduction of lust and anger\sphinxhyphen{}aversion at the first
level, but it might be counter balanced by the volatility of
personality.

\sphinxAtStartPar
As I mentioned earlier, personality profiles after the first level of
enlightenment become exaggerated, and this is probably because of
sensitivity and volatility. Personalities seem to become more
exaggerated with each level of enlightenment, and I haven’t met an
Arahant that wasn’t a unique caricature. It is difficult to draw the
line between neuroses and characteristics of personality. In any case,
it seems that neurotic behavior may become exaggerated, although I am
unaware of anyone becoming generally more neurotic as a result of
meditation.

\sphinxAtStartPar
My teacher, Sayadaw U Pandita, once said, “Because of habit patterns, it
is possible for an Arahant to be obnoxious. However, the difference with
Arahants is that, if it is pointed out to them that they are being
obnoxious, they are capable of reflecting on situations and changing
their behavior.” I am certainly glad he said that, because U Pandita can
be quite obnoxious sometimes. His only interest seems to be that people
get enlightened, and he does not seem to care if people like him. There
are a number of things he will do, especially in private interviews,
that are calculated to irritate people if they are not being mindful. He
once confided to students of his, who were teachers, that he frequently
pretends to totally ignore a student during their interview by reading a
book, or doing something else. He said that this was a pretense, and he
really is watching them very carefully. Sometimes he is quite sarcastic
or brutal in his comments about reports on practice that people give
him. At his 1984 IMS retreat, 25\% of the class of teachers and advanced
students dropped out of the three\sphinxhyphen{}month\sphinxhyphen{}course because his teaching was
too difficult for them.


\subsection{Being rational}
\label{\detokenize{saints:being-rational}}
\sphinxAtStartPar
Another thing that U. Pandita said was, “Until the mind is purified of
all defilements, it is possible that even one defilement can be
activated and overcome consciousness.” This is worth remembering when
evaluating how rational the enlightened seem. Even Non\sphinxhyphen{}returners have
the defilements of: subtle craving for material realms, subtle craving
for immaterial realms, restlessness, conceit and the last veil of
unknowing. It is theoretically possible that although great progress has
been made in purifying the mind, individual circumstances may
chronically aggravate a particular neurosis. When we evaluate the
behavior of the enlightened, we should keep in mind that rationality and
reasonableness are not standards for evaluating enlightenment. The
tendency is for the personality to be exaggerated, and individual
characteristics should be considered.


\subsection{Standards of the Theravada Commentaries}
\label{\detokenize{saints:standards-of-the-theravada-commentaries}}
\sphinxAtStartPar
One of the most embarrassing controversies in Theravada Buddhism is
whether or not a stream\sphinxhyphen{}winner, or higher, would ever break a precept. I
touched on this subject in the chapter on Saints. The precepts are: 1)
Not to kill any sentient being; 2) Speak only the truth and never lie;
3) Not to steal or take anything which is not freely offered; 4) Not to
engage in sexual misconduct; 5) Not to take substances which dull the
consciousness.

\sphinxAtStartPar
I have mentioned before that the Southeast Asian mind seems to think in
terms of generalities, and does not seem to think in terms of
percentages or probabilities. Classical scholars give no wiggle room in
terms of adherence to precepts. They say that a stream\sphinxhyphen{}winner would not
take even a small sip of alcohol. One prominent scholar said that if a
stream\sphinxhyphen{}winner took a vow to fast and then took a drink of milk, the
solids would separate and only clear water would be consumed.

\sphinxAtStartPar
For years, I have had an opportunity to closely observe teachers and
meditators whom I believe have attained at least the first level of
enlightenment. They are mostly laypeople and Westerners, which is a
fundamentally different situation for monks and nuns who have a primary
duty of following precepts as a commitment to continuous mindfulness. In
my opinion, they have a very strong tendency to be highly moral and many
make a sincere effort to follow the precepts. Even those who make no
specific effort to follow the precepts are intuitively inclined to
follow them.

\sphinxAtStartPar
However, I have also observed that there have been occasions when,
because of neuroses, personality characteristics and cultural
conditioning, they would violate all or some of the five basic precepts.
At the Insight Meditation Society it was decided that overwhelming
infestations of cockroaches and flies were to be dealt with by
poisoning. Many would take a glass of wine at a party and some would
take psychedelic drugs on occasion. Some have failed to report income on
their tax returns. Some have answered questions falsely in meditation
interviews. Many have had sex with people they were not married to. Some
have taken food from the kitchen, that was not freely offered and eaten
it after the noon meal, while on retreat. Some have consciously or
unconsciously swatted mosquitos. According to the Commentaries, it would
be impossible for a stream\sphinxhyphen{}winner to do any of these things.


\subsection{Differences in Eastern and Western logic.}
\label{\detokenize{saints:differences-in-eastern-and-western-logic}}
\sphinxAtStartPar
There is a fundamental difference between Asian and Western scholarship,
and that is the primacy given ancient writings. A common view in all
Asian countries is that ancient writings are superior to modern
writings. A common Asian belief is that the world and culture is in a
state of continuous decline from a previous Golden Age when people lived
thousands of years in perfect harmony and health, and there was no war,
crime, disease nor insanity. It is believed that scholars had perfect
understandings of things that are no longer understood properly. It is
also believed that this decline will continue in the future when things
will get worse until eventually a new Golden Age will come.

\sphinxAtStartPar
Traditional Asian scholars will give the highest authority to the oldest
books. As it was in the West during the middle ages, a statement in an
ancient book is given higher authority than empirical observations. The
scientific method is now entering Asian culture, but it has not yet
affected traditional scholarship. Statements in the Buddhist
Commentaries were based on logical deductions from Buddhist psychology,
and in some cases on personal observations or conclusions of the ancient
scholars. If someone behaves contrary to the description of enlightened
behavior in the Commentaries, then traditionally that person is regarded
as not being enlightened.


\subsection{Clear thinking and correct thinking}
\label{\detokenize{saints:clear-thinking-and-correct-thinking}}
\sphinxAtStartPar
It is unthinkable to a traditional Buddhist scholar that the Buddha
would ever make a mistake. The Buddha believed that one cause of
earthquakes was that land is floating on a sea and winds on the sea
cause big waves which shake the land. A scholar explained to me that
there are four elements in the Buddhist scheme of things: fire, earth,
air and water. The wind is regarded as energy in this system. Since
earthquakes are a release of energy, the Buddha was correct in saying
that the wind was the cause of earthquakes. This explanation seems to me
to have more rationalization than logic.

\sphinxAtStartPar
I believe that it is more important that the Buddha thought clearly,
rather than he thought correctly. If a computer is supplied incorrect
information, it will supply incorrect answers. This is no reflection on
the ability of the computer to function correctly. On a number of
occasions it seems to me that Arahants have made mistakes based on
incorrect information. One Arahant instructed his followers to build a
huge pagoda in California which was far beyond their financial
resources. Another Arahant said that the reason there were commercials
in the middle of movies on TV was that people were unable to concentrate
so long without a break. To me, it is more important that Arahants are
able to change their minds when presented with correct information, than
that they believe erroneous facts. It seems reasonable to me that if an
Arahant can make a mistake because of incorrect information, then a
Buddha could too. This is contrary to traditional Buddhist belief.


\subsection{The Scientific method and Buddhism}
\label{\detokenize{saints:the-scientific-method-and-buddhism}}
\sphinxAtStartPar
Whenever Buddhism enters a new culture, it takes on some of the
characteristics of that culture. The scientific method of questioning
and testing beliefs with new information is a very strong characteristic
of Western culture. The Buddhism that takes root in the West is very
likely to have a basis in tested and proven facts and concepts. Although
some cherished beliefs are likely to fall, Buddhism should emerge from
this process much stronger, and with deep roots in our culture. It will
probably be the only ancient religion that can largely survive the
rigorous questioning and testing of the scientific method, and become
integrated with mainstream thinking. We are very likely to come up with
a different definition as to what enlightenment is, and how to judge
whether or not someone is enlightened. It is also possible that
enlightenment may become as widely a cherished ideal as getting a good
education. Perhaps we may enter a new Golden Age.


\subsection{They want to be enlightened}
\label{\detokenize{saints:they-want-to-be-enlightened}}
\sphinxAtStartPar
One of my teachers was a student of Anagarika Munindra, and he spent
many years studying with him in India. During this time Munindra would,
making an Asian attempt at statistics, frequently mention in his talks,
“It only takes ten weeks to get enlightened.” After a few years of
hearing Munindra say this, he was getting quite worried. Eventually he
was successful and has become an outstanding teacher. However, a
characteristic of his teaching is to make no here and now type
references to people attaining enlightenment.

\sphinxAtStartPar
Once it becomes obvious that some people are getting enlightened, there
is a risk that those who are not enlightened will become embarrassed
because they are not. They may become embarrassed because they may have
done long periods without success. Some people might be embarrassed
because they have not become enlightened in a ten\sphinxhyphen{}day retreat. Some
might be embarrassed because they believe their friends have become
enlightened, and they have not, and some people might be embarrassed
because they are status conscious achievers.

\sphinxAtStartPar
Even in the time of the Buddha, there were great variations in the
amount of time that it took people to attain enlightenment The
charioteer who took the Buddha\sphinxhyphen{}to\sphinxhyphen{}be on his first trips into the world,
later became one of the Buddha’s monks. However, he did not attain even
the first level of enlightenment until after the Buddha died. The Buddha
taught for 45 years, and the charioteer had the Buddha for a teacher.
Another case from the time of the Buddha was of a monk who practiced
continuously for 65 years before attaining the first level of
enlightenment. The fastest student of the Buddha was a man who attained
all four levels as the Buddha gave him the following brief discourse:
“In the seen, there is just the seen. In the heard, there is just the
heard. In taste, there is just taste. In the felt, there is just the
felt. In smell, there is just smell. In the thought, there is just the
thought.”

\sphinxAtStartPar
Fortunately, Buddhist etiquette makes enlightenment a private, personal
matter. I will be discussing this in greater detail in the chapter on
\sphinxstyleemphasis{The Etiquette of Enlightenment}. Many teachers will not even directly
discuss with their students whether or not they have attained
enlightenment. However, it seems common all over the world that people
who have done long periods of practice together, and have been
successful, will have an intuitive awareness of which of their friends
have been successful. They will discretely discuss their attainments
with close friends that they are sure have had similar experiences.


\subsection{Enlightened teachers}
\label{\detokenize{saints:enlightened-teachers}}
\sphinxAtStartPar
Except when it relates to status of a teacher, the enlightened do not
seem to feel that there is any particular social status in their
enlightenment. I haven’t seen any trend among the enlightened to choose
their friends, or even spouses, based on enlightenment. The Buddha spoke
of the enlightened as having a high status, and he urged both
enlightened and unenlightened to hang out with the enlightened. There is
wisdom in this advice, but the enlightened do not seem to have a
particular inclination for it. Anyone who wants to attain enlightenment
with the goal of achieving some social status, other than being a
teacher, is pursuing an illusion.

\sphinxAtStartPar
Actually, there is little guarantee that someone teaching Buddhist
meditation is enlightened. The Buddha indicated a preference that
teachers be enlightened, and most teachers follow this standard before
encouraging their students to teach. However, the Buddha established no
system for qualifying teachers, so there is uneven application of this
standard in Buddhism as a whole. Some people who are teaching think that
they are enlightened, but are not, and are empowering their students to
teach. Although there should be some concern about this situation, I
personally am not too concerned, because in the long run, more people
will be drawn to the enlightened teachers. Students will get better
results with enlightened teachers.


\subsection{Enlightenment and attainment}
\label{\detokenize{saints:enlightenment-and-attainment}}
\sphinxAtStartPar
There are a couple of other things that should be said about the desire
to attain enlightenment: Enlightenment really isn’t an attainable,
therefore a wrong effort to attain it could block its attainment. I
suppose this statement sounds Zen, but the point that I want to make is
that there is a difference between conventional language and technical
language. Perhaps it is a sloppy rule, but meditation teachers giving
meditation instructions are using technical language. When meditation
teachers are talking about ideas, philosophy, or anything else, they are
using conventional language. A parallel to this could be a physicist
speaking technically about his car being empty fields of energy patterns
and then saying, “Be careful not to run over the garbage can!”

\sphinxAtStartPar
Technically, there should be no effort to attain anything when doing
\sphinxstyleemphasis{vipassana} meditation. It is essential to settle back into a profound
examination of the present moment. Any effort to attain something will
focus your attention on that which does not exist, and you will not be
able to penetrate into the true reality of the present moment. Sometimes
I tell my students, “The only way to get to the next step is to be 100\%
with the step you are on.” Sometimes even the subtlest anticipation of
what is going to happen next in the practice can block progress.

\sphinxAtStartPar
Although technically there should be no effort to attain anything in
\sphinxstyleemphasis{vipassana} meditation, the attainment of enlightenment is a very
worthwhile and difficult objective. It is highly recommended to take on
the attainment of enlightenment as the most important objective of a
lifetime. It is not easy to arrange your life to do a three\sphinxhyphen{}month
retreat, but it is worth doing even if it involves considerable effort
and sacrifice. You should be careful that an attitude of effort to
attain in the conventional sense does not creep into your practice in a
technical sense.

\sphinxAtStartPar
The creeping of unskillful attitudes into meditation practice can be a
major problem. I know several people who have made the attainment of
enlightenment the most important part of their lives, and have done many
long retreats over many years without success. I get the impression that
a common denominator which many of these people have is a quality of
grasping in their personality. I have often suspected that a tendency to
grasp experiences creeps into their practice.

\sphinxAtStartPar
Nevertheless, people who have done long periods of practice without
attainments do feel that the practice of meditation has been beneficial
to them. Even people who have done more than ten years of intensive
practice before attaining the first level of enlightenment feel that the
time they spent has been very worthwhile.

\sphinxAtStartPar
There is one big advantage to taking a long time to attain
enlightenment, and that is people who do so seem to make better
teachers. They are better able to identify with and help people who are
having problems with their practice. One of my teachers likened this to
a monk who went into a village seeking alms with his begging bowl. He
became lost in the village and people kept giving him food. By the time
he found his way back to the monastery, he had enough food to provide a
feast for the other monks.

\sphinxAtStartPar
Regardless of how long it takes to attain enlightenment, it is not
appropriate to compare yourself to others. Some people are going to be
fast, and other people are going to be slow. You are going to be
whatever you are going to be, and it is not a race. Enlightenment is a
noncompetitive activity, and it is possible for everyone to win. One of
my favorite quotes from Munindra is, “The Buddha’s enlightenment solved
his problem. You must solve your problem.” How long it takes someone
else has no relevance to how long it will take you.


\subsection{They think they are enlightened}
\label{\detokenize{saints:they-think-they-are-enlightened}}
\sphinxAtStartPar
One truly embarrassing situation, which is common, is that
sometimes people think they are enlightened, and they are not. There are
a number of reasons people mistakenly think that they are enlightened:
the most common reason is mistaking experiences in meditation practice
for enlightenment; and neurotics may imagine that they are enlightened
for a variety of psychological reasons. Of course psychopaths may simply
claim to be enlightened.


\subsection{The calm and concentrated mind}
\label{\detokenize{saints:the-calm-and-concentrated-mind}}
\sphinxAtStartPar
However, there is value in understanding the common experience of
mistaking experiences in the practice for enlightenment. One of these
experiences is when a meditator first experiences a very calm
concentrated mind. It is said that the first three insights a meditator
has are: 1) The mind is in a state of turmoil; 2) It seems impossible to
stop or control this turmoil; 3) It has always been this way and we did
not know it. After these insights have been experienced, it is possible
that deep concentration can develop. The mind becomes very clear,
powerful and objects can be seen vividly. You can look at a tree and see
it as you never have before. Instead of seeing the tree through a fog of
discursive thought, fear and desire for the past and future, you simply
see a vivid, beautiful image of a tree. You see subtleties of color,
texture and shape in a profound new way. The mind is cool, quiet and
peaceful as you have never experienced before. It is truly a profound,
wonderful state of mind. This is it! This must be the wonderful
indescribable experience of \sphinxstyleemphasis{Nirvana!}

\sphinxAtStartPar
Fortunately, this is not \sphinxstyleemphasis{Nirvana}: \sphinxstyleemphasis{Nirvana} is even better than this.
Sometimes people hold on to this concentrated state of mind for the rest
of the retreat. When they go home, and are confronted with the
complexities of real life, this clear state of mind will shatter like a
pane of glass. The buzzing fog of confusion returns to their minds, and
they are quite likely to become upset or depressed that peace and
clarity have vanished.

\sphinxAtStartPar
This clear concentrated state of mind is part of the normal evolution of
meditation practice. This is actually the development of \sphinxstyleemphasis{samatha}
meditation, or pure concentration of mind. It is the preliminary to
developing true \sphinxstyleemphasis{vipassana} meditation. As I have mentioned
previously, it is necessary to develop concentration before \sphinxstyleemphasis{vipassana}.
The recent rediscovery is that it is possible to simultaneously
develop both \sphinxstyleemphasis{samatha} and \sphinxstyleemphasis{vipassana} has resulted in a revival of
practice in Theravada Buddhism. However, even when \sphinxstyleemphasis{samatha} and
\sphinxstyleemphasis{vipassana} are being simultaneously cultivated, \sphinxstyleemphasis{samatha} develops
first.

\sphinxAtStartPar
The word \sphinxstyleemphasis{vipassana} is sometimes defined as \sphinxstyleemphasis{seeing things clearly}.
However, when we get into a study of the derivation of this word, the
meaning is \sphinxstyleemphasis{to see things clearly with great power and penetrating
examination.} Objects can be seen clearly with just \sphinxstyleemphasis{samatha,} but
with the penetrating power of \sphinxstyleemphasis{vipassana} added to \sphinxstyleemphasis{samatha,} the
true nature of objects can be seen clearly. When the true nature of
objects is seen clearly, they will have the characteristics of change,
of being unsatisfactory and of emptiness. These characteristics will be
seen consecutively, in this order, as the practice evolves.


\subsection{The Progress of Insight}
\label{\detokenize{saints:the-progress-of-insight}}
\sphinxAtStartPar
In \sphinxstyleemphasis{vipassana} meditation, the object of meditation is not
particularly important. What is important is the way the objects are
perceived. The most common primary objects for \sphinxstyleemphasis{vipassana} are
changing objects such as the breath or the body, especially movement of
the body during walking meditation. The most common objects for \sphinxstyleemphasis{samatha}
are fixed visual objects such as colored disks mounted on the wall or
a candle flame. As \sphinxstyleemphasis{samatha} develops, objects become more vivid,
solid and real. Even though the objects of \sphinxstyleemphasis{vipassana} are changing,
the first component to develop is \sphinxstyleemphasis{samatha.} After \sphinxstyleemphasis{samatha} is
adequately developed, a sudden shift occurs in the practice which is
called \sphinxstyleemphasis{deep insight.} At this point the practice technically shifts
into \sphinxstyleemphasis{mahavipassana,} and the primary impression of objects changes
from solid reality to changing discontinuity of objects.

\sphinxAtStartPar
It is \sphinxstyleemphasis{deep insight} that is most commonly mistaken for enlightenment.
Previously the mind had been focused into a one pointed place. When the
mind shifts into the overdrive of \sphinxstyleemphasis{mahavipassana,} suddenly the mind
takes on a perception of space, with objects arising and passing within
space. The mind is flooded with profound insights into the true nature
of reality, the laws of \sphinxstyleemphasis{karma,} and frequently, but not always, there
are lights and visions which reflect and symbolize these insights.
Suddenly, there is a deep experiential comprehension of what before had
been just philosophical understandings of the \sphinxstyleemphasis{dharma.} The change is
so profoundly different and wonderful, that many think, “This certainly
must be enlightenment.”

\sphinxAtStartPar
Although many suspect that they have experienced enlightenment at this
point, there are also reasons to suspect that this is not enlightenment.
Very likely the mind is in a state of revolution and turmoil. The mind
before had been quiet and peaceful, but now it is flooded with
innumerable objects that are constantly changing. But after awhile the
mind again settles down to observing the continuous flow of arising
objects. As the mind settles down to a strong practice, the suspicion
may again arise that enlightenment has been attained. Sometimes the
energy and concentration are strong enough that meditators can sit for
hours without moving.

\sphinxAtStartPar
However, there is another shift in the practice as the emphasis abruptly
changes to the passing of objects from consciousness. The exhilaration
of the practice vanishes. Thoughts and images of death, disease and
decay are likely to arise. The mind is colored with fear, dread and
uncertainty. It is difficult to sit even an hour, and the mind tends to
wander as objects vanish from the mind. After this, the mind and body
seem to become a mass of suffering, and the meditator is willing to let
go of all attainments in the practice, including the idea of
enlightenment.

\sphinxAtStartPar
Again there is a shift in the practice. A great peace arises in the mind
as the meditator is able to observe the flow of empty phenomena passing
through consciousness. As this phase of the practice deepens, equanimity
becomes profound, and the meditator may again be able to sit for hours
without moving. At this time, the meditator may be able to temporarily
experience the quality of mind of an Arahant. The mind is so clear and
powerful, that again thoughts may arise that enlightenment has been
attained. It has not, but it is close.

\sphinxAtStartPar
If intensive meditation practice is interrupted at this point, the
conscious access to formerly unconscious processes will be lost. There
will be some permanent changes in attitude and philosophy as a result of
insights that have occurred. Meditators will retain a profound respect
for the dharma, their teachers and the potential of meditation. They
will have a change in self definition, and will know that only their own
work purifying their minds will be their salvation. Their practice will
be much stronger in daily life, and they will have a strong urge to
return to intensive practice and work on themselves.

\sphinxAtStartPar
If intensive practice is continued at some point, they will have an
experience of \sphinxstyleemphasis{Nirvana. It} is an interesting point to ponder as to
whether \sphinxstyleemphasis{Nirvana} changes the mind, or the experience of \sphinxstyleemphasis{Nirvana}
occurs because the mind has made a fundamental change. In any case,
the first brief glimpse of \sphinxstyleemphasis{Nirvana} represents a fundamental shift in
consciousness. The first stratum of the unconscious mind is now
permanently accessible to the consciousness even after the \sphinxstyleemphasis{high} of
the retreat has worn off. The first level of enlightenment has been
attained. New ways to be embarrassed about enlightenment can now be
experienced.


\subsection{They do not know}
\label{\detokenize{saints:they-do-not-know}}
\sphinxAtStartPar
I have discussed the confusion which may arise as to whether or not
enlightenment has been attained as certain experiences in the practice
occur. Some people will decide that they are enlightened when they are
not. Others will not be able to decide whether or not they are
enlightened. This inability to decide can sometimes continue even after
enlightenment has been attained.

\sphinxAtStartPar
Usually this occurs in situations where teachers do not want to openly
discuss with their students their level of attainment. The primary
reason for this is the general trend to regard enlightenment as a
personal and private matter even between student and teacher. Some
teachers may not be absolutely sure of exactly what level of attainment
their students have made. Other teachers may feel confident that they
can appraise their students progress, but simply do not want to take
time in interviews discussing anything other than meditation
instructions.

\sphinxAtStartPar
Even from the viewpoint of an experienced teacher, it is not always easy
to determine exactly what is happening in some students’ practice. There
is a tremendous variety of individual responses to the same level of
development as well as different ways that people can report similar
phenomena. Sometimes psychological problems arise in the practice which
obscure developments in the practice. However, after teachers have
worked with students for long continuous periods, they usually have a
good idea of their level of development. Whether they are willing to
discuss their opinions with their students depends on the teacher.

\sphinxAtStartPar
In the Mahasi Sayadaw tradition, when teachers feel that their students
may have attained enlightenment, they are given a tape recording of
other students to listen to. The tape recording is of other students
describing their experience of the progress of insights leading up to
enlightenment. The students listening to the tape are told to compare
their experiences to the experiences described and decide for themselves
if they are enlightened. Being told to listen to the tape is regarded as
an opinion of the teacher that they are enlightened, which is probably
why teachers in the West do not follow this tradition. If teachers are
unwilling to discuss their evaluations of students’ practice, then it is
quite possible that the students may be confused about whether or not
they are enlightened.


\subsection{They do not want to say that they are enlightened}
\label{\detokenize{saints:they-do-not-want-to-say-that-they-are-enlightened}}
\sphinxAtStartPar
The number of people who are enlightened, but are embarrassed because
they do not want to say that they are enlightened, is larger than the
group that do not know if they are enlightened. I will be discussing
this more in the chapter on the \sphinxstyleemphasis{Etiquette of Enlightenment,} but here
I want to say something about the feeling of embarrassment. The feeling
of embarrassment seems to come from, at least, these reasons: 1) There
is an expectation that people will have arbitrary expectations for them
to live up to. 2) It is difficult for them to describe what
enlightenment is like. 3) People tend to regard enlightenment as
something that has been acquired, when, in fact, nothing has been
acquired.


\subsection{Living up to expectations}
\label{\detokenize{saints:living-up-to-expectations}}
\sphinxAtStartPar
Back in the 1920’s, there was a circus performer who had an unusual act
of being shot in the stomach with a cannon ball. The cannon ball was an
unusually large hollow steel ball, so it was not exactly like being shot
by a regular cannon, but it was an impressive act. He did have unusually
strong stomach muscles, and when hit by the ball he was propelled
backward into a safety net. One day he was standing in line at a bank,
and an admirer came up and unexpectedly hit him in the stomach. His
stomach muscles were relaxed, and the blow ruptured his liver. He died
from internal bleeding.

\sphinxAtStartPar
As I mentioned previously, people tend to have all sorts of mistaken
ideal standards which they expect the enlightened to live up to. Some of
these standards could be dangerous, but in any event it is better not to
have to deal with them at all. Life is much safer and simpler
for the enlightened without public recognition.


\subsection{Explaining the unexplainable}
\label{\detokenize{saints:explaining-the-unexplainable}}
\sphinxAtStartPar
Another way that life is simpler for the enlightened is that if people
are not aware of their status, they do not have to explain the
unexplainable. Most people who become enlightened have a paradoxical
feeling that everything is fundamentally the same while at the same time
they have a feeling that everything is different. They have a great deal
of difficulty describing what the difference is before and after
enlightenment. Actually, one of the differences is that some processes
which were unconscious are now conscious. This is hard to describe
because all the conscious processes seem to be normal conscious
processes.


\subsection{Progress is measured by loss}
\label{\detokenize{saints:progress-is-measured-by-loss}}
\sphinxAtStartPar
Another thing that is hard to describe is that enlightenment is more
like letting go of something than acquiring something. It is said that
spiritual gain is measured by loss. In the Hindu tradition this is
described as the higher realms of existence having fewer elements. There
are similar descriptions in the Buddhist tradition. In the practice of
meditation, teachers evaluate students’ progress in part by the loss of
certain characteristics. Paradoxically, as the mind becomes simpler, it
can perceive greater complexity.

\sphinxAtStartPar
It is difficult to describe the simplicity of seeing one’s self as a
cause and effect natural phenomenon. This is the result of letting go of
a previous definition of self as being inflexible and in control. The
previous definition involved all sorts of complex assumptions and
rationalizations, and the new definition is a simple natural flow of
energy. It seems better to just be, than to explain, and that is one of
the reasons that the enlightened seek no special status.


\subsection{Partly enlightened}
\label{\detokenize{saints:partly-enlightened}}
\sphinxAtStartPar
A subtle form of embarrassment comes from being partly enlightened but
not fully enlightened. Experiences in deep meditation have given even
stream\sphinxhyphen{}winners a taste of the potential of what it would be like to have
the mind free of defilements. At the same time, being partly enlightened
tends to increase awareness of the existence of neuroses and other
defilements of mind. An awareness of the potential and limitations of
the current state of mind result in humility that is a form of
embarrassment about being enlightened. Trungpa Rinpoche described this
as, “Clarity of mind does not result in a feeling of how good you are
now, but of how stupid you were before.” Even Arahants do not have a
sense of pride about their enlightenment. Conceit is one of the
defilements of mind that is eliminated by attaining the fourth level of
enlightenment.


\subsection{The amount of time they have meditated}
\label{\detokenize{saints:the-amount-of-time-they-have-meditated}}
\sphinxAtStartPar
From time to time people find themselves in an embarrassing situation
because they expect that someone who has done many years of practice
must be either enlightened or very enlightened. The opposite situation
can be true if they assume someone is not enlightened because they have
done very little meditation practice. As I indicated before, there is a
wide variation on how rapidly people progress in the practice. The
Buddha suggested that we could look for the results of being an Arahant
or a Non\sphinxhyphen{}returner when someone has done between seven days and seven
years of practice. In the time of the Buddha, there were cases of people
getting these results in less than a week, and some did not get any
results in much longer periods. The basic message is that we should be
open minded as to whether someone is, or is not enlightened, and it is
only important when evaluating teachers. Certainly, to avoid
embarrassment, you should avoid voicing any opinions to someone, or
about someone, as to whether or not they are enlightened, based on
practice they have done. It is a good idea to be discreet and follow the
guidelines in the chapter on \sphinxstyleemphasis{The Etiquette of Enlightenment.}


\section{The Etiquette of Enlightenment}
\label{\detokenize{saints:the-etiquette-of-enlightenment}}
\sphinxAtStartPar
Since enlightenment is such an important and embarrassing subject, we
should have a proper etiquette. Fortunately or unfortunately, there is
no universal etiquette for enlightenment, but each culture and situation
evolves its own customs. This is fortunate in that it allows each
culture, teacher, monastery or meditation center to evolve practices
which are appropriate for that situation. It is unfortunate in that the
etiquette is mostly unspoken and unexplained. Awkward situations are
likely to happen.

\sphinxAtStartPar
It seems that there are many similarities between the etiquette of
enlightenment and the etiquette of sex. Both are highly charged and
delicate personal subjects. If there is a group of people, and some of
them are sexually active, and others are not, it is unlikely that there
would be an open discussion of individuals’ sex lives. A meditation
retreat might have some similarities to a sex education class, but who
is getting it and who is not would still likely be a private matter. If
you had a problem or a question about your personal sex life, you might
talk about it with a teacher or friend. If you have friends whom you
feel have had similar experiences, you might go into great detail about
your personal experiences.

\sphinxAtStartPar
In the Theravada tradition, there are four great rules which, if broken,
require monks to disrobe, and they can not be monks again. The rules
involve: taking life, stealing, sex and claiming any attainment in
meditation which they do not have. Since there is always some
uncertainty about attainments, and mistakes are common, it is very rare
that a Theravada monk will make any type of claim. This tends to carry
over as a standard for nuns and laypeople even though there are no
official sanctions against them making erroneous claims.


\subsection{Evolving etiquette}
\label{\detokenize{saints:evolving-etiquette}}
\sphinxAtStartPar
Because of the recent revival of meditation in Southeast Asia, social
customs concerning enlightenment are still evolving. In the early days
at the Mahasi Sayadaw monastery, Mahasi Sayadaw would give students a
certificate of attainment for Stream\sphinxhyphen{}winners. I have heard no
explanation as to why this practice was discontinued, but it seems that
there must have been some problems. Generally, at the Mahasi monastery
today there is no public discussion of specific individual attainments.

\sphinxAtStartPar
I have heard stories about the first teachers in other areas who would
openly discuss attainments of students. This seems to have developed
interest in people in that area to pursue the practice. After there is
sufficient public interest, it seems that the trend is for attainments
to be kept more confidential. No doubt problems will arise with people
mistaking their experiences in the practice for enlightenment, and some
people may simply start to make false claims. It seems that the natural
evolution would be that public awareness of attainments would be
inspirational at first, but later problems would make privacy
preferable.


\subsection{Silence}
\label{\detokenize{saints:silence}}
\sphinxAtStartPar
One of the ways of acknowledging attainments is silence. Frequently
people would invite the Buddha to come to their houses to receive alms
food. It was the custom of the Buddha to remain silent when he assented.
Perhaps this was his way of avoiding breaking his word if circumstances
arose which kept the visit from happening. Anyway, people knew if he
remained silent that they could expect him to show up the next day.
Perhaps this is why some monks, nuns and teachers remain silent when
someone asks them if they are enlightened.

\sphinxAtStartPar
Tungpulu Sayadaw of Burma was one of the most widely recognized Arahants
of the twentieth century. I heard the following story about how it
became known that he was an Arahant: While he was meditating in his cave
for thirty\sphinxhyphen{}nine years, he would be visited by an old friend who had
practiced with him under the guidance of the same teacher. At first, he
asked Tungpulu if he had become a \sphinxstyleemphasis{Once\sphinxhyphen{}Returner,} and Tungpulu said
“No, I have not become a \sphinxstyleemphasis{Once\sphinxhyphen{}Returner.”} After a few years, he again
asked Tungpulu if he had become a \sphinxstyleemphasis{Once\sphinxhyphen{}Returner,} and he remained
silent. Then he asked him if he had become a \sphinxstyleemphasis{Non\sphinxhyphen{}Returner,} and
Tungpulu said, “No, I have not become a \sphinxstyleemphasis{Non\sphinxhyphen{}Returner.”} After a few
more years, he asked him if he had become a \sphinxstyleemphasis{Non\sphinxhyphen{}Returner,} and
Tungpulu remained silent. Then he asked him if he had become an Arahant,
and he said “No, I have not become an Arahant.” Finally, after many
years he asked him if he had become an Arahant, and Tungpulu remained
silent. Once a woman asked him in public if he was an Arahant. Tungpulu
raised up his hand to her indicating that she should not ask that
question.


\subsection{The Bodhisattva Vow}
\label{\detokenize{saints:the-bodhisattva-vow}}
\sphinxAtStartPar
Tungpulu Sayadaw was an unusual Theravadan because he announced that he
had taken The Bodhisattva Vow. This was after he was famous and had
established many monasteries. Many traditional Theravadans concluded
that he could not have attained even Stream\sphinxhyphen{}winner because a Bodhisattva
supposedly vows to delay enlightenment. This was paradoxical because he
was such an extraordinary teacher and had many enlightened students. It
is unlikely that an unenlightened teacher would have enlightened
students.

\sphinxAtStartPar
The Mahayana Bodhisattva Vow of delaying enlightenment until all beings
are enlightened is one of the most heated disputes between the Theravada
and Mahayana Buddhists. The Mahayanas view the Theravadans as being
selfish because they regard their own liberation more important than the
liberation of others. The Theravadans take the vow literally believing
it refers to \sphinxstyleemphasis{Stream\sphinxhyphen{}winning,} and point out that becoming enlightened
is the best way to help others become enlightened. If the vow refers to
\sphinxstyleemphasis{Stream\sphinxhyphen{}winning,} and if everyone delays their enlightenment until
everyone becomes enlightened, then logically no one will ever become
enlightened. Also, it seems questionable to take a vow that one cannot
comprehend. How long will it be until all sentient beings get
enlightened? How many sentient beings will have to get enlightened
first? Another complication is that some Buddhists believe that the
number of sentient beings is increasing.

\sphinxAtStartPar
The Mahayana Buddhists have different beliefs as to exactly what the
Bodhisattva Vow means, but many believe it refers only to not going into
\sphinxstyleemphasis{Parinirvana} which is the final phase of enlightenment. There are
many carefully documented and verified cases of Tibetan Arahants that
have taken this vow and have reincarnated. This seems a contradiction to
the Buddha’s claim that Arahants will not reincarnate. According to
Buddhist psychology, rebirth is the result of craving, and Arahants are
free of craving. However, compassion is a motivating factor in Arahants,
and it seems that they have substituted compassion for craving as a
cause of rebirth.

\sphinxAtStartPar
When Tungpulu Sayadaw came to teach at the Insight Meditation Society in
1984, I thought that he might be able to shed some light on the
disagreement between the Theravada and Mahayana Buddhists. I arranged a
private interview with him, but it was delayed until late at night. When
I came into the interview, the interpreter explained that since it was
so late, I would only be allowed one question. I had intended to ask two
questions, one of which was, “Why would it be impossible for an Arahant
to take a vow to reincarnate?” I elected to ask my other question about
mind moments. Tungpulu gave a lengthy answer to my one question.
Interestingly, he also to included an answer to my second unasked
question. He said “The reincarnation of Arahants is a paradox.” I
thought that was a particularly skillful answer because he avoided the
controversy of contradicting the traditional Theravada view by directly
saying that Arahants reincarnate. At the same time, he acknowledged that
Arahants can reincarnate by saying it was a paradox. If they did not
reincarnate, there would be no paradox.


\subsection{Clues}
\label{\detokenize{saints:clues}}
\sphinxAtStartPar
People who have spent long periods of time studying with a teacher are
quite likely to have picked up various clues as to the level of
attainment of the teacher. Most of these clues come from their teacher’s
talks when he/she illustrates certain experiences in
meditation with his/her own experiences. For example, one of my teachers
said that when he was practicing with his teacher, he sat with the vow
not to arise from his seat until his mind was free of all defilements.
Knowing how committed he was to following vows, and considering that he
arose from his seat, I felt safe in concluding that he was an Arahant,
free of all mental defilements. Based on this statement I could have
legitimately challenged him to deny he was an Arahant, but that would
have been rude. Very likely, he would have pretended to ignore the fact
that I asked him a question.

\sphinxAtStartPar
If you are interested in determining the level of attainment of
teachers, it would be impolite to ask them directly. However, you might
take an opportunity to discretely ask one of their senior students for
their opinion of the level of attainment of their teacher, and why they
believe it. Remember that enlightenment is generally a delicate, private
matter, and it is as sensitive as inquiring into someone’s sex life. At
the same time, the etiquette of enlightenment is a dynamic and changing
situation, and it is possible to push the edge. Discrete conversations
with your peers and senior students of teachers are an excellent way to
pick up clues.


\section{Conclusion}
\label{\detokenize{saints:id3}}
\sphinxAtStartPar
It has become popular to remind people of an ancient Chinese curse, “May
you be born in interesting times.” Times are very interesting these days
with simultaneous revolutions in transportation, media, information
availability, as well as changes brought on by the end of the Cold War.
The teachings of all of the world’s great religions are now
simultaneously available for comparison for the first time. The
different traditions of Buddhism, which have evolved in isolation from
each other, are now simultaneously encountering each other for the first
time. There are research scientists who have not only been trained in
the Western scientific method, but who have also attained enlightenment,
and are able to do qualified research into the nature of enlightenment.
Enlightened scholars have not only been able to establish common
understanding between different Buddhist traditions, but also they have
found common threads of understanding in different types of religions.
It is a time of great ferment and change.

\sphinxAtStartPar
It is helpful to remind ourselves that living in times of change is not
only exciting, but it is also a curse. Times of spiritual change are
breeding grounds for both saints and psychopaths. Hopefully, I have
provided some basis for understanding both saints and psychopaths, and
given you some ability to distinguish which is which. There are extra
copies of the \sphinxstylestrong{Checklist for Saints and Psychopaths} on the last
pages of this book, and I suggest that you post them on a wall as a
reminder, or give them to friends. Reviewing this list from time to time
may save you and your friends a great deal of trouble in this time of
change.

\sphinxAtStartPar
I hope that the sharing of my personal spiritual journey will be of help
to others on the path of spiritual development. I was not eager to
reveal the mistakes that I have made, nor how psychopaths have taken
control of my life at times. However, the thought that honestly sharing
the mistakes that I have made may help others to avoid similar mistakes
made the decision easy.

\appendix
% move PDF bookmarks to the top leve
\bookmarksetup{startatroot}
% demote sections again, same as in frontmatter
\let\part\chapter
\let\chapter\section
\let\section\subsection
\let\subsection\subsubsection

\sphinxstepscope


\chapter{Glossary}
\label{\detokenize{glossary:glossary}}\label{\detokenize{glossary::doc}}\begin{description}
\sphinxlineitem{Anagami}
\sphinxAtStartPar
One who has attained the third level of enlightenment. The
classic definition of an \sphinxstyleemphasis{Anagami} is that he or she has eliminated
the mental defilements of lust and aversion, but still retains the five
defilements of mind: 1) craving for subtle material realms, 2) craving
for subtle immaterial realms, 3) conceit, 4) restlessness, 5) the final
vil of unknowing or delusion.

\sphinxlineitem{Arahant}
\sphinxAtStartPar
A fully enlightened being. One who has uprooted all the
defilements and experiences no more mental suffering. According to the
Theravada tradition, just before death an Arahant will pass into and
remain in Nirvana and will not be reborn in any form. According to the
Mahayana tradition, Arahants may be reborn if they have made a vow to
reincarnate to help other sentient beings attain enlightenment.

\sphinxlineitem{atta}
\sphinxAtStartPar
The Hindu concept of self as being an unchanging, immutable aspect
of the consciousness of God. Enlightenment in this tradition is regarded
as a transcendence of concepts which separate oneself from God.
Ultimately, all separate self existence is relinquished, and all that
remains is everything which is God. In both the Hindu and Buddhist
tradition individual self existence ultimately ceases when one realizes
fully that the individual self is only an illusion, and when one is able
to transcend the craving which causes rebirth.

\sphinxlineitem{Commentaries}
\sphinxAtStartPar
In the 5th century A.D. Buddhaghosa, an Indian scholar in
Sri Lanka, compiled and translated a great body of texts written
in Sinhalese and Sanskrit into Pali. The Theravadans believe Buddhaghosa
regarded the Buddha as the source of original thought, and the various
explanatory texts as fortification of the teachings of the Buddha.
Western scholars have suggested that Buddhaghosa must have simply
translated any texts that were available as they range in quality from
wonderfully insightful to trivial and erroneous. In any case, the body
of Pali texts which were compiled became the basis for making Pali the
scholarly language of the Theravada tradition in Southeast Asia.

\sphinxlineitem{contemplative}
\sphinxAtStartPar
As used in this book, \sphinxstyleemphasis{contemplative} refers to
spiritual traditions which involve mental disciplines as opposed to
disciplines which involve service, study, teaching, or work.

\sphinxlineitem{co\sphinxhyphen{}psychopath}
\sphinxAtStartPar
Co\sphinxhyphen{}psychopaths are close associates of psychopaths who
are caught up in a web of control and deception. They may be a spouse,
partner, student or disciple. They, themselves are usually not morally
defective, but they have accepted the artificial reality that the
psychopaths have created. Once they have developed some doubt about
their psychopathic associate, they have an excellent chance of
extricating themselves from their role. Usually to fully recover, they
will need support and understanding from others who have had similar
experiences.

\sphinxlineitem{deep insight}
\sphinxAtStartPar
An abbreviation of \sphinxstyleemphasis{Deep Insight into Arising and Passing
of Phenomenon.} This occurs when concentration and the ability to
observe change are developed to the point that thought processes can be
observed to arise and pass in the mind. The nature of objects of
consciousness appears to change, and a great energy and enthusiasm for
meditation practice arises in the meditator. After deep insight, insight
into faster and subtler components of the thought process is realized,
and the meditator acquires an intuitive wisdom about the nature of
consciousness and reality.

\sphinxlineitem{defilements of mind}
\sphinxAtStartPar
There are three basic defilements of the mind:
greed, hatred\sphinxhyphen{}aversion, and delusion\sphinxhyphen{}confusion\sphinxhyphen{}unknowing. There are
innumerable individual defilements which are permutations and
combinations of the basic three such as: fear, sloth, conceit, jealousy,
paranoia and resentment. The root cause of defilements is craving, or a
strong preference for things to be a certain way such as, having what
you want and not having an experience which you do not want. The mind is
defiled when there is craving that is unconscious and determines
behavior. Only Arahants are able to be conscious of the arising of the
subtlest forms of craving.

\sphinxlineitem{Dhammapada}
\sphinxAtStartPar
The \sphinxstyleemphasis{Dhammapada} was compiled in Sri Lanka about 2,000
years ago from the 100,000 page record of the Buddha’s discourses.
Occasionally during his discourses, the Buddha would make a point in
spontaneous poetry. The \sphinxstyleemphasis{Dhammapada} is an anthology of some of these
poems. Since the record of the discourses are chants of summaries, and
since the poems could not be rendered into summaries, the \sphinxstyleemphasis{Dhammapada}
contains the purest record of what the Buddha actually said.

\sphinxlineitem{dharma}
\sphinxAtStartPar
The natural law for any conditioned object. The truth. The
teachings of the Buddha. Cause and effect. The way things are. Dharma is
Sanskrit for the Pali word \sphinxstyleemphasis{dhamma.} It has a great range of
meanings depending on the context in which it is used. Usually it
applies to spiritual truths or laws. Sometimes it is used as a synonym
for karma which is one aspect of the dharma. Generally, \sphinxstyleemphasis{dharma}
refers to the natural law or the basic truth such as the law of
gravity (also a dharma), and it is not strictly speaking a law, but the
definition of a process. Dharma is not regarded in Buddhism as a law
decreed by the Buddha or any deity.

\sphinxlineitem{Eight Fold Path}
\sphinxAtStartPar
The Buddha’s basic teaching is the Eight Fold Path. It
consists of eight things which if understood and done correctly will
lead to enlightenment. They are: Right Understanding, Right Thought,
Right Speech, Right Action, Right Livelihood, Right Effort, Right
Mindfulness, Right Concentration. All of the Buddha’s teachings are
explanations and elaborations on these eight basic teachings.

\sphinxlineitem{enlightenment}
\sphinxAtStartPar
As used in this book, refers to having attained a stable
access to the first stratum of the unconscious processes and corresponds
to the level of stream\sphinxhyphen{}winner. The characteristics of clarity,
compassion and intuitive wisdom deepen as access to deeper strata are
attained. Arahants have the greatest access and are regarded as being
fully enlightened.

\sphinxlineitem{\sphinxstyleemphasis{karma}}
\sphinxAtStartPar
Karma is the cause and effect relationship of how past
actions condition our present experience. Karma is created by the
volitional intentions of action, speech and thought and the effect they
have on sentient beings. One’s reality is the result of one’s karma and
is a composite of good and bad karma. Good karma results in pleasure.
Bad karma results in suffering. Actions based on intentions to bring
pleasure result in good karma. Actions based on intentions to bring harm
bring suffering. Actions which result in good karma do not cancel out
bad karma, but they do upgrade the mix.

\sphinxlineitem{LSD}
\sphinxAtStartPar
A powerful psychedelic drug, lysergic acid diethylamide, that alters
consciousness and perception.

\sphinxlineitem{levels of enlightenment}\begin{enumerate}
\sphinxsetlistlabels{\arabic}{enumi}{enumii}{}{)}%
\item {} 
\sphinxAtStartPar
\sphinxstyleemphasis{Sotapanna}; 2) \sphinxstyleemphasis{Sakadagami}; 3) \sphinxstyleemphasis{Anagami}; 4) \sphinxstyleemphasis{Arahant.}

\end{enumerate}

\sphinxlineitem{\sphinxstyleemphasis{Mahasatipatthana}}
\sphinxAtStartPar
A long discourse by the Buddha in the \sphinxstyleemphasis{Digha
Nikayti} describing the basic principles of meditation and different
forms of meditation practice. It is traditionally regarded as being a
single discourse given in the market town of Karnmasadhamma. However, a
study of the structure and content of the text indicates it was
compilation on the subject of meditation from fragments of various
discourses. The editor’s apparent lack of skill in meditation seem to
have created some odd combinations of topics based on words instead of
how methods would be used in actual practice. It seems unlikely that a
skillful teacher such as the Buddha would arrange his topics in such a
way. It is possible that the Buddha expected that the details of
meditation practice be taught individually and orally by a qualified
teacher. This may be why his discourses contain so little specific
information on meditation technique.

\sphinxlineitem{Mahayana}
\sphinxAtStartPar
Buddhists who take the Bodhisattva Vow to delay their
enlightenment until all sentient beings attain enlightenment. The
purpose is to remain in existence so that they can help to all other
sentient beings attain enlightenment. Mahayana Buddhists have different
interpretations of what this vow means, but many regard this as not
entering \sphinxstyleemphasis{Parinirvana} where individual existence would cease. In the
Mahayana tradition it is believed that the Bodhisattva vow makes it
possible to override the limitations of rebirth that the Buddha
described for different levels of enlightenment.

\sphinxlineitem{metta}
\sphinxAtStartPar
A meditation practice of systematically cultivating feelings of
unconditional loving kindness. \sphinxstyleemphasis{Metta} is a concentration practice
sometimes used in conjunction with, but not at the same time as
\sphinxstyleemphasis{vipassana} meditation. \sphinxstyleemphasis{Metta} is a wish that all beings should be
safe, happy and healthy. Advanced practitioners experience heavenly
mental states, and sometimes have a profound positive effect on other
beings.

\sphinxlineitem{nama\sphinxhyphen{}rupa}
\sphinxAtStartPar
Mind and matter. A stage of development in meditation
practice when it becomes very clear that mind and matter are separate
but interdependent.

\sphinxlineitem{Nirvana}
\sphinxAtStartPar
Sanskrit: lit, \sphinxstyleemphasis{Extinction} (nir+va to cease blowing, to
become extinguished.) Nirvana is beyond time and space; therefore it has
neither beginning nor end—birth nor death. One can enter into Nirvana
and leave it, but Nirvana is always there and unchanging. The ultimate
objective of Buddhism is to enter Nirvana and not leave it. Some of the
synonyms the Buddha used for are Nirvana are: The Peaceful State,
Deathless, Bliss Supreme, Stilling of Conditioned Things, Unconditioned,
Uncreated, Unmade, Unoriginated, Unformed, Unborn, Uncompounded, The
Way, Highest Goal, Fruit, Dharma, Happiness Supreme, That Beyond
Happiness, Destruction of Conditioned Things, Gone Beyond. Nirvana is
experienced briefly at the culmination of each path at the time that
enlightenment or higher level of enlightenment is attained. Since there
are no reference points in Nirvana to base a description
on, any description of Nirvana is not a description of
Nirvana.

\sphinxlineitem{Non\sphinxhyphen{}returner}
\sphinxAtStartPar
An Anagami. People who have attained the third level of
enlightenment The Buddha gave them the name \sphinxstyleemphasis{Non\sphinxhyphen{}returners} because
they will not be reborn again in this world as a human. They will
experience one more birth in a formless heavenly realm of gods before
they go into \sphinxstyleemphasis{Parinirvana.}

\sphinxlineitem{Pali}
\sphinxAtStartPar
Summaries of the Buddha’s discourses were rendered into chants, and
for two hundred years after the death of the Buddha, nearly the only
record of what the Buddha said, was this oral tradition. Then King Asoka
sponsored the writing of the Buddha’s discourses into Pali, the language
of his kingdom. Northern India was ruled by the kingdom of Magadha
during the Buddha’s lifetime, so it seems reasonable that the Buddha
spoke Magadhi in public, and possibly at times spoke Sanskrit with
scholars. Both Magadhi and Pali have very near roots in Sanskrit. The
similarity of the three languages must have the advantage of
transmitting subtle understanding as well as subtle misunderstandings.
Traditionally, the Theravada view is that Pali is an absolutely accurate
record of the Buddha’s discourses, and the Pali Commentaries are
accurate elaborations of the Buddha’s teachings.

\sphinxlineitem{parinirvana}
\sphinxAtStartPar
Immediately before physical death, an Arahant’s
consciousness passes out of conditioned existence and enters into
Nirvana, and will not reenter conditioned existence.

\sphinxlineitem{progress of insight}
\sphinxAtStartPar
The sequence of insights leading up to
enlightenment and immediately after enlightenment.

\sphinxlineitem{psychedelic}
\sphinxAtStartPar
Of alterations in consciousness or perceptions. From the
Greek psyche (mind) + delos (clear). Psychedelic drugs have an effect of
creating a sense that the mind is seeing clearer and consciousness is
expanded. In larger doses, they have an increased effect of distorting
perceptions, confusing the mind and causing hallucinations. In some
cases, psychedelic drugs have induced a profound examination of the
present moment, and have resulted in \sphinxstyleemphasis{deep insight.}

\sphinxlineitem{Saint}
\sphinxAtStartPar
A person considered to be holy and worthy of veneration. As used
in this book, it is someone who has purified their consciousness of
defilements and is enlightened.

\sphinxlineitem{Sakadagami}
\sphinxAtStartPar
Also known as a Once\sphinxhyphen{}returner. Someone who has reached the
second level of enlightenment. Traditionally, a \sphinxstyleemphasis{Sakadagami} is said
to have reduced the defilements of mind of lust and aversion. A
\sphinxstyleemphasis{Sakadagami} has greater clarity of mind and a deeper intuitive wisdom
than a stream\sphinxhyphen{}winner.

\sphinxlineitem{shaktipat}
\sphinxAtStartPar
The transmission of psychic energy from teacher to student
that results in a variety of phenomenon, such as deep trance, visions,
energy sensations and uncontrollable body motions.

\sphinxlineitem{Samatha}
\sphinxAtStartPar
Pure concentration meditation practice. \sphinxstyleemphasis{Samatha} meditation
is essentially a form of self\sphinxhyphen{}hypnosis. The mind is focused on a fixed
object which seems more solid, real and unchanging as concentration
deepens. As one is able to focus the mind precisely on the object
of concentration with less wavering to other objects or thoughts,
specific levels of concentration called \sphinxstyleemphasis{jhanas} are attained. Any
sense object—sight, sound, smell, taste, touch or thought—can be used as
an object of \sphinxstyleemphasis{samatha} meditation. Sensations of breathing are
frequently used as objects of meditation in \sphinxstyleemphasis{vipassana} practice
because the breath is a changing object. \sphinxstyleemphasis{Samatha} is an essential
component to \sphinxstyleemphasis{vipassana} and is developed first. True \sphinxstyleemphasis{vipassana}
practice begins to develop when the level of \sphinxstyleemphasis{nama\sphinxhyphen{}rupa} is
reached. \sphinxstyleemphasis{Vipassana} deconditions the hypnotic state of mind which
results in an illusory view of reality. One of the reasons that
vipassana meditation is so difficult is that \sphinxstyleemphasis{samatha} is constantly
being destroyed by \sphinxstyleemphasis{vipassana.}

\sphinxlineitem{Self}
\sphinxAtStartPar
In the Hindu tradition, the true self is the \sphinxstyleemphasis{atta} which is an
unchanging aspect of the consciousness of God. The objective of Hindu
practice is to come to the full realization that the \sphinxstyleemphasis{atta} is the
true self and the relinquishment of any sense that you are anything else
but the \sphinxstyleemphasis{atta.} Most Westerners do not have this concept of \sphinxstyleemphasis{atta},
but because of unawareness, there is a natural tendency to view the
self as being in control and unchanging. As \sphinxstyleemphasis{vipassana} meditation
develops, the meditator comes to see that the self is the result of
cause and effect processes, and has no solid continuous existence.

\sphinxlineitem{sotapanna}
\sphinxAtStartPar
A Stream\sphinxhyphen{}winner.

\sphinxlineitem{sutta}
\sphinxAtStartPar
Lit. a seat. The Buddha would sit and give a discourse, and after
it, a senior monk would compose a chant which was a summary of what the
Buddha had said. Monks and nuns still spend a considerable amount of
time reciting these chants in groups. The chanting is a combination of
concentration meditation and education

\sphinxlineitem{Stream\sphinxhyphen{}winner}
\sphinxAtStartPar
One who has attained the first level of enlightenment.
Defilements uprooted are: Doubt that the Eightfold Path will lead to
total purification of the mind, belief that rites and rituals will
result in enlightenment, and belief in the usual sense of self. In this
book I define a \sphinxstyleemphasis{Stream\sphinxhyphen{}winner} as one who has permanently accessed
the first stratum of the unconscious mind.

\sphinxlineitem{Theravada}
\sphinxAtStartPar
Thera means elder, and vada is the way of, so it is literally
the way of the elders. The Theravadans regard the Pali record of the
Buddha’s discourses, which were written two hundred years after the
Buddha, and the Pali Commentaries, compiled by Buddhaghosa a thousand
years after the time of the Buddha, as the only valid version of the
Buddha’s teaching. The roots of the Theravada tradition comes from King
Asoka who sponsored the writing of the Buddha’s discourses in Pali.
Since Asoka richly patronized Buddhist monasteries, it became necessary
to clearly define which traditions were true Buddhists and which were
false. The Theravada tradition existed along with other traditions of
Buddhism in India for a thousand years, but all forms of Buddhism were
eliminated in India by Muslim invaders. However, King Asoka’s son, who
was a monk and his daughter, who was a nun, established the Theravada
tradition in Sri Lanka which survived and has spread throughout
Southeast Asia.

\sphinxlineitem{\sphinxstyleemphasis{Vipassana}}
\sphinxAtStartPar
To see clearly with great penetration and effort. When
objects of consciousness are seen with enough penetrating awareness,
they will be seen as being in a continuous state of change, because
consciousness is continuously changing.

\sphinxlineitem{Zen}
\sphinxAtStartPar
A branch of Mahayana Buddhism which was brought into China by
Bodhidarma. From China it spread into Korea and Japan. Zen evolved and
became deeply integrated with Chinese culture before it acquired many of
the texts and traditions of older forms of Buddhism. A significant
difference is that Zen monks follow a different set of rules than the
ones prescribed by the Buddha for monks. The emperors of China placed
great importance on acquiring more of the Buddha’s discourses and
rewarded people who acquired them. Some of the books of discourses
acquired have similar titles but different contents than older versions.

\end{description}

\sphinxstepscope


\chapter{Checklist for Saints and Psychopaths}
\label{\detokenize{checklist:checklist-for-saints-and-psychopaths}}\label{\detokenize{checklist::doc}}

\section{Psychopaths}
\label{\detokenize{checklist:psychopaths}}
\sphinxAtStartPar
When confronted with wrong\sphinxhyphen{}doing, psychopaths will typically respond in this sequence:
\begin{enumerate}
\sphinxsetlistlabels{\arabic}{enumi}{enumii}{}{)}%
\item {} 
\sphinxAtStartPar
Ignore the issue

\item {} 
\sphinxAtStartPar
Deny that they have done something wrong

\item {} 
\sphinxAtStartPar
Attack the accuser, usually accusing the accuser of being the one who has done wrong

\item {} 
\sphinxAtStartPar
Threaten to harm the accuser, someone else, something, or self

\item {} 
\sphinxAtStartPar
Apologize and admit that they have done wrong, and ask for a clean slate or new start

\end{enumerate}

\sphinxAtStartPar
Psychopaths will resist completing this five step sequence of responses, and will complete it only if confrontation persists.


\section{Saints}
\label{\detokenize{checklist:saints}}
\sphinxAtStartPar
When confronted with wrong\sphinxhyphen{}doing, saints will typically respond in this sequence:
\begin{enumerate}
\sphinxsetlistlabels{\arabic}{enumi}{enumii}{}{)}%
\item {} 
\sphinxAtStartPar
Acknowledge errors and misunderstandings

\item {} 
\sphinxAtStartPar
Admit that they have made an error

\item {} 
\sphinxAtStartPar
Apologize

\item {} 
\sphinxAtStartPar
Offer compensation or correction

\item {} 
\sphinxAtStartPar
Avoid that type of error or misunderstanding in the future

\end{enumerate}

\sphinxAtStartPar
Saints are inclined to complete this five step sequence without continued confrontation.

\sphinxAtStartPar
Allow for personality characteristics and cultural factors when evaluating responses to wrong\sphinxhyphen{}doing.

\bgroup\footnotesize


\begin{savenotes}\sphinxattablestart
\sphinxthistablewithglobalstyle
\centering
\begin{tabulary}{\linewidth}[t]{TT}
\sphinxtoprule
\sphinxstyletheadfamily 
\sphinxAtStartPar
\sphinxstylestrong{SAINTS}
&\sphinxstyletheadfamily 
\sphinxAtStartPar
\sphinxstylestrong{PSYCHOPATHS}
\\
\sphinxmidrule
\sphinxtableatstartofbodyhook
\sphinxAtStartPar
SAY MEAN DO consistency
&
\sphinxAtStartPar
SAY MEAN DO disparity
\\
\sphinxhline
\sphinxAtStartPar
Adhere to own moral standards
&
\sphinxAtStartPar
Breaks own rules
\\
\sphinxhline
\sphinxAtStartPar
Pay debts
&
\sphinxAtStartPar
Many bad debts, writes bad checks
\\
\sphinxhline
\sphinxAtStartPar
Keep promises
&
\sphinxAtStartPar
Break promises
\\
\sphinxhline
\sphinxAtStartPar
Truth is highest standard
&
\sphinxAtStartPar
No true regard for truth
\\
\sphinxhline
\sphinxAtStartPar
Insists dose associates tell the truth
&
\sphinxAtStartPar
Tell close associates to lie
\\
\sphinxhline
\sphinxAtStartPar
Un\sphinxhyphen{}aggressive philosophy
&
\sphinxAtStartPar
Push philosophy aggressively
\\
\sphinxhline
\sphinxAtStartPar
Attractive but not drawing
&
\sphinxAtStartPar
Attractive and drawing
\\
\sphinxhline
\sphinxAtStartPar
Waits for you to seek help
&
\sphinxAtStartPar
Comes on with unsolicited advice
\\
\sphinxhline
\sphinxAtStartPar
Good reputation endures \& improves
&
\sphinxAtStartPar
Good reputation fades in time
\\
\sphinxhline
\sphinxAtStartPar
Projects \& organization grow \& improve
&
\sphinxAtStartPar
Projects \& organization degenerate
\\
\sphinxhline
\sphinxAtStartPar
In the long run things turn out well
&
\sphinxAtStartPar
In the long run things turn out badly
\\
\sphinxhline
\sphinxAtStartPar
People have long term benefit from association
&
\sphinxAtStartPar
People are damaged by long term association
\\
\sphinxhline
\sphinxAtStartPar
Have concern for effect of actions on self and others
&
\sphinxAtStartPar
Are unconcerned for effect of actions on self and others
\\
\sphinxhline
\sphinxAtStartPar
Will immediately apologize for errors
&
\sphinxAtStartPar
Apologize as last resort
\\
\sphinxhline
\sphinxAtStartPar
Look for their own mistakes \& will apologize
&
\sphinxAtStartPar
Ignore their own mistakes and apologizes only if cornered
\\
\sphinxhline
\sphinxAtStartPar
If trapped will not renounce principles
&
\sphinxAtStartPar
If trapped will do or say anything to escape
\\
\sphinxhline
\sphinxAtStartPar
Typically have good health
&
\sphinxAtStartPar
Typically have variable exotic health problems
\\
\sphinxhline
\sphinxAtStartPar
Typically have few accidents \& injuries
&
\sphinxAtStartPar
Typically have many accidents and injuries
\\
\sphinxhline
\sphinxAtStartPar
Felt loved when a child
&
\sphinxAtStartPar
Felt unloved when a child
\\
\sphinxhline
\sphinxAtStartPar
Can sit very still
&
\sphinxAtStartPar
Can sit still only when center of attention
\\
\sphinxhline
\sphinxAtStartPar
Encourage associates to be self reliant
&
\sphinxAtStartPar
Enslave people around them
\\
\sphinxhline
\sphinxAtStartPar
Refrains from using mind\sphinxhyphen{}dulling substances
&
\sphinxAtStartPar
Substance abuse common
\\
\sphinxhline
\sphinxAtStartPar
Are comfortable being in the background
&
\sphinxAtStartPar
Compulsion to become the center of attention
\\
\sphinxhline
\sphinxAtStartPar
May adopt a spiritual name one time
&
\sphinxAtStartPar
Adopt many aliases
\\
\sphinxhline\sphinxstartmulticolumn{2}%
\begin{varwidth}[t]{\sphinxcolwidth{2}{2}}
\sphinxAtStartPar
\sphinxstyleemphasis{Any one psychopath or saint is unlikely to have all of the characteristics listed. Just because someone has
some of these characteristics does not mean he or she is a psychopath or saint.}
\par
\vskip-\baselineskip\vbox{\hbox{\strut}}\end{varwidth}%
\sphinxstopmulticolumn
\\
\sphinxbottomrule
\end{tabulary}
\sphinxtableafterendhook\par
\sphinxattableend\end{savenotes}

\egroup



\renewcommand{\indexname}{Index}
\printindex
\end{document}